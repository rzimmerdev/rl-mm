\section{Methodology}

\subsection{Problem Definition}
The market-making problem addressed in this work involves designing an optimal trading policy for an agent using reinforcement learning (RL). The agent aims to maximize profit while managing risks, particularly inventory risk. The market dynamics are modeled by the limit order book (LOB) and its dynamics, which define how the book evolves over time based on order flow and price movements. Our agent interacts with this environment by quoting bid and ask prices and adjusting offered quantities. As discussed previously, the main challenge for choosing an adequate agent and its policy lies in balancing profitability with risk management, especially regarding inventory at the close of the market, where overnight positions can expose the agent to significant risks.

\subsection{Formal Description of the RL Environment} In modeling the RL environment, we initially utilize a continuous-time, continuous-state Markov Chain framework, and later transition to a discrete representation to address computational space constraints. We choose a state space that tries to best incorporate the historical events of the limit order book into a single observable state using commonly used indicators and LOB levels, as well as intrinsic features to the agent as proposed by <multiple references, inserir paper com review de RL para MM>. Given our performed bibliographical research, we chose the agent's current inventory for the intrinsic feature and a set of indicators for the extrinsic features: the Relative Strength Index (RSI); order imbalance (O); and micro price (MP). Additionally, for a fixed number $D$ of LOB price levels the pair $(\delta^d, Q^d)$, where $\delta^d$ is the half-spread distance for the level $d \leq D$, and $Q^d$ the amount of orders posted at that level is added to the state as a set of tuples, for both the ask and bid sides of the book. The state space can therefore be formally expressed as:
$$
X_t = (I_t, \, RSI_t, \, O_t, \, \text{MP}_t, \, \{ (\delta_t^{d, \text{ask}}, Q_t^{d, \text{ask}}) \}_{d=1}^D, \, \{ (\delta_t^{d, \text{bid}}, Q_t^{d, \text{bid}}) \}_{d=1}^D)
$$
As mentioned, the evolution is continuous in time, meaning state changes occur at any point in continuous time, and the next event occurs after a sampled waiting time. The specific case in which a Markov Chain also has an associated reward distribution $R$ for each state transition is called a Markov Reward Process and given that the MM problem also has a decision process that affects the transition probabilities it is therefore called a Markov Decision Process in control literature, and is generically defined as a 4-tuple $ (\mathcal{S}, \mathcal{A}, \mathbb{P}, R) $, where:

\begin{itemize}
	\item $\mathcal{S}$ is a set of states called the state space.
	\item $\mathcal{A}$ is a set of actions called the action space.
	\item $P: \mathcal{S} \times \mathcal{A} \times \mathcal{S} \to [0, 1]$ is the transition probability function for the MDP.
	\item $R: \mathcal{S} \times \mathcal{A} \times \mathcal{S}' \rightarrow \mathbb{R}$ is the reward function associated with each state transition.
\end{itemize}

Furthermore, the decision process will be defined in detail, as well as its effects on the choice of actions and the transition probability.

\subsection{State Transition Distribution and Environment Evolution Dynamics}

The previously mentioned transition probability density $P$ is given by a Stochastic Differential Equation expressed by the Kolmogorov forward equation for Markov Decission Processes:

\begin{equation}
	\frac{\partial P(s', t | s, a)}{\partial t}  = \int_{\mathcal{S}} \mathcal{L}(s' | a, t) P(x| s, a, t) dx
\end{equation}

for all $s, s' \in \mathcal{S}$ and all times $t$ before market end $T$, that is, $t \le T$, where $a$ is choosen by our control agent according to a policy $\pi (s)$. $\mathcal{L}$ is the generator operator and governs the dynamics of the state transitions given the current time. It's expression is obtained by deriving the closed-form expression for the dynamics of the underlying model, and for simple models such as those proposed by <avellaneda & stoikov, gueant>, the closed-form expression has been calculated. The underlying model for this paper is more complex, and solving for the respective generator operator is outside the scope of this paper, and a numerical approximation will be used furthermore when we define the models for the Proximal Policy Optimization (PPO) and Advantage Actor Critic (A2C) in Section 5 <inserir link>.

\subsubsection{Chosen State Space}
In continuous-time and continuous-state MDPs, the state $S$ evolves as a continuous-time stochastic process with dynamics reflected by $\mathcal{L}$ and - as mentioned - modern approaches to Reinforcement Learning require solving either numerically or approximating its transition probabilities. For our choosen market simulation model, we first formally define each component of our proposed state space:

\begin{itemize}
	$$
	\mathbf{S}_t = \left( \text{RSI}_t, \text{OI}_t, P_{\text{micro},t}, \Delta P_{1,t}, \Delta P_{2,t}, \dots, \Delta P_{d,t} \right) \in \mathbb{R}^3 \times \mathbb{R}^{2d}.
	$$
	
	The components of the state space are defined as follows:
	
	- \textbf{Order Imbalance (OI):} Order imbalance measures the relative difference between buy and sell orders at a given time. It is defined as:
	$$
	\text{OI}_t = \frac{Q_t^{\text{bid}} - Q_t^{\text{ask}}}{Q_t^{\text{bid}} + Q_t^{\text{ask}}},
	$$
	where \( Q_t^{\text{bid}} \) and \( Q_t^{\text{ask}} \) represent the total bid and ask quantities at time \( t \), respectively. \( \text{OI}_t \in [-1, 1] \), with \( \text{OI}_t = 1 \) indicating complete dominance of bid orders, and \( \text{OI}_t = -1 \) indicating ask order dominance.
	
	- \textbf{Relative Strength Index (RSI):} The RSI is a momentum indicator that compares the magnitude of recent gains to recent losses to evaluate overbought or oversold conditions. It is given by:
	$$
	\text{RSI}_t = 100 - \frac{100}{1 + \frac{\text{Average Gain}}{\text{Average Loss}}},
	$$
	where the \textit{Average Gain} and \textit{Average Loss} are computed over a rolling window (commonly 14 periods). Gains are the price increases during that window, while losses are the price decreases.

	- \textbf{Micro Price (\( P_{\text{micro}} \)):} The micro price is a weighted average of the best bid and ask prices, weighted by their respective quantities:
	$$
	P_{\text{micro},t} = \frac{P_t^{\text{ask}} Q_t^{\text{bid}} + P_t^{\text{bid}} Q_t^{\text{ask}}}{Q_t^{\text{bid}} + Q_t^{\text{ask}}},
	$$
	where \( P_t^{\text{ask}} \) and \( P_t^{\text{bid}} \) represent the best ask and bid prices at time \( t \).
	$$
\end{itemize}

\subsubsection{Chosen Action Space}

The control, or agent, interacts with the environment choosing actions from the set of possible actions, such that $a \in \mathbf{A}$ in response to observed states $s \in \mathbf{S}$ according to a policy $\pi (s, a)$ which we define shortly, and the goal is to maximize cumulative rewards over time. The system's dynamics depend on the agent's chosen action, so as to introduce features of market impact into our model, and the transition probabilities between states therefore depend on the agent's actions.

The action space $\mathcal{A}_t$ includes the decisions made by the agent at time $t$, specifically the desired bid and ask spreads pair $\delta_t^{\text{ask}}, \delta_t^{\text{bid}}$ and the corresponding posted order quantities $Q_t^{\text{ask}}, Q_t^{\text{bid}}$. Formally:
$$
\mathbf{A}_t = \left( \delta_t^{\text{ask}}, \delta_t^{\text{bid}}, q_t^{\text{ask}}, q_t^{\text{bid}} \right) \in \mathbb{R}^2 \times \mathbb{Z}^2.
$$

\subsubsection{Episodic Reward Function and Returns}

The episode reward function $r_t \in \mathbb{R}$ reflects the agent's profit and inventory risk obtained during a specific time in a past episode, it's value is given by the immediate reward function $R$, and differently from the instantaneous reward function it is discrete in time, so as to match observed event times (order arrivals and transactions occured). It depends on the spread and executed quantities, as well as the inventory cost and was choosen according to commonly used reward structures taken from the literature review.

The overall objective is to maximize cumulative utility while minimizing risk associated with inventory positions, and later insert restrictions so the risk for inventory is either limited at zero at market close, or incurring in larger penalties on the received rewards. For our model the utility chosen is based on a running Profit and Loss (PnL) score while still managing inventory risk. The choosen reward function is based on a risk-aversion enforced utility function, specifically the \textit{constant absolute risk aversion (CARA)}:

The PnL at time $t$ is computed as:
$$
\text{PnL}_t = \delta_t^{\text{ask}} q_t^{\text{ask}} - \delta_t^{\text{bid}} q_t^{\text{bid}} + \text{I}_t \cdot \Delta M_t.
$$

The agent starting penalty for holding large inventory positions  is discounted from the \textit{PnL} score, as follows:
$$
\text{Penalty}_t = \eta \left( \text{Inventory}_t \cdot \Delta M_t \right)^+,
$$
$$
\text{PnL}_t := \text{PnL}_t - \text{Penalty}_t
$$
where \( \eta \) is the penalty factor applied to positive inventory changes.

$$
R_t = U(\text{PnL}_t) = -e^{-\gamma \cdot \text{PnL}_t},
$$
where \( \gamma \) is the risk aversion parameter.

<inserir explicacao do para que serve o return>

\begin{equation}
G(\tau) = \int_0^T e^{-\gamma t} R(s_{t+dt}, s_t, a_t) \, dt
\end{equation}

\subsection{Model Description and Environment Dynamics}

For our model of the limit order book the timing of events follows a \textit{Hawkes process} so as to represent a continuous-time MDP that captures the usual observed pattern of clustered order arrival times.

The Hawkes process is a \textit{self-exciting process}, where the intensity \( \lambda(t) \) depends on past events. Formally, the intensity \( \lambda(t) \) evolves as:
$$
\lambda(t) = \mu + \sum_{t_i < t} \phi(t - t_i),
$$
where \( \mu > 0 \) is the baseline intensity, and \( \phi(t - t_i) \) is the \textit{kernel function} that governs the decay of influence from past events \( t_i \). A common choice for \( \phi \) is an exponential decay:
$$
\phi(t - t_i) = \alpha e^{-\beta (t - t_i)},
$$
where \( \alpha \) controls the magnitude of the self-excitation and \( \beta \) controls the rate of decay.

The bid and ask prices for each new order are modeled by two separate \textit{Ornstein-Uhlenbeck (OU) processes} to capture the mean-reversion behavior of spreads over the midprice:
$$
ds_t = \theta(\mu - s_t) dt + \sigma dW_t,
$$
where \( s_t \) is the market spread at time \( t \), \( \theta \) is the rate of mean reversion, \( \mu_{spread} \) is the long-term spread mean and \( \sigma \) its volatility. The Wiener process \( W_t \) is used to represent random market fluctuations.

The bid and ask spreads \( \delta_t^{\text{bid}} \) and \( \delta_t^{\text{ask}} \) for orders conditioned on their arrival follow normal distributions:
$$
\delta_t^{\text{ask}} \sim \mathcal{N}(\mu + M_{t-1} + s_t, \sigma^2), \quad a_t^{\text{bid}} \sim \mathcal{N}(\mu + M_{t-1} - s_t, \sigma^2),
$$
where \( \mu_{\text{ask}} \) and \( \mu_{\text{bid}} \) are the respective means and \( \Delta a_t \) is the spread adjustment. Whenever a new limit order that narrows the bid-ask spread or a market order arrive the mid-price is updated to reflect the top-of-book orders. The mid-price \( M \) at time $t+1$ is then obtained by averaging the maximum and minimum bid and ask prices in the book:

$$
M_{t+1} = \frac{2M_t + \delta^{ask}_{t} - \delta^{bid}_{t}}{2} 
\sim \mathcal{N}(\mu + M_{t}, \frac{\sigma^2}{2})
$$ and at $t = 0$, the midprice is defined according to some starting point of the model, usually an observed historical price as is chosen for our simulation as well.

The average of the top of book bid and ask prices therefore evolves according to a \textit{Brownian Motion} process:
$$
dM_t = \mu dt + \frac{\sigma^2}{2} dW_t,
$$
where \( W_t \) is a Wiener process and since the midprice is given by the sum of the top of book ask and bid prices, orders that cross the spread are usually rare and reflect a common stylized fact of LOBs in the market <inserir referencia book HFT>.

The order quantities \( q_t^{\text{ask}} \) and \( q_t^{\text{bid}} \) are modeled as Poisson random variables:
$$
q_t^{\text{ask}}, q_t^{\text{bid}} \sim \text{Poisson}(\lambda_q),
$$
where \( \lambda_q \) is the average order size.

\subsection{Decision Process and Steps to Maximize the Agent's Objective}

To express the Bellman equation for continuous-time MDPs, we consider the value function $V(s', a, s), which represents the expected total reward from state \( X \) under the optimal policy \( \pi^* \). The Bellman equation for a continuous-time MDP is:

<nao sei o que nao sei o que la> - pegar da proposta II enviada à FAPESP.
