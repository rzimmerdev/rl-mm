\documentclass{article}

\usepackage{arxiv}

\usepackage{amsmath,amssymb,amsfonts}
\usepackage{algorithm}
\usepackage{algorithmicx}
\usepackage{algpseudocode}
\usepackage{cite}
\usepackage{epsfig}
\usepackage{float}
\usepackage{graphicx}
\usepackage{hyperref}
\usepackage{mathtools}
\usepackage{natbib}
\usepackage{textcomp}
\usepackage{pgf}
\usepackage{subcaption}
\usepackage{siunitx}
\usepackage{tikz}
\usepackage{xcolor}
\usetikzlibrary{shapes.geometric, arrows, positioning}

%\usepackage[left=54pt, right=54pt, bottom=54pt]{geometry}

%\geometry{papersize={8.5in,11in}, top=64pt} % First-page top margin
%\usepackage{fancyhdr}
%
%\pagestyle{fancy}
%\fancyhf{}
%\renewcommand{\headrulewidth}{0pt} % Remove header rule
%
%% Adjust top margin for subsequent pages
%\usepackage{afterpage}
%\afterpage{\newgeometry{left=54pt, right=54pt, bottom=54pt, top=54pt}}
%
\setlength{\abovedisplayskip}{.5cm}  % Reduce space above equations
\setlength{\belowdisplayskip}{.5cm}  % Reduce space below equations
\setlength{\abovedisplayshortskip}{.5cm}  % For short equations
\setlength{\belowdisplayshortskip}{.5cm}  % For short equations
%\pdfminorversion=4

\usepackage{enumitem}
\setlist[itemize]{left=0pt,labelsep=5pt}

\newcommand{\Title}{Reinforcement Learning-Based Market Making as a Stochastic Control on Non-Stationary Limit Order Book Dynamics}

\IEEEoverridecommandlockouts
\title{
    \LARGE
    \bfseries
    Reinforcement Learning-Based Market Making as a Stochastic Control on Non-Stationary Dynamics
    \thanks{This work was supported by the Foundation for Research of the State of São Paulo (FAPESP) under grant 2023/16028-3.}
}

\author{
    \IEEEauthorblockN{\href{https://orcid.org/0009-0008-6064-9895}{Rafael Zimmer \includegraphics[scale=0.06]{images/orcid}} and \href{https://orcid.org/0000-0001-5989-7287}{Oswaldo Luiz Do Valle Costa \includegraphics[scale=0.06]{images/orcid}}, Senior Member, IEEE}
    \thanks{Rafael Zimmer is with the Institute of Mathematics and Computer Sciences, University of São Paulo, São Paulo, Brazil \href{mailto:rafael.zimmer@usp.br}{\tt\small rafael.zimmer@usp.br}}
    \thanks{Oswaldo Luiz Do Valle Costa is with the Polytechnic School of the University of São Paulo, São Paulo, Brazil \href{mailto:mailto:oswaldo.costa@usp.br}{\tt\small oswaldo.costa@usp.br}}
}

\maketitle

\begin{abstract}
    Reinforcement Learning has emerged as a promising framework for developing adaptive and data-driven strategies,
    enabling market makers to optimize decision-making policies based on interactions with the limit order book environment.
    Market-making strategies can be expressed as closed-form solutions or sets of heuristic rules,
    and state-of-the-art approaches use machine learning techniques to train agents on historical data or against adversarial market agents,
    but these approaches often rely on static environments or simplified market models, and may not capture the full complexity of changing market dynamics.
    This paper explores the integration of a reinforcement learning agent in a market-making context,
    where the underlying market dynamics have been explicitly modeled as stochastic processes and combined to capture observed stylized facts of real markets,
    including clustered order arrival times, non-stationary spreads and return drifts, stochastic order quantities and price volatility.

    Our contributions include a limit order book simulator with fully parameterizable and independently distributed dynamics,
    combined to model non-stationary market conditions and adverse market regimes.
    We compare the performance of a custom reinforcement learning agent based on the Proximal-Policy Optimization (PPO) algorithm
    against additional benchmarks under our proposed simulated environment.
    The results obtained suggest complex stochastic environments can serve as a valuable tool for training RL agents under markets with non-stationary regimes.
\end{abstract}

\begin{IEEEkeywords}
    Reinforcement Learning, Market Making, Limit Order Book, Stochastic Control, Market Microstructure
\end{IEEEkeywords}


\renewcommand{\undertitle}{}
\renewcommand{\shorttitle}{}
\renewcommand{\headeright}{}

\hypersetup{
pdftitle={A template for the arxiv style},
pdfsubject={q-bio.NC, q-bio.QM},  % https://arxiv.org/category_taxonomy
pdfauthor={Rafael Zimmer, Elias D.~Striatum},
pdfkeywords={First keyword, Second keyword, More},
}
%%% check with `pdfinfo preprint.pdf`

\begin{document}
    \twocolumn[
    \maketitle
    \vspace{-3em}
    \begin{abstract}
        Reinforcement Learning has emerged as a promising framework for developing adaptive and data-driven strategies,
        enabling market makers to optimize decision-making policies based on interactions with the limit order book environment.
        This paper explores the integration of a reinforcement learning agent in a market-making context,
        where the underlying market dynamics have been explicitly modeled to capture observed stylized facts of real markets,
        including clustered order arrival times, non-stationary spreads and return drifts, stochastic order quantities and price volatility.
        These mechanisms aim to enhance stability of the resulting control agent,
        and serve to incorporate domain-specific knowledge into the agent policy learning process.
        Our contributions include a practical implementation of a market making agent based on the Proximal-Policy Optimization (PPO) algorithm,
        alongside a comparative evaluation of the agent's performance under varying market conditions via a simulator-based environment.
        As evidenced by our analysis of the financial return and risk metrics when compared to a closed-form optimal solution,
        our results suggest that the reinforcement learning agent can effectively be used under non-stationary market conditions,
        and that the proposed simulator-based environment can serve as a valuable tool for training and
        pre-training reinforcement learning agents in market-making scenarios.
    \end{abstract}
    \keywords{Reinforcement Learning \and Market Making \and Limit Order Book \and Stochastic Control \and Market Microstructure}
    \vspace{2em}
    ]

    Há um crescente foco na literatura da computação financeira para com estratégias intra-diárias de \textit{market-making} \citep{Gueant2017}, que consistem em criar ordens de limite buscando lucrar em cima da diferença entre o preço de bid (Melhor preço de compra) e o preço de ask (Melhor preço de venda), chamada de Bid-Ask-Spread (\textit{BAS}). Por serem estratégias intra-diárias, utilizam técnicas de gerenciamento de risco estritamente voltadas para os horários em que o mercado está aberto e permite a criação de novas ordens. Durante esses horários, um agente de \textit{market-making} (\textit{MM}) pode ofertar simultaneamente ordens limite de compra e venda, sob o risco de poder não ter uma ordem executada até o período de fechamento do mercado \citep{Gu_ant_2012} \citep{selser2021optimal} \citep{bakshaev2020marketmaking}.

Se o agente não tem suas ordens finalizadas, ou escolhe manter um inventário após o fechamento do pregão, estará se sujeitando ao risco associado à posição durante a noite, chamado também de risco noturno, ou \textit{overnight}. Esse tipo de risco advém do fato de que o mercado não processa nenhuma ordem existente durante a noite até sua abertura no próximo dia. Logo, qualquer posição \textit{overnight} está vulnerável a eventos inesperados durante esse período, como notícias financeiras, mudanças macroeconômicas e outros que possam resultar em aberturas de mercado voláteis potencialmente desfavoráveis à posição mantida. Um agente capaz de \textbf{maximizar o retorno das operações diárias} e \textbf{minimizar o risco associado à posição noturna} seria portanto uma contribuição crítica para estratégias de \textit{MM} existentes usadas por fundos de investimento, corretoras e bancos.

Com o objetivo de obter uma política de criação de ofertas condicionada a minimizar o risco \textit{overnight} teremos como primeiro desafio determinar uma metodologia para obter dados financeiros e validar nossas hipóteses sobre o ambiente de simulação. É necessário ter em mente que as ordens criadas geram mudanças no estado do mercado e conseguintemente alteram a amostra histórica. Uma amostragem não estática impede o uso de técnicas para \textit{backtesting} regulares, pois um agente de MM não é um consumidor de preços (\textit{price-taker} em inglês), mas sim um fornecedor de preços (\textit{price-maker}). Em um cenário real, não é ideal desconsiderar o impacto das interações do agente com o mercado: um agente de \textit{MM} que esteja atuando no mercado tentará sempre manter suas ofertas no melhor patamar de preço possível, de modo a ter suas ordens executadas primeiro. Assim, o envio de novas ofertas afetará as ordens subsequentes deste outro agente.

Outro detalhe importante sobre o ambiente é a frequência dos processos de chegada das ordens. Uma nova oferta que ultrapasse o melhor preço do disponível no mercado causa quase de imediato (a depender da corretora) a atualização do livro de ordens. O tempo de chegada irá afetar muito mais a política da estratégia ótima obtida quando comparado com dados usados para estratégias de longo prazo ou de \textit{price-taking}.

Em suma, tem-se duas etapas adicionais a serem consideradas em conjunto do problema principal de conceitualizar e treinar um agente: 
\begin{itemize}
    \item uso de dados históricos \underline{estáticos} para simulação de um ambiente dinâmico;
    \item tratamento de dados com \underline{alta frequência} que possam influenciar a política obtida;
\end{itemize}

Como conclusão da pesquisa, esperamos obter um agente que execute uma estratégia de \textit{MM} que maximize o valor esperado do retorno diário em múltiplos ativos e que se adapte aos processos estocásticos de chegada de ofertas e transações. Iremos impor a restrição de um risco noturno máximo associado ao inventário do agente. 


    \section{Bibliography Review}
\label{sec:bibliography-review}

A bibliographical research was performed to determine the state of the art in the field of the project, both in terms of the models and algorithms used as well as the chosen and best-performing state and action spaces, with regard to realistic environment simulations, and the use of reinforcement learning techniques. The search was made using the “Web of Science” repository, covering the last 5 years, with the following search key:

\begin{verbatim}
    ("reinforcement learning" OR "dynamic programming" OR
     "optimal control" OR "control theory" OR "machine learning")
    AND
    ("market making" OR "market maker")
\end{verbatim}

Initially, 64 references were selected and deemed relevant to the project, with 23 of them being selected for further analysis and effectively used in our analysis,
with 4 of them being added to the list after the initial selection.
The references were selected based on their relevance to the project and tagged according to the following groups (for each non-binary category, combinations of tags were allowed):
type of data (simulated or real-time connections);
chosen state space (tensors of bid-ask spreads, order imbalance, and N-depth order books);
chosen action space (limit orders, market orders, and others); reward space (spread, volume, profit, and others);
algorithms used (Q-Learning, Deep Q-Learning, Actor-Critic, or others); whether using a multi-agent approach (dueling or market agents);
the use of model-free environments; and chosen metrics (Sharpe Ratio, PnL, and others);
finally, results were tagged by benchmarking (against other models or strategies).

State-of-the-art methodologies defined state spaces based on market-level observations, such as inventory, bid-ask spreads, and order book depths,
which align well with the real-time data available to market-making agents \cite{he2023integrating, bakshaev2020marketmaking}.
Action spaces are typically designed to include price adjustments, order placements, and cancellations,
reflecting the dynamic nature of market interactions \cite{sun2020marketmaking, gasperov2021marketmaking}.
Reward structures are tailored to balance profitability, risk, and liquidity provision,
often incorporating penalties for inventory imbalances or adverse market impacts \cite{sun2020marketmaking, gasperov2021marketmaking}.

Recent trends in reinforcement learning research emphasize model-free approaches, particularly Deep Q-Learning and Actor-Critic methods,
which leverage neural networks to approximate value functions or policies in high-dimensional spaces \cite{patel2018optimizing, ganesh2019reinforcement}.
These algorithms demonstrate strong adaptability to evolving market conditions while maintaining computational feasibility.
However, the inclusion of constraints, such as inventory and volatility penalties,
is essential to ensure practical applicability and risk management in live trading scenarios \cite{jerome2022modelbased, selser2021optimal}.

Multi-agent frameworks have also gained attention, enabling the simulation of competitive market environments where agents adapt their strategies in response to others.
These frameworks are particularly useful for studying market impact and stability under varying conditions \cite{ganesh2019reinforcement, jericevich2021simulation}.
Metrics such as the Sharpe Ratio and profit-and-loss analysis remain key benchmarks for evaluating the performance of these algorithms,
providing a robust basis for comparing different strategies \cite{bakshaev2020marketmaking, he2023integrating}.

In summary, the bibliographical review underscores the relevance of reinforcement learning in optimizing market-making strategies,
particularly when combined with realistic state, action, and reward spaces.
The insights gained from this review inform the design of our proposed restrictions-based RL framework,
which seeks to enhance the stability and interpretability of market-making algorithms while addressing critical risk management challenges.

\subsection{State, Action, and Reward Spaces}
% graph: images/state_space.png
\begin{figure}[H]
    \centering
    \includegraphics[width=0.8\textwidth]{images/state_space}
    \caption{State Space}
    \label{fig:state-space}
\end{figure}

% Choosen State Space
% State variable
% Total times used
%{
%    "inventory": "Agent inventory",
%    "bidask spread": "Bid-Ask spread",
%    "lob": "Price Levels",
%    "midprice": "Midprice",
%    "order status": "Posted order status",
%    "volume": "Transaction volume",
%    "order imbalance": "Order imbalance",
%    "volatility": "Volatility",
%    "price prediction": "Price prediction",
%    "time": "Market time"
%}

    \section{Methodology}

\subsection{Problem Definition}
The market-making problem addressed in this work involves designing an optimal trading policy for an agent using reinforcement learning (RL). The agent aims to maximize profit while managing risks, particularly inventory risk. The market dynamics are modeled by the limit order book (LOB) and its dynamics, which define how the book evolves over time based on order flow and price movements. Our agent interacts with this environment by quoting bid and ask prices and adjusting offered quantities. As discussed previously, the main challenge for choosing an adequate agent and its policy lies in balancing profitability with risk management, especially regarding inventory at the close of the market, where overnight positions can expose the agent to significant risks.

\subsection{Formal Description of the RL Environment} In modeling the RL environment, we initially utilize a continuous-time, continuous-state Markov Chain framework, and later transition to a discrete representation to address computational space constraints. We choose a state space that tries to best incorporate the historical events of the limit order book into a single observable state using commonly used indicators and LOB levels, as well as intrinsic features to the agent as proposed by <multiple references, inserir paper com review de RL para MM>. Given our performed bibliographical research, we chose the agent's current inventory for the intrinsic feature and a set of indicators for the extrinsic features: the Relative Strength Index (RSI); order imbalance (O); and micro price (MP). Additionally, for a fixed number $D$ of LOB price levels the pair $(\delta^d, Q^d)$, where $\delta^d$ is the half-spread distance for the level $d \leq D$, and $Q^d$ the amount of orders posted at that level is added to the state as a set of tuples, for both the ask and bid sides of the book. The state space can therefore be formally expressed as:
$$
X_t = (I_t, \, RSI_t, \, O_t, \, \text{MP}_t, \, \{ (\delta_t^{d, \text{ask}}, Q_t^{d, \text{ask}}) \}_{d=1}^D, \, \{ (\delta_t^{d, \text{bid}}, Q_t^{d, \text{bid}}) \}_{d=1}^D)
$$
As mentioned, the evolution is continuous in time, meaning state changes occur at any point in continuous time, and the next event occurs after a sampled waiting time. The specific case in which a Markov Chain also has an associated reward distribution $R$ for each state transition is called a Markov Reward Process and given that the MM problem also has a decision process that affects the transition probabilities it is therefore called a Markov Decision Process in control literature, and is generically defined as a 4-tuple $ (\mathcal{S}, \mathcal{A}, \mathbb{P}, R) $, where:

\begin{itemize}
	\item $\mathcal{S}$ is a set of states called the state space.
	\item $\mathcal{A}$ is a set of actions called the action space.
	\item $P: \mathcal{S} \times \mathcal{A} \times \mathcal{S} \to [0, 1]$ is the transition probability function for the MDP.
	\item $R: \mathcal{S} \times \mathcal{A} \times \mathcal{S}' \rightarrow \mathbb{R}$ is the reward function associated with each state transition.
\end{itemize}

Furthermore, the decision process will be defined in detail, as well as its effects on the choice of actions and the transition probability.

\subsection{State Transition Distribution and Environment Evolution Dynamics}

The previously mentioned transition probability density $P$ is given by a Stochastic Differential Equation expressed by the Kolmogorov forward equation for Markov Decission Processes:

\begin{equation}
	\frac{\partial P(s', t | s, a)}{\partial t}  = \int_{\mathcal{S}} \mathcal{L}(s' | a, t) P(x| s, a, t) dx
\end{equation}

for all $s, s' \in \mathcal{S}$ and all times $t$ before market end $T$, that is, $t \le T$, where $a$ is choosen by our control agent according to a policy $\pi (s)$. $\mathcal{L}$ is the generator operator and governs the dynamics of the state transitions given the current time. It's expression is obtained by deriving the closed-form expression for the dynamics of the underlying model, and for simple models such as those proposed by <avellaneda & stoikov, gueant>, the closed-form expression has been calculated. The underlying model for this paper is more complex, and solving for the respective generator operator is outside the scope of this paper, and a numerical approximation will be used furthermore when we define the models for the Proximal Policy Optimization (PPO) and Advantage Actor Critic (A2C) in Section 5 <inserir link>.

\subsubsection{Chosen State Space}
In continuous-time and continuous-state MDPs, the state $S$ evolves as a continuous-time stochastic process with dynamics reflected by $\mathcal{L}$ and - as mentioned - modern approaches to Reinforcement Learning require solving either numerically or approximating its transition probabilities. For our choosen market simulation model, we first formally define each component of our proposed state space:

\begin{itemize}
	$$
	\mathbf{S}_t = \left( \text{RSI}_t, \text{OI}_t, P_{\text{micro},t}, \Delta P_{1,t}, \Delta P_{2,t}, \dots, \Delta P_{d,t} \right) \in \mathbb{R}^3 \times \mathbb{R}^{2d}.
	$$
	
	The components of the state space are defined as follows:
	
	- \textbf{Order Imbalance (OI):} Order imbalance measures the relative difference between buy and sell orders at a given time. It is defined as:
	$$
	\text{OI}_t = \frac{Q_t^{\text{bid}} - Q_t^{\text{ask}}}{Q_t^{\text{bid}} + Q_t^{\text{ask}}},
	$$
	where \( Q_t^{\text{bid}} \) and \( Q_t^{\text{ask}} \) represent the total bid and ask quantities at time \( t \), respectively. \( \text{OI}_t \in [-1, 1] \), with \( \text{OI}_t = 1 \) indicating complete dominance of bid orders, and \( \text{OI}_t = -1 \) indicating ask order dominance.
	
	- \textbf{Relative Strength Index (RSI):} The RSI is a momentum indicator that compares the magnitude of recent gains to recent losses to evaluate overbought or oversold conditions. It is given by:
	$$
	\text{RSI}_t = 100 - \frac{100}{1 + \frac{\text{Average Gain}}{\text{Average Loss}}},
	$$
	where the \textit{Average Gain} and \textit{Average Loss} are computed over a rolling window (commonly 14 periods). Gains are the price increases during that window, while losses are the price decreases.

	- \textbf{Micro Price (\( P_{\text{micro}} \)):} The micro price is a weighted average of the best bid and ask prices, weighted by their respective quantities:
	$$
	P_{\text{micro},t} = \frac{P_t^{\text{ask}} Q_t^{\text{bid}} + P_t^{\text{bid}} Q_t^{\text{ask}}}{Q_t^{\text{bid}} + Q_t^{\text{ask}}},
	$$
	where \( P_t^{\text{ask}} \) and \( P_t^{\text{bid}} \) represent the best ask and bid prices at time \( t \).
	$$
\end{itemize}

\subsubsection{Chosen Action Space}

The control, or agent, interacts with the environment choosing actions from the set of possible actions, such that $a \in \mathbf{A}$ in response to observed states $s \in \mathbf{S}$ according to a policy $\pi (s, a)$ which we define shortly, and the goal is to maximize cumulative rewards over time. The system's dynamics depend on the agent's chosen action, so as to introduce features of market impact into our model, and the transition probabilities between states therefore depend on the agent's actions.

The action space $\mathcal{A}_t$ includes the decisions made by the agent at time $t$, specifically the desired bid and ask spreads pair $\delta_t^{\text{ask}}, \delta_t^{\text{bid}}$ and the corresponding posted order quantities $Q_t^{\text{ask}}, Q_t^{\text{bid}}$. Formally:
$$
\mathbf{A}_t = \left( \delta_t^{\text{ask}}, \delta_t^{\text{bid}}, q_t^{\text{ask}}, q_t^{\text{bid}} \right) \in \mathbb{R}^2 \times \mathbb{Z}^2.
$$

\subsubsection{Episodic Reward Function and Returns}

The episode reward function $r_t \in \mathbb{R}$ reflects the agent's profit and inventory risk obtained during a specific time in a past episode, it's value is given by the immediate reward function $R$, and differently from the instantaneous reward function it is discrete in time, so as to match observed event times (order arrivals and transactions occured). It depends on the spread and executed quantities, as well as the inventory cost and was choosen according to commonly used reward structures taken from the literature review.

The overall objective is to maximize cumulative utility while minimizing risk associated with inventory positions, and later insert restrictions so the risk for inventory is either limited at zero at market close, or incurring in larger penalties on the received rewards. For our model the utility chosen is based on a running Profit and Loss (PnL) score while still managing inventory risk. The choosen reward function is based on a risk-aversion enforced utility function, specifically the \textit{constant absolute risk aversion (CARA)}:

The PnL at time $t$ is computed as:
$$
\text{PnL}_t = \delta_t^{\text{ask}} q_t^{\text{ask}} - \delta_t^{\text{bid}} q_t^{\text{bid}} + \text{I}_t \cdot \Delta M_t.
$$

The agent starting penalty for holding large inventory positions  is discounted from the \textit{PnL} score, as follows:
$$
\text{Penalty}_t = \eta \left( \text{Inventory}_t \cdot \Delta M_t \right)^+,
$$
$$
\text{PnL}_t := \text{PnL}_t - \text{Penalty}_t
$$
where \( \eta \) is the penalty factor applied to positive inventory changes.

$$
R_t = U(\text{PnL}_t) = -e^{-\gamma \cdot \text{PnL}_t},
$$
where \( \gamma \) is the risk aversion parameter.

<inserir explicacao do para que serve o return>

\begin{equation}
G(\tau) = \int_0^T e^{-\gamma t} R(s_{t+dt}, s_t, a_t) \, dt
\end{equation}

\subsection{Model Description and Environment Dynamics}

For our model of the limit order book the timing of events follows a \textit{Hawkes process} so as to represent a continuous-time MDP that captures the usual observed pattern of clustered order arrival times.

The Hawkes process is a \textit{self-exciting process}, where the intensity \( \lambda(t) \) depends on past events. Formally, the intensity \( \lambda(t) \) evolves as:
$$
\lambda(t) = \mu + \sum_{t_i < t} \phi(t - t_i),
$$
where \( \mu > 0 \) is the baseline intensity, and \( \phi(t - t_i) \) is the \textit{kernel function} that governs the decay of influence from past events \( t_i \). A common choice for \( \phi \) is an exponential decay:
$$
\phi(t - t_i) = \alpha e^{-\beta (t - t_i)},
$$
where \( \alpha \) controls the magnitude of the self-excitation and \( \beta \) controls the rate of decay.

The bid and ask prices for each new order are modeled by two separate \textit{Ornstein-Uhlenbeck (OU) processes} to capture the mean-reversion behavior of spreads over the midprice:
$$
ds_t = \theta(\mu - s_t) dt + \sigma dW_t,
$$
where \( s_t \) is the market spread at time \( t \), \( \theta \) is the rate of mean reversion, \( \mu_{spread} \) is the long-term spread mean and \( \sigma \) its volatility. The Wiener process \( W_t \) is used to represent random market fluctuations.

The bid and ask spreads \( \delta_t^{\text{bid}} \) and \( \delta_t^{\text{ask}} \) for orders conditioned on their arrival follow normal distributions:
$$
\delta_t^{\text{ask}} \sim \mathcal{N}(\mu + M_{t-1} + s_t, \sigma^2), \quad a_t^{\text{bid}} \sim \mathcal{N}(\mu + M_{t-1} - s_t, \sigma^2),
$$
where \( \mu_{\text{ask}} \) and \( \mu_{\text{bid}} \) are the respective means and \( \Delta a_t \) is the spread adjustment. Whenever a new limit order that narrows the bid-ask spread or a market order arrive the mid-price is updated to reflect the top-of-book orders. The mid-price \( M \) at time $t+1$ is then obtained by averaging the maximum and minimum bid and ask prices in the book:

$$
M_{t+1} = \frac{2M_t + \delta^{ask}_{t} - \delta^{bid}_{t}}{2} 
\sim \mathcal{N}(\mu + M_{t}, \frac{\sigma^2}{2})
$$ and at $t = 0$, the midprice is defined according to some starting point of the model, usually an observed historical price as is chosen for our simulation as well.

The average of the top of book bid and ask prices therefore evolves according to a \textit{Brownian Motion} process:
$$
dM_t = \mu dt + \frac{\sigma^2}{2} dW_t,
$$
where \( W_t \) is a Wiener process and since the midprice is given by the sum of the top of book ask and bid prices, orders that cross the spread are usually rare and reflect a common stylized fact of LOBs in the market <inserir referencia book HFT>.

The order quantities \( q_t^{\text{ask}} \) and \( q_t^{\text{bid}} \) are modeled as Poisson random variables:
$$
q_t^{\text{ask}}, q_t^{\text{bid}} \sim \text{Poisson}(\lambda_q),
$$
where \( \lambda_q \) is the average order size.

\subsection{Decision Process and Steps to Maximize the Agent's Objective}

To express the Bellman equation for continuous-time MDPs, we consider the value function $V(s', a, s), which represents the expected total reward from state \( X \) under the optimal policy \( \pi^* \). The Bellman equation for a continuous-time MDP is:

<nao sei o que nao sei o que la> - pegar da proposta II enviada à FAPESP.


    \section{Implementation and Model Description}
\label{sec:implementation-and-model-description}

The chosen model architecture is an Actor-Critic model, which consists of two neural networks: the Actor and the Critic.
The Actor network is the model that learns the policy, which is the probability distribution of actions given a state.
The Critic network is the model that learns the value function, which is the expected return of the policy.

We use a neural network to approximate the policy and value functions, which are trained using the Proximal Policy Optimization (PPO) algorithm.
The PPO algorithm is a model-free, on-policy algorithm that optimizes the policy directly, using a clipped surrogate objective function to ensure stable training,
as discussed in \hyperref[alg:ppo]{Section~\ref{alg:ppo}}, and follows the usual Generalized Policy Iteration (GPI) framework.

\subsection{Model Architecture}
\label{subsec:model-architecture}

\begin{figure}[H]
    \centering
    \includegraphics[width=0.8\textwidth]{images/policy}
    \caption{Actor Network Architecture}
    \label{fig:actor-architecture}
\end{figure}

The chosen model architecture is an Actor-Critic model, which consists of two neural networks: the Actor and the Critic.
The Actor network is the model that learns the policy, which is the probability distribution of actions given a state.
As input layers, we separated the state space into two different inputs: the market features and the LOB data.
The market features are the general information about the market, such as the spread, and the volume,
and fabricated features such as the 10-period moving average of the spread and the relative-strength index.
The LOB data is the information about the limit order book, specifically the N-best bid and ask prices and volumes.
The architecture passes the LOB data to an attention mechanism,
aiming to allow the network to create relationships between the different levels of the LOB,
and the market features are passed through a feed-forward neural network.
The output of the attention mechanism is concatenated with the market features and passed through the final dense layers of the Actor network.

The Critic network is the model that learns the value function, which is the expected return of the policy.
The chosen model architecture is a simple feed-forward neural network with two hidden layers, of 128 and 64 units, respectively.
The Critic network is trained using the Generalized Advantage Estimation (GAE)~\cite{Schulman2015} to estimate the advantages of the actions taken by the Actor network.

The Actor and Critic networks are trained using the Proximal Policy Optimization (PPO) algorithm~\cite{Schulman2017}.

\subsection{Main Training Loop}
\label{subsec:main-training-loop}

Our training loop is shown in \hyperref[alg:algorithm]{Algorithm~\ref{alg:algorithm}}.
The loop consists of collecting trajectories, computing the Generalized Advantage Estimation (GAE) and returns,
and updating the Actor and Critic networks according to the PPO algorithm and the backpropagation of the loss
shown in \hyperref[alg:ppo]{Algorithm~\ref{alg:ppo}}.

\begin{algorithm}[H]
    \begin{algorithmic}[1]
        \Require Environment, PPO model, optimizer, number of episodes $num\_episodes$
        \For{each episode in range $num\_episodes$}
            \State \textbf{Collect trajectories:}
            \State \hspace{1em} Initialize the state: $\text{s} = \text{env.reset()}$
            \State \hspace{1em} Initialize an empty trajectory buffer
            \For{each timestep in the episode}
                \State \hspace{1em} Select action $a \sim \pi_{\theta}(s)$
                \State \hspace{1em} Observe reward $R_t$ and next state $s'$
                \State \hspace{1em} Store transition $(s, a, r)$ in the trajectory buffer
                \State \hspace{1em} Set $s \leftarrow s'$
                \If{environment done}
                    \State \hspace{1em} \textbf{end} episode
                \EndIf
            \EndFor
            \State \textbf{Compute GAE and Returns:}
            \State \hspace{1em} Compute advantages and returns using GAE
            \State \textbf{Update actor and critic according to \hyperref[alg:algorithm]{Algorithm~\ref{alg:ppo}}}
        \EndFor
    \end{algorithmic}
    \caption{Training Loop}
    \label{alg:algorithm}
\end{algorithm}

We train the model for a fixed number of episodes, and at the end of each episode, we update the model using the collected trajectories.


%\begin{algorithmic}
%
%\end{algorithmic}



    \section{Realized Experiments and Result Discussion}
\label{sec:realized-experiments-and-result-discussion}

\subsection{Experiment Setup}
\label{subsec:experiment-setup}

The experiments were conducted using a Red-black tree implementation for the limit order book, while new orders follow the event dynamics described in
~\hyperref[subsec:market-model-description-and-environment-dynamics]{Section~\ref*{subsec:market-model-description-and-environment-dynamics}}.
Both restricted agents were implemented through the Proximal Policy Optimization (PPO) algorithm, which is a model-free, on-policy algorithm that optimizes the policy directly.
A max episode value of 10,000 episodes was used, with each episode consisting on average of 390 observations (or 1 event per corresponding market minute).
Per gathered trajectory, 500 epochs were used for the policy improvement step, with a batch size of 64.
We used the Adam optimizer with a learning rate of $10^{-4}$ and a discount factor of 0.99.
For comparison metrics, we used the Sharpe ratio, the average daily return, and the average daily volatility.

\subsection{Experiment Results}
\label{subsec:experiment-results}

% Graphs:
% average financial return + confidence interval (+- volatility) x episode number
% average financial return (+- volatility) x current timestep (per 100x trajectories after training)
% average inventory x current timestep (per 100x trajectories after training)
% average reward moving average x episode number

% Table:
% Cols: agent with restriction, agent without restriction, benchmark
% Training
% Rows: Training time +- std
%       Time per episode +- std
%       Mean processing time actor +- std per episode
%       Mean processing time critic +- std per episode
%       Mean financial return +- std

% Test (after last episode or convergence)
% Rows: Mean financial return +- std
%       Mean Sharpe ratio +- std
%       Agent action latency +- std
%       Mean inventory at market close


    \section{Conclusion}
\label{sec:conclusion}
We have presented the design and implementation of a reinforcement learning agent aiming to show
the effects of adverse market conditions and non-stationary environments on control agents for market-making.
As discussed in~\autoref{subsec:market-model-description-and-environment-dynamics},
our approach for a environment models the dynamics of a limit order book (LOB)
according to a set of parameterizable stochastic processes configured to mimic observed stylized facts in real markets
The resulting market model replicates stylized facts for the midprice, spread, price volatility, and order arrival rate,
as well as the impact of market orders on the agent's inventory, return standard deviation and end-of-day Profit and Loss score.

The fine-controlled dynamics where the agent interacts with the environment allows us to model the effects of market impact and inventory risk on the agent's performance,
and use it to evaluate the agent's performance under adverse market conditions, providing a more realistic training environment for RL agents in market-making scenarios.
The proposed methodology was able to capture the effects of market impact and inventory risk on the trading agent's performance,
and affected the resulting agent's decision-making policies, as shown in~\autoref{subsec:experiment-results},
which we verified by comparing the agent's performance against simpler benchmark agents that
do not account for these effects in their decision-making policies.

Prospective extensions of the presented work and future research on existing market model simulators include
developing hybrid world models to combine both model-based and model-free approaches and leverage the capability of RL agents
to learn from non-stationary environments and historical observations of real markets.
Hybrid world models could further improve the capability of RL agents in adapting to changing market conditions
and could provide a more realistic training environment for market-making agents.


    \bibliographystyle{plain}
    \bibliography{references}

\end{document}
