\documentclass{article}

\usepackage{arxiv}

\usepackage{amsmath,amssymb,amsfonts}
\usepackage{algorithm}
\usepackage{algorithmicx}
\usepackage{algpseudocode}
\usepackage{cite}
\usepackage{epsfig}
\usepackage{float}
\usepackage{graphicx}
\usepackage{hyperref}
\usepackage{mathtools}
\usepackage{natbib}
\usepackage{textcomp}
\usepackage{pgf}
\usepackage{subcaption}
\usepackage{siunitx}
\usepackage{tikz}
\usepackage{xcolor}
\usetikzlibrary{shapes.geometric, arrows, positioning}

%\usepackage[left=54pt, right=54pt, bottom=54pt]{geometry}

%\geometry{papersize={8.5in,11in}, top=64pt} % First-page top margin
%\usepackage{fancyhdr}
%
%\pagestyle{fancy}
%\fancyhf{}
%\renewcommand{\headrulewidth}{0pt} % Remove header rule
%
%% Adjust top margin for subsequent pages
%\usepackage{afterpage}
%\afterpage{\newgeometry{left=54pt, right=54pt, bottom=54pt, top=54pt}}
%
\setlength{\abovedisplayskip}{.5cm}  % Reduce space above equations
\setlength{\belowdisplayskip}{.5cm}  % Reduce space below equations
\setlength{\abovedisplayshortskip}{.5cm}  % For short equations
\setlength{\belowdisplayshortskip}{.5cm}  % For short equations
%\pdfminorversion=4

\usepackage{enumitem}
\setlist[itemize]{left=0pt,labelsep=5pt}

\newcommand{\Title}{Reinforcement Learning-Based Market Making as a Stochastic Control on Non-Stationary Limit Order Book Dynamics}

\title{\Title}

%\date{September 9, 1985}	% Here you can change the date presented in the paper title
\date{}

\author{%
  \href{https://orcid.org/0009-0008-6064-9895}{\includegraphics[scale=0.06]{orcid}\hspace{1mm}Rafael Zimmer}\thanks{Use footnote for providing further information about author (webpage, alternative address)---\emph{not} for acknowledging funding agencies.} \\
  Institute of Mathematics and Computer Sciences\\
  University of São Paulo\\
  \texttt{rafael.zimmer@usp.br}
  \And
  \href{https://orcid.org/0000-0001-5989-7287}{\includegraphics[scale=0.06]{orcid}\hspace{1mm}Elias L. V.~Costa} \\
  School of Polytechnic Engineering\\
  University of São Paulo\\
  \texttt{oswaldo.costa@usp.br}
}%
%\author{
%    \begin{minipage}{0.45\textwidth}
%        \centering
%        \href{https://orcid.org/0009-0008-6064-9895}{\includegraphics[scale=0.06]{images/orcid} Rafael Zimmer}\\
%        \textit{Institute of Mathematics and Computer Sciences} \\
%        São Paulo, Brazil \\
%%        \texttt{rafael.zimmer@usp.br}
%        \href{mailto:rafael.zimmer@usp.br}{rafael.zimmer@usp.br}
%    \end{minipage}
%    \hfill
%    \begin{minipage}{0.45\textwidth}
%        \centering
%        \href{https://orcid.org/0000-0001-5989-7287}{\includegraphics[scale=0.06]{images/orcid} Oswaldo Luiz Do Valle Costa}\\
%        \textit{Escola Politécnica of the University of São Paulo} \\
%        São Paulo, Brazil \\
%        \href{mailto:oswaldo.costa@usp.br}{oswaldo.costa@usp.br}
%    \end{minipage}
%}

\renewcommand{\undertitle}{}
\renewcommand{\shorttitle}{}
\renewcommand{\headeright}{}

\hypersetup{
pdftitle={A template for the arxiv style},
pdfsubject={q-bio.NC, q-bio.QM},  % https://arxiv.org/category_taxonomy
pdfauthor={Rafael Zimmer, Elias D.~Striatum},
pdfkeywords={First keyword, Second keyword, More},
}
%%% check with `pdfinfo preprint.pdf`

\begin{document}
    \twocolumn[
    \maketitle
    \vspace{-3em}
    \begin{abstract}
        Reinforcement Learning has emerged as a promising framework for developing adaptive and data-driven strategies,
        enabling market makers to optimize decision-making policies based on interactions with the limit order book environment.
        This paper explores the integration of a reinforcement learning agent in a market-making context,
        where the underlying market dynamics have been explicitly modeled to capture observed stylized facts of real markets,
        including clustered order arrival times, non-stationary spreads and return drifts, stochastic order quantities and price volatility.
        These mechanisms aim to enhance stability of the resulting control agent,
        and serve to incorporate domain-specific knowledge into the agent policy learning process.
        Our contributions include a practical implementation of a market making agent based on the Proximal-Policy Optimization (PPO) algorithm,
        alongside a comparative evaluation of the agent's performance under varying market conditions via a simulator-based environment.
        As evidenced by our analysis of the financial return and risk metrics when compared to a closed-form optimal solution,
        our results suggest that the reinforcement learning agent can effectively be used under non-stationary market conditions,
        and that the proposed simulator-based environment can serve as a valuable tool for training and
        pre-training reinforcement learning agents in market-making scenarios.
    \end{abstract}
    \keywords{Reinforcement Learning \and Market Making \and Limit Order Book \and Stochastic Control \and Market Microstructure}
    \vspace{2em}
    ]

    \section{Introduction}
\label{sec:introduction}

Market making in financial markets involves continuously quoting buy and sell prices and dealing with supply and demand imbalances.
Market makers are essential, as they narrow the bid-ask spreads, reduce price volatility, and maintain market stability,
particularly in times of uncertainty by providing liquidity, which helps to maintain a fair and efficient market~\cite{Glosten1985, OHara1995}.
With the advent of electronic trading, placing optimal bid and ask prices as a market maker is becoming an almost completely automatized task,
but such a transition comes with the hardships of dealing with additional caveats, including slippage, market impact, adverse market regimes~\cite{Cont2010, Bouchaud2018}
and non-stationary conditions~\cite{Gasperov2021}.

The Reinforcement Learning (RL) paradigm has shown promising results for optimizing market making strategies,
where agents learn to adapt their quote placing policies through trial and error given a numerical outcome reward score\footnote{Not to be confused with financial returns.}.
The RL approach is based on the Bellman equation for state values or state-action pair values,
which recursively define the state-value of a policy as the expected return of discounted future rewards.
State-of-the-art RL algorithms, such as Proximal Policy Optimization (PPO) and Deep Q-Learning (DQN),
solve the Bellman equation by approximating the optimal policy using neural networks to learn either the policy, the value function~\cite{Sutton2018},
or both~\cite{Schulman2015, Mnih2015}, aiming to allow the agent to its quoted prices based adaptively on the observed market conditions~\cite{He2023, Bakshaev2020}.

Using historical data to train RL agents is a common practice in the literature, but it has some limitations,
as it is computationally expensive and requires a large amount of data,
besides not including the effects of market impact and inventory risk~\cite{Frey2023, Ganesh2019} during the training process.
An additional approach to creating realistic environments are agent-based simulations,
where generative agents are trained against observed market messages and used to simulate the limit order book environment~\cite{Frey2023, Ganesh2019},
but this approach has the disadvantage of trading off control over the market dynamics for replicating observed market flow,
as well as limiting the agent's adaptability to unseen market regimes, which can lead to underfitting~\cite{Jerome2022, Selser2021}
and suboptimal decision-making under scenarios where market impact and slippage have a more prominent effect on the agent's performance.
On the other hand, using stochastic models to simulate the limit order book environment is computationally cheaper and faster to train~\cite{Gasperov2021, Sun2022},
but might not generalize well to unseen market conditions.

In the context of this paper, we implement a RL-based market making agent and test its robustness and generalization capabilities under non-stationary environments.
The agent is trained on a crafted simulator that models the dynamics of a limit order book (LOB) according to a set of parameterizable stochastic processes.
This approach aims to demonstrate the effects of non-stationary environments and adverse market conditions on
a RL-based market making agent, and how the agent's decision-making policies are affected by market impact and inventory risk.
Our main contribution is implementing multiple non-stationary dynamics into a single limit order book simulator,
and how using fine-controlled non-stationary environments can enhance the agent's performance under adverse market conditions
and provide a more realistic training environment for RL agents in market-making scenarios.
We also perform a comparative analysis of the agent's performance under changing market conditions during the
trading day and benchmark it against a closed-expression optimal solution (under a simplified market model),
aiming to validate the usage of RL-based market making agents in markets with more complex dynamics.



    \section{TODO: Inserir Resultados da Pesquisa Bibliografica}

    \section{Methodology}
\label{sec:methodology}

\subsection{Problem Definition}
\label{subsec:problem-definition}

The market-making problem addressed in this work involves designing an optimal trading policy for an agent using reinforcement learning.
The policy should be able to balance profit and risk, particularly inventory risk under the restriction of having zero inventory at market close,
and interacts with the environment by quoting bid and ask prices and adjusting offered quantities.
The environment dynamics are modeled by an underlying limit order book (LOB) and its behaviour will be described in the following sections.
The agent's objective is to maximize cumulative rewards over time, while minimizing inventory risk and market impact costs.

\subsection{A Formal Description of the Chosen RL Environment}
\label{subsec:formal-description-of-the-rl-environment}
In modeling the environment, we initially utilize a continuous-time, continuous-state Markov Chain framework,
and later transition to a discrete implementation to address computational space constraints.
The specific case in which a Markov Chain also has an associated reward distribution $R$ for each state transition is called a Markov Reward Process
and given that the agent's decisions also affect the transition probabilities due to market impact it is therefore called a Markov Decision Process (MDP) in control literature.
A Markov Decision Process can generically be defined as a 4-tuple $ (\mathcal{S}, \mathcal{A}, \mathbb{P}, R) $, where:

\begin{itemize}
    \item $\mathcal{S}$ is a set of states called the state space.
    \item $\mathcal{A}$ is a set of actions called the action space.
    \item $P: \mathcal{S} \times \mathcal{A} \times \mathcal{S} \to [0, 1]$ is the transition probability function for the MDP.
    \item $R: \mathcal{S} \times \mathcal{A} \times \mathcal{S}' \rightarrow \mathbb{R}$ is the reward function associated with each state transition.
\end{itemize}

Then, the agent's interaction with the environment is made through actions chosen from the action space $\mathcal{A}$ in response to observed states $s \in \mathcal{S}$,
according to a policy $\pi (s, a)$.
The agent's goal is to maximize cumulative rewards over time, which are obtained from the reward function $R$.
The return $G_t$ is used to estimate the value of a state-action pair by summing the rewards obtained from time $t$ to the end of the episode (usually, as well as in our case,
discounted by a factor $\gamma$ to favor immediate rewards), where $R_{t}$ is the observed reward at time $t$ and $T$ is the time of the episode's end:

\[
    G_t = \int_{t}^{T} \gamma^{k-t} R_{k} dk
\]

%\[
%    \pi^{*} = \arg \max_{\pi} \mathbb{E} \left[ G_t | s_t, \pi \right]
%\]

\subsubsection{Chosen State Space}
We choose a state space that tries to best incorporate the historical events of the limit order book into a single observable state using commonly used indicators and LOB levels,
as well as intrinsic features to the agent deemed relevant and chosen through our initial bibliography research, as well as ~\cite{Gasperov2021}.
Given our performed bibliographical research, we chose the agent's current inventory for the intrinsic feature and a set of indicators for the extrinsic features:
the Relative Strength Index (RSI); order imbalance (O); and micro price (MP).
Additionally, for a fixed number $D$ of LOB price levels the pair $(\delta^d, Q^d)$, where $\delta^d$ is the half-spread distance for the level $d \leq D$,
and $Q^d$ the amount of orders posted at that level is added to the state as a set of tuples, for both the ask and bid sides of the book.
The state space can therefore be represented by the following expression:

\[
    s_{t} \in \mathcal{S} = \left\{ \text{RSI}_t, \text{OI}_t, MP_{t}, \text{Inventory}_t, \left( \delta_t^{d, ask}, Q_t^{d, ask} \right)_{d=1}^{D}, \left( \delta_t^{d, bid}, Q_t^{d, bid} \right)_{d=1}^{D} \right\}
\]
where $0 < t < T$.

The indicators for our chosen market simulation framework are defined individually by values directly obtained from the observed LOB,
and serve as market state summaries for the agent to use:

\begin{itemize}
    \item \textbf{Order Imbalance (OI):} Order imbalance measures the relative difference between buy and sell orders at a given time.
    It is defined as:
    \[
        \text{OI}_t = \frac{Q_t^{\text{bid}} - Q_t^{\text{ask}}}{Q_t^{\text{bid}} + Q_t^{\text{ask}}},
    \]
    where \( Q_t^{\text{bid}} \) and \( Q_t^{\text{ask}} \) represent the total bid and ask quantities at time \( t \), respectively.
    \( \text{OI}_t \in [-1, 1] \), with \( \text{OI}_t = 1 \) indicating complete dominance of bid orders, and \( \text{OI}_t = -1 \) indicating ask order dominance.

    \item \textbf{Relative Strength Index (RSI):} The RSI is a momentum indicator that compares the magnitude of recent gains to recent losses to evaluate overbought or oversold conditions. It is given by:
    \[
        \text{RSI}_t = 100 - \frac{100}{1 + \frac{\text{Average Gain}}{\text{Average Loss}}},
    \]
    where the \textit{Average Gain} and \textit{Average Loss} are computed over a rolling window (in our case, fixed 5 minute observation intervals).
    Gains are the price increases during that window, while losses are the price decreases.

    \item \textbf{Micro Price (\( P_{\text{micro}} \)):} The micro price is a weighted average of the best bid and ask prices, weighted by their respective quantities:
    \[
        P_{\text{micro},t} = \frac{P_t^{\text{ask}} Q_t^{\text{bid}} + P_t^{\text{bid}} Q_t^{\text{ask}}}{Q_t^{\text{bid}} + Q_t^{\text{ask}}},
    \]
    where \( P_t^{\text{ask}} \) and \( P_t^{\text{bid}} \) represent the best ask and bid prices at time \( t \).

\end{itemize}

\subsubsection{Chosen Action Space}

The control, or agent, interacts with the environment choosing actions from the set of possible actions,
such that $a \in \mathcal{A}$ in response to observed states $s \in \mathcal{S}$ according to a policy $\pi (s, a)$ which we will define shortly,
and the end goal is to maximize cumulative rewards over time.
The agent's chosen action impacts the evolution of the system's dynamics by inserting orders into the LOB that might move the observed midprice,
to introduce features of market impact into our model.

The action space $\mathcal{A}$ includes the decisions made by the agent at time $t$, specifically the desired bid and ask spreads pair
$\delta^{\text{ask}}, \delta^{\text{bid}}$ and the corresponding posted order quantities $Q^{\text{ask}}, Q^{\text{bid}}$:
$$
\mathcal{A} = \left\{ (\delta^{\text{ask}}, \delta^{\text{bid}}, Q^{\text{ask}}, Q^{\text{bid}}), \forall \delta \in \mathbb{R}^+, \forall Q \in \mathbb{Z}\} \right.
$$

\subsubsection{Episodic Reward Function and Returns}

The episode reward function $R_t \in \mathbb{R}$ reflects the agent's profit and inventory risk obtained during a specific time in the episode.
It depends on the spread and executed quantities, as well as the inventory cost and was choosen according to commonly used reward structures taken from the literature review.

The overall objective is to maximize cumulative utility while minimizing risk associated with inventory positions,
and later insert restrictions so the risk for inventory is either limited at zero at market close, or incurring in larger penalties on the received rewards.
For our model the utility chosen is based on a running Profit and Loss (PnL) score while still managing inventory risk.
The choosen reward function is based on a risk-aversion enforced utility function, specifically the \textit{constant absolute risk aversion (CARA)}~\cite{Arrow1965, Pratt1964}
and depends on the realized spread $\delta$ and the realized quantity $q$\footnote{Differs from the agent's posted order quantity $Q$, as $q$ is a stochastic variable dependent on the underlying market dynamics.}.
The running PnL at time $t$ is computed as follows, where a penalty for holding large inventory positions is discounted from the \textit{PnL} score:

\begin{gather*}
    \text{Running PnL}_t = \delta_t^{\text{ask}} q_t^{\text{ask}} - \delta_t^{\text{bid}} q_t^{\text{bid}} + \text{I}_t \cdot \Delta M_t, \\
    \text{Penalty}_t = \eta \left( \text{Inventory}_t \cdot \Delta M_t \right)^+,\\
    \text{PnL}_t \coloneqq \text{Running PnL}_t - \text{Penalty}_t\\
\end{gather*}
where \( \eta \) is the penalty factor applied to positive inventory changes.

Finally, the reward function is defined as the negative of the exponential of the running PnL, a common choice for risk-averse utility functions according to literature~\cite{Gueant2022, Selser2021a, FalcesMarin2022}.
The chosen CARA utility function is defined as follows, where \( \gamma \) is the risk aversion parameter:
\[
    R_t = U(\text{PnL}_t) = -e^{-\gamma \cdot \text{PnL}_t},
\]

\subsection{State Transition Distribution}
\label{subsec:state-transition-distribution}

The previously mentioned transition probability density $P$ is given by a Stochastic Differential Equation expressed by the Kolmogorov forward equation for Markov Decission Processes:

\begin{equation}
    \label{eq:equation2}
    \frac{\partial P(s', t | s, a)}{\partial t}  = \int_{\mathcal{S}} \mathcal{L}(x | s, a, t) P(s'| x, a, t) dx
\end{equation}

for all $s, s' \in \mathcal{S}$ and all times $t$ before market close $T$, that is, $t \le T$,
where $a$ is choosen by our control agent according to a policy $\pi (s)$.
$\mathcal{L}$ is the generator operator and governs the dynamics of the state transitions given the current time.

In continuous-time and state MDPs, the state dynamics is reflected by $\mathcal{L}$ and modern approaches to optimal control
solve analytically by obtaining a closed-form expression for the model's evolution equations, as in~\citet{Avellaneda2008, Gueant2017}.
or numerically by approximating its transition probabilities, as in~\citet{Gueant2022, Selser2021a, FalcesMarin2022}.
Closed-form expressions for $\mathcal{L}$ are obtainable for simple models, by usually disconsidering market impact,
which is not the case for our proposed model, and solving for the generator operator is therefore outside the scope of this paper.
A numerical approach will be used furthermore as an approximation for the policy distribution according to observed environment trajectories.
We will use a neural network as the approximator function for the actor's policy through the Proximal Policy Optimization (PPO) algorithm
for the optimization of the policy distribution, as described in~\hyperref[sec:implementation-and-models]{Section~\ref*{sec:implementation-and-model-description}}.


\subsection{Market Model Description and Environment Dynamics}
\label{subsec:market-model-description-and-environment-dynamics}
Our approach to the problem of market making leverages \textbf{online reinforcement learning} by means of a simulator that models the dynamics of
a limit order book (LOB) according to a set of stylized facts observed in real markets.
For our model of the limit order book the timing of events follows a \textit{Hawkes process}
to represent a continuous-time MDP that captures the observed stylized fact of clustered order arrival times.
The Hawkes process is a \textit{self-exciting process}, where the intensity \( \lambda(t) \) depends on past events.
Formally, the intensity \( \lambda(t) \) evolves according to the following equation:
\begin{gather*}
    \lambda(t) = \mu + \sum_{t_i < t} \phi(t - t_i),\\
    \phi(t - t_i) = \alpha e^{-\beta (t - t_i)},\\
\end{gather*}
where \( \mu > 0 \) is the baseline intensity, and \( \phi(t - t_i) \) is the \textit{kernel function} that governs the decay of influence from past events \( t_i \).
A common choice for \( \phi \) is the exponential decay function, where \(\alpha\) controls the magnitude of the self-excitation and \(\beta\) controls the rate of decay.
\newline

The bid and ask prices for each new order are modeled by two separate \textit{Geometric Brownian Motion} processes to capture the normally distributed returns observed in real markets.
The underlying partial differential equation governing the ask and bid prices are given by:
\begin{gather*}
    dX_{t}^{ask} = (\mu_t + s_t) X_{t}^{mid} dt + \sigma dW_t,\\
    dX_{t}^{bid} = (\mu_t - s_t) X_{t}^{mid} dt + \sigma dW_t,
\end{gather*}

where $\mu$ is the price drift, $s_t$ is the mean spread, and $\sigma$ the price volatility.
The drift rate process follows a \textit{Ornstein-Uhlenbeck} process, which is a mean-reverting process,
while the spread rate similarly follows a \textit{Cox-Ingersoll-Ross} process, accounting for non-negative spreads.

Whenever a new limit order that narrows the bid-ask spread or a market order arrive the midprice is updated to reflect the top-of-book orders.
The midprice $X_{t}^{mid}$ is therefore obtained by averaging the current top-of-book bid and ask prices:

\[
    X_{t}^{mid} = \frac{X_{t}^{ask} + X_{t}^{bid}}{2}
\]
and while there are no orders on both sides of the book, the midprice is either the last traded price,
observed or an initial value set for the simulation, in the given order of priority.
By analyzing the midprice process, we can make sure the returns are distributed according to $\mathcal{N}\left(\mu_t, \frac{\sigma^2}{2}\right)$,
and are therefore normally distributed, assuring the model reflects a stylized fact of LOBs commonly observed in markets~\cite{Gueant2022}.

Finally, the order quantities $q_t^{\text{ask}}$ and $q_t^{\text{bid}}$ are modeled as Poisson random variables, where the arrival rate $\lambda_q$ is a constant parameter.
\[
    q_t^{\text{ask}}, q_t^{\text{bid}} \sim \text{Poisson}(\lambda_q),
\]

Our simulator was implemented using the Red-black tree structure for the limit order book, while new orders follow the event dynamics described
by the Geometric Brownian Motion and Poisson processes for order details.
The individual market regime variables, specifically the spread, order arrival density and return drift,
are sampled from the aforementioned set of pre-defined distributions, and inserted into the simulator at each event time step.

\subsection{Decision Process}
\label{subsec:decision-process}

In reinforcement learning, the Bellman equation is a fundamental recursive relationship that expresses the value of a
state in terms of the immediate reward and the expected value of subsequent states.
For a given policy $\pi$, the Bellman equation for the value function $V(s)$ with respect to the chosen reward function is:
\begin{equation*}
    \begin{aligned}
        V^\pi(s) &= \sum_{a \in \mathcal{A}} \pi(a \mid s) \sum_{s' \in \mathcal{S}} P(s'_{t+1} \mid s_{t}, a) \left[ R_t + \gamma V^\pi(s'_{t+1}) \right]\\
        V^\pi(s) &= \mathbb{E}_\pi \left[ R_t + \gamma V^\pi(s_{t+1}) \mid s_t = s \right]\\
    \end{aligned}
\end{equation*}

where $V(s)$ is the value function, representing the expected return (cumulative discounted rewards) starting from state $s$,
going to state $s'$ by taking action $a$ and continuing the episode, $R_t$ is the reward obtained from the action taken,
and $\gamma$ is the discount factor, used to weigh the value favorably towards immediate rather than future rewards.
The Bellman equation underpins the process of optimal policy derivation, where the goal is to find the optimal policy $\pi^*$,
that is, the control for our action space $\mathcal{A}$ that maximizes the expected return,
and can solved by maximizing for the value function:
\begin{equation*}
    \begin{aligned}
        V^*(s) &= \max_a \mathbb{E} \left[ R_t + \gamma V^*(s_{t+1}) \mid s_t = s, a_t = a \right]\\
        \pi^*(s) &= \arg \max_{a \in \mathcal{A}} \mathbb{E} \left[ R_t + \gamma V^*(s_{t+1}) \mid s_t = s, a_t = a \right]\\
    \end{aligned}
\end{equation*}
The Bellman equation is used to recursively define state values in terms of immediate rewards and expected value of subsequent states,
and for simpler environments, the equation can be solved directly by dynamic programming methods.
Such methods, like value iteration or policy iteration, although effective, become computationally expensive and sometimes unfeasible for large state spaces,
or require a model of the environment, which is not always available in practice.
Neural networks have been successfully used to approximate both the value function or the optimal actions directly,
demonstrating state-of-the-art results in various domains, including games, robotics, and finance
~\cite{He2023, Bakshaev2020, Patel2018, Ganesh2019, Sun2022, Gasperov2021a}.
Our simulation environment is complex enough that the state space is too large to store all possible state-action pairs,
and since we assume no knowledge of the underlying dynamics of the environment when training the agent, a model-free approach is necessary.

\subsubsection{Generalized Policy Iteration and Policy Gradient}
\label{subsubsec:gpi}
Generalized Policy Iteration algorithms are a family of algorithms that combines policy evaluation and policy improvement steps iteratively,
so that the policy is updated based on the value function, and the value function is updated based on the policy.
While simpler methods directly estimate state values or state-action values, such as Q-learning, more complex methods
such as Actor-Critic algorithms, estimate both the policy and the value function.
The critic estimates the value function $V^\pi$, or $A^{\pi} = Q^{\pi} - V^{\pi}$, if using the advantage function, while the actor learns the policy $\pi$.
We use a \textbf{policy gradient} approach, where the agent optimizes the policy directly by maximizing the expected return,
through consecutive gradient ascent steps on the policy parameters given episodes of experience.
For policy gradient methods, the following equations express the gradient ascent step for the
policy parameters $\theta$ and the value function parameters $\phi$, using a $\alpha$ learning rate and the gradient operator $\nabla$:
\begin{gather*}
    \nabla_{\theta} J(\theta) = \mathbb{E}_{\tau \sim \pi_{\theta}} \left[ \sum_{t=0}^{T} \nabla_{\theta} \log \pi_{\theta}(a_t \mid s_t) A^{\pi}(s_t, a_t) \right]\\
    \nabla_{\phi} J(\phi) = \mathbb{E}_{\tau \sim \pi_{\theta}} \left[ \sum_{t=0}^{T} \nabla_{\phi} \left( V^{\pi}(s_t) - R_t \right)^2 \right]\\
    \pi_{\theta} \leftarrow \pi_{\theta} + \alpha \nabla_{\theta} J(\theta), \phi \leftarrow \phi + \alpha \nabla_{\phi} J(\phi)
\end{gather*}
\begin{figure}
    \centering
    \includegraphics[width=.8\columnwidth]{images/gpi}
    \caption{Diagram of the Generalized Policy Iteration loop for the implemented agent.}
    \label{fig:gpi}
\end{figure}

In the context of \textbf{Proximal Policy Optimization (PPO)}
the model is trained to improve its policy and value function by maximizing the expected return,
according to a clipped surrogate objective function, which ensures that the policy does not deviate excessively from the previous policy,
promoting stability during training.
For \textbf{PPO}, the loss function used for the policy gradient ascent is defined as:
\begin{gather*}
    L^{\text{actor}}(\theta) = \mathbb{E}_t \left[ \min \left( r_t(\theta) \hat{A}_t, \text{clip}(r_t(\theta), 1 - \epsilon, 1 + \epsilon) \hat{A}_t \right) \right],\\
    \hat{A}_t = \sum_{l=0}^{\infty} (\gamma \lambda)^l \delta_{t+l}.
\end{gather*}
where \( r_t(\theta) \) is the probability ratio between the new and old policies and $\hat{A}_t$ is the advantage function.
The critic loss is the mean square error between the predicted value $V(s_t)$ and the actual return $R_t$, defined as:
\begin{gather*}
    L^{\text{critic}} = \mathbb{E}_t \left[ \left( V(s_t) - R_t \right)^2 \right],\\
\end{gather*}
Algorithmically, this is implemented by gathering trajectories from the environment and estimating the expression above using finite differences
as shown in \hyperref[alg:ppo]{Algorithm~\ref{alg:ppo}}.
\begin{algorithm}
    \begin{algorithmic}[1]
        \Require Policy $\pi_\theta$, value function $V_\phi$, trajectories $\tau$, epochs $K$, batch size $B$
        \State \textbf{Compute Advantage Estimation:}
        \State $\hat{A}_t = \sum_{l=0}^{\infty} (\gamma \lambda)^l \delta_{t+l}$
        \State \textbf{Compute Returns:} $R_t = \hat{A}_t + V(s_t)$
        \State \textbf{Construct dataset} $\mathcal{D} = (s_t, a_t, \hat{A}_t, R_t, \log \pi_\theta(a_t | s_t))$
        \For{epoch $k = 1$ to $K$}
            \For{minibatch $(s, a, \hat{A}, R, \log \pi_{\theta_{\text{old}}}(a | s))$ in $\mathcal{D}$}
                \State \textbf{Policy Update:}
                \State Compute probability ratio:
                \(
                r_t(\theta) = \frac{\pi_\theta(a | s)}{\pi_{\theta_{\text{old}}}(a | s)}
                \)
                \State Compute loss separately for actor and critic:
                \begin{equation*}
                    \begin{aligned}
                        L^{\text{actor}} &= \mathbb{E} \left[ \min(r_t(\theta) \hat{A}_t, \text{clip}(r_t(\theta), 1 - \epsilon, 1 + \epsilon) \hat{A}_t) \right]\\
                        L^{\text{critic}} &= \mathbb{E} \left[ (R_t - V(s_t))^2 \right]\\
                    \end{aligned}
                \end{equation*}
                \State Perform backpropagation and gradient update:
                \[
                    \theta \gets \theta + \alpha \nabla_\theta L^{\text{actor}}, \quad \phi \gets \phi + \alpha \nabla_\phi L^{\text{critic}}
                \]
            \EndFor
        \EndFor
    \end{algorithmic}
    \caption{Actor-Critic with PPO Updates}
    \label{alg:ppo}
\end{algorithm}

Each pass updates both the policy parameters $\theta$ and the value function parameters $\phi$
by applying the usual backward propagation pass for neural networks to minimize the loss function according to a chosen optimizer.
A more detailed explanation of the PPO algorithm can be found in the original paper by Schulman et al. (2017)~\cite{Schulman2017}.
Our implementation of the algorithm is a distributed-parallel version of the original version,
where multiple learners interact with the environment in parallel and share the experience to update the policy and value function.
Short of the V-trace algorithm, this approach is similar to the IMPALA algorithm~\cite{Espeholt2018}.
Besides the proposed approach for simulating dynamic limit order book environments,
we also implement a hybrid neural network for the agent's policy and value functions,
composed of a sequence of self-attention layers to capture the spatial dependencies between the different levels of the LOB,
concatenated with dense layers to process the market features, as discussed in~\autoref{sec:implementation-and-model-description}.

\subsubsection{Benchmark Closed-Form Expression for Simplified Model}
To ensure the agent's performance is able to capture the complexity of the environment, we use a benchmark closed-form expression
to compare the agent's performance against.
We chose the closed-form expression for the optimal bid-ask spread pair as proposed by Avellaneda et al. (2008)~\cite{Avellaneda2008},
which is given by the following expression:
\begin{equation}
    \begin{aligned}
        \delta^* &= \frac{\sigma}{\sqrt{2}} \text{erf}^{-1} \left( \frac{1}{2} \left( 1 + \frac{\mu}{\sigma} \right) \right),\\
        &\text{erf}(x) = \frac{2}{\sqrt{\pi}} \int_{0}^{x} e^{-t^2} dt.
    \end{aligned}
    \label{eq:avellaneda}
\end{equation}
where $\sigma$ is the volatility, $\mu$ is the mean spread, and $\text{erf}^{-1}$ is the inverse error function.
The error function $\text{erf}(x)$ serves simply as a measure of the spread of the normal distribution.

The closed-form expression is derived from a simple market model which assumes normally distributed non-mean reverting spreads,
constant executed order size and exponentially distributed event time dynamics.
As it depends on a simpler model for the market to be implemented, our expectations are that the agent will under-perform in a more complex environment.
The optimal bid-ask spread pair according to \autoref{eq:avellaneda} is then given by $p_\text{bid} = \mu - \delta^*$ and $p_\text{ask} = \mu + \delta^*$.
A fixed quantity of $1$ quoted share as originally proposed is used.



    \section{Implementation and Model Description}
\label{sec:implementation-and-model-description}

As previously stated in \autoref{sec:methodology}, the chosen model follows the Actor-Critic architecture,
with the Actor network learning the policy and the Critic network learning the value function.
For the actor network input, we separate the state space into two tensors: the market features and the LOB data.
The market features contain general high-level information, including the microprice, 10, 15 and 30-period moving averages,
agent inventory, and the Relative Strength Index (RSI) and Order Imbalance (OI) indicators.
The LOB data contains the ordered $N$th best bid-ask price and volume pairs.

We use a sequence of self-attention layers to capture the spatial dependencies between the different levels of the LOB,
while the market features are passed through a dense layer and concatenated with the output of the self-attention layers,
before being passed through the final dense layers as shown in \autoref{fig:actor-architecture}.
We implement the Critic network as a simple feed-forward neural network with two hidden layers, of 128 and 64 units, respectively,
with the same input tensors as the Actor network.

\begin{figure}
    \centering
    \includegraphics[width=.8\columnwidth]{images/policy}
    \caption{Actor Network Architecture}
    \label{fig:actor-architecture}
\end{figure}

To train our policy and value networks, we use the aforementioned PPO algorithm\cite{Schulman2017},
which is a model-free, on-policy approach to optimize the policy directly, as discussed in~\hyperref[subsubsec:gpi]{Subsection~\ref{subsubsec:gpi}}.
Our training loop shown in \hyperref[alg:algorithm]{Algorithm~\ref{alg:algorithm}} consists of collecting trajectories,
computing the Generalized Advantage Estimation (GAE), which is used instead of the usual returns $G_t$.
The chosen approach can be classified as \textbf{online reinforcement learning}, where the agent learns from interactions with the environment,
without the need for a pre-existing dataset, as opposed to offline reinforcement learning, where the agent learns from a static dataset.
Additionally, we use a replay buffer to store the trajectories and sample mini-batches for training the policy and value networks.
Since the replay buffer is discarded immediately after a small fixed number of epochs, the approach is closer to \textbf{on-policy learning},
as the data collected by the policy is the same as the data used for gradient updates.

\begin{algorithm}
    \begin{algorithmic}[1]
        \Require Environment, PPO model, optimizer, number of episodes $num\_episodes$
        \For{each episode in range $num\_episodes$}
            \State Reset environment and observe initial state $s$
            \For{each timestep until episode ends}
                \State Select action $a \sim \pi_{\theta}(s)$
                \State Observe reward $R_t$ and next state $s'$
                \State Store transition $(s, a, r)$ in the trajectory buffer
                \State Set $s \leftarrow s'$
            \EndFor
            \State \textbf{Compute GAE and Returns} // Policy evaluation
            \State \textbf{Update parameters} $\boldsymbol{\theta}$ \textbf{and} $\boldsymbol{\phi}$ // Policy improvement
        \EndFor
    \end{algorithmic}
    \caption{Training Loop}
    \label{alg:algorithm}
\end{algorithm}


    \section{Realized Experiments and Results}
\label{sec:realized-experiments-and-results}

\subsection{Experiment Setup}
\label{subsec:experiment-setup}

A max episode value of 10,000 episodes was used, with each episode consisting on average of 390 observations (or 1 event per corresponding market minute).
Per gathered trajectory, 10 epochs were used for the policy improvement step.
For comparison metrics, we used the Sharpe ratio, the daily return, and the daily volatility,
averaged over 50 episodes after training using the same simulator hyperparameters used for training.

% hyperparam optimization
% --gamma=0.9 --epsilon=0.25 --lambd=0.85 --entropy=0.0012 --lr_policy=3e-4 --lr_value=3e-4  --batch_size=256
We used the Adam optimizer with a learning rate of $3 \times 10^{-4}$ for both the policy and value networks.
The discount factor $\gamma$ was set to 0.9, the GAE parameter $\lambda$ was set to 0.85, and the PPO clipping parameter $\epsilon$ was set to 0.25.
The entropy coefficient was set to 0.0012, and the batch size set to 256 samples per episode/update.
We optimized the hyperparameters using a simple grid search approach.

\subsection{Experiment Results}
\label{subsec:experiment-results}

% Graphs:
% average financial return + confidence interval (+- volatility) x episode number
% average financial return (+- volatility) x current timestep (per 100x trajectories after training)
% average inventory x current timestep (per 100x trajectories after training)
% average reward moving average x episode number

% Table:
% Cols: rl-agent, benchmark agent
% Training
% Rows: Training time
%       3:08:54, , -
%       Time per episode +- std
%       0.8458 \pm 0.1044
%       Mean processing time actor +- std per episode
%       Mean processing time critic +- std per episode
%       Mean financial return +- std

% Mean PnL: -1.174902750662374e-05, Std PnL: 4.761125180139532e-05, Sortino Ratio: -0.5046819237780464
% Mean Stoikov PnL: -0.000423402308920138, Std Stoikov PnL: 0.0010584297592714442, Sortino Ratio: -0.6399344498955829

% Test (after last episode or convergence)
% Rows: Mean financial return +- std
%       Mean Sharpe ratio +- std
%       Agent action latency +- std
%       Mean inventory at market close

To evaluate the financial performance of the trained reinforcement learning agent, we analyzed key metrics such as
financial return, return volatility, and the Sortino ratio.
The results were averaged over 50 episodes after training.

As shown in Table~\ref{tab:test-results}, the reinforcement learning agent exhibited a mean financial return of $-1.174 \times 10^{-5}$
(annualized return of about $-0.3\%$.), an almost neutral performance under adverse market conditions,
while the benchmark agent had a mean financial return of $-0.0004$ (annualized return of about $-10.5\%$),
a clear underperformance under the same conditions.
The return volatility for the RL-agent was lower at $4.7611 \times 10^{-5}$ (annualized volatility of about $1.2\%$),
compared to $0.001$ for the benchmark agent (annualized volatility of about $25.2\%$), indicating more stable financial returns.
Additionally, the Sortino ratio of the RL-agent was $-0.5046$, also outperforming the benchmark's ratio of $-0.6399$.

\begin{table}
    \centering
    \centering
    \small
    \begin{tabular}{|c|c|c|}
        \hline
        \textbf{Training}      & \textbf{Metric}                           \\
        \hline
        Training Time          & $3\text{h}08\text{m}54\text{s}$           \\
        Time per Episode       & $0.8458 \pm 0.1044$ \text{ (s)}           \\
        Processing Time Actor  & $0.0029 \pm 0.001 \text{ (s)}$            \\
        Processing Time Critic & $0.0002 \pm 3 \times 10^{-5} \text{ (s)}$ \\
        \hline
    \end{tabular}
    \caption{Test Results}
    \label{tab:test-results}
    \centering
    \vspace{0.5cm}
    \small
    \begin{tabular}{|c|c|c|}
        \hline
        \textbf{Test}     & \textbf{RL-Agent}       & \textbf{Benchmark} \\
        \hline
        Financial Return  & $-1.174 \times 10^{-5}$ & $-0.0004$          \\
        Return Volatility & $4.7611 \times 10^{-5}$ & $0.0010$           \\
        Sortino Ratio     & $-0.5046$               & $-0.6399$          \\
        \hline
    \end{tabular}
    \caption{Training Results}
    \label{tab:training-results}
\end{table}


% reward.png and returns.png

\begin{figure}
    \centering
    \begin{minipage}{0.45\textwidth}
        \centering
        \includegraphics[width=1\textwidth]{images/reward}
        \caption{Exponential moving average of the training reward per episode, with a linear trend line.}
        \label{fig:average-reward-moving-average}
    \end{minipage}
    \hspace{0.04\textwidth} % Adjust horizontal space between figures
    \begin{minipage}{0.45\textwidth}
        \centering
        \includegraphics[width=1\textwidth]{images/returns}
        \caption{Financial return, averaged over 100 trajectories with a 1 standard deviation confidence interval.}
        \label{fig:average-financial-return}
    \end{minipage}
\end{figure}



    \section{Conclusion}
\label{sec:conclusion}
We have presented the design and implementation of a reinforcement learning agent aiming to show
the effects of adverse market conditions and non-stationary environments on control agents for market-making.
As discussed in~\autoref{subsec:market-model-description-and-environment-dynamics},
our approach for a environment models the dynamics of a limit order book (LOB)
according to a set of parameterizable stochastic processes configured to mimic observed stylized facts in real markets
The resulting market model replicates stylized facts for the midprice, spread, price volatility, and order arrival rate,
as well as the impact of market orders on the agent's inventory, return standard deviation and end-of-day Profit and Loss score.

The fine-controlled dynamics where the agent interacts with the environment allows us to model the effects of market impact and inventory risk on the agent's performance,
and use it to evaluate the agent's performance under adverse market conditions, providing a more realistic training environment for RL agents in market-making scenarios.
The proposed methodology was able to capture the effects of market impact and inventory risk on the trading agent's performance,
and affected the resulting agent's decision-making policies, as shown in~\autoref{subsec:experiment-results},
which we verified by comparing the agent's performance against simpler benchmark agents that
do not account for these effects in their decision-making policies.

Prospective extensions of the presented work and future research on existing market model simulators include
developing hybrid world models to combine both model-based and model-free approaches and leverage the capability of RL agents
to learn from non-stationary environments and historical observations of real markets.
Hybrid world models could further improve the capability of RL agents in adapting to changing market conditions
and could provide a more realistic training environment for market-making agents.


    \bibliographystyle{plain}
    \bibliography{references}

\end{document}
