\subsection{Definição formal do \textit{market making}}

Os mercados normalmente definem uma quantidade mínima por lote ofertado, diferente em cada bolsa (e.g de 100 em 100 ações como no mercado padrão da B3). Um vendedor pode consequentemente enviar uma ordem de venda de 100 ações, mas não de 50. Se os preços ofertados no momento permitirem a execução de um negócio, a transação ocorrerá em cima da quantidade disponível, ou seja, se o mesmo vendedor que oferta a venda de 100 ações fecha um negócio com um comprador buscando 200 ações por exemplo, a quantidade negociada será de apenas 100. O restante da oferta de compra, i.e. 100 ações, continua no livro de oferta de compras, enquanto no lado da venda, a próxima oferta de maior preferência ficará no topo do livro. 

Se definirmos como $a^{0}_{t}$ a oferta de venda com menor preço\footnote{A melhor oferta está na posição 0 da fila, decrescente para vendas e crescente para compras.} no instante $t$ e $b^{0}_{t}$ como a oferta de compra com maior preço, então no cenário em que não ocorre uma transação exatamente no instante $t$, tem-se que $a^{j}_{t} > b^{i}_{t} \ \forall i ,j \in \mathbb{N}$. Além disso, a diferença entre as ofertas $a^{0}_{t} - b^{0}_{t} = \mathbf{\Delta}_t$ é chamado de \textit{spread} no momento $t$.

O objetivo de uma estratégia de \textit{market making} (\textit{MM}) nesse contexto é criar ofertas: 

\begin{itemize}
    \item de venda, representadas por $a^{mm}_{t} < a^{0}_{t}$; ou 
    \item de compra, representadas por $b^{mm}_{t} > b^{0}_{t}$;
\end{itemize}
ou seja, quaisquer ordens criadas pelo agente não geram novas transações no instante \textit{t + 1}.

Para ilustrar melhor o funcionamento do livro de ordens, imagine que um agente qualquer de \textit{MM} tenha em aberto uma oferta de venda pelo preço $a^{mm}_{t}$. Caso surja uma nova oferta de compra com preço $b^{n}_{t} \geq a^{mm}_{t}$ (note que caso isso ocorra também valerá $b^{n}_{t} \geq b^{0}_{t}$, pois o preço da oferta do agente de \textit{MM} sempre está atrás da melhor oferta, que neste caso é $b_{t}^{0}$), isso acarretará em um negócio entre $b^{n}$ e $a^{mm}$ no instante $t$, e a remoção de ambas ofertas do livro de ordens. Em seguida, o agente tem sua posição $r_{t}$ ajustada:
\begin{equation}
    r_{t + 1}:= r_{t} + a^{mm}_{t}\cdot q,
\end{equation}
sendo $r_{t + 1}$ a nova posição do agente e $q$ a quantidade de ativos ofertada em questão.

O agente então escolhe esperar ou realizar uma das ações abaixo:
\begin{enumerate}
    \item inserir uma nova oferta de venda, substituindo a oferta $a_t^{mm}$ anterior;
    \item inserir ou ajustar uma oferta de compra existente no livro de ofertas de compras;
\end{enumerate}

A decisão do agente irá depender de sua expectativa sobre a evolução do mercado, incluindo, mas não limitado à
\begin{itemize}
    \item probabilidade de chegar uma oferta de compra com preço superior $Pr(b_{t + s}^{n} > b_{t}^{mm})$, $s > 0$; \footnote{\label{agent}\textit{n} é outro agente ativo no mercado}
    \item probabilidade de chegar uma oferta de venda com preço inferior $Pr(a_{t + s}^{n} \leq a_{t}^{mm})$, $s > 0$; \footref{agent}
    \item risco futuro da posição ultrapassar os limites estabelecidos pelas corretoras.
\end{itemize}

O retorno $r$ obtido pelo agente após um período $s$ em cima de preços de compra $a$ e $b$ é dado por
\begin{equation*}
    r_{t + s}^{a, b} = q_{t + s}a_{t + s} - q_{t}b_{t} - c, s > 0,
\end{equation*}
sendo $q_{t}$ a quantidade de ações inicialmente compradas, $q_{s + t}$ a quantidade de ações que foram vendidas e $c$ é o custo por transação\footnote{Os custos de transação incluem tipicamente os custos da corretora e emolumentos das bolsas. Custos de custódio de terceiros não pertencem aos cálculos, já que o agente não pretende manter a posição depois do fechamento do pregão.}. Note que $b_{t} = b_{t}^{mm}$ e $a_{t} = a_{t}^{mm}$ representam os preços de compra e venda, respectivamente, que foram executados nas diversas operações. 
O agente de \textit{MM}, por fim, não observa apenas o retorno individual da última operação, mas sim da sua posição acumulada $R_{T}$, que pode ser reconstruída a partir das operações de compra $b\in B_t$ e de venda $a\in A_t$ executadas no instante \textit{t}:
\begin{equation}
    R_{T} = \sum_{t=0}^{T} \sum_{a \in A_t, b \in B_t} r_t
\end{equation}
Define-se matematicamente o valor da posição $R_T$ como o processo de consumo, composto pelos custos de compra, as receitas das vendas e os custos de transação. 

\subsection{Definição do objetivo do agente de MM}
O objetivo principal do agente de \textit{MM} é maximizar o valor esperado da sua carteira (portfólio). Isso é feito pelas etapas intermediárias de maximizar o \textit{bid-ask spread} $\Delta_{t}$ e a quantidade esperada $q_{t}$ de transações realizadas em cima desse \textit{spread}:

\begin{equation}
	\begin{aligned}
		\max_{b \in B_{t}, a \in A_{t}} \quad &  \ \sum_{t=0}^{T} \sum_{a \in A_t, b \in B_t} \mathbb{E} \ [q_{t + s}]a_{t + s} - \mathbb{E} [ q_{t} ] b_{t} - c \\
		= \max_{b \in B_{t}, a \in A_{t}} \quad &  \mathbb{E} \ [R_T] 	\label{eq:target_fct}\\
	\end{aligned}
\end{equation}

Dada a equação acima que define uma função objetivo estocástica para  o problema de maximização do retorno do agente, uma política $\pi$ de escolha de preços será considerada ótima quando dado uma trajetória (histórico de $T$ observações) do livro de ordens $\tau$, consiga maximizar o retorno $R_{T}$ esperado. Este problema pode ser modelado pelo paradigma de Aprendizado por Reforço, especificamente a busca por uma função de valor ótimo a partir do estado atual $s$ do livro de ordens limite:
\begin{equation}
	V^{*}(s) = \underset{\pi}{\max} \ \mathbb{E}_{\tau \sim \pi} \left[ V(\tau) \, | \, s_0 = s \right]
\end{equation}

De modo a tornar o agente adverso ao risco noturno, insere-se também uma restrição adicional, de que ao final do dia não haja exposição a riscos de mercado. 
Existem algumas alternativas para formalizar matematicamente essa restrição:
\begin{enumerate}
    \item No final do dia, o agente não pode ter nenhum ativo em posição: 
    \begin{equation}
        \sum_{b=1}^{B_t} q_b  - \sum_{a=1}^{A_t} q_a = 0\label{eq:eod_restriction}
    \end{equation}
    \item No final do dia, se houver alguma posição, o agente precisa \textit{headgear} a exposição ao risco, comprando (vendendo) futuros ou outros derivativos, dependendo da posição remanescente.\footnote{De maneira simplificada, o \textit{hedge} consiste em comprar ou vender ativos que tenham uma exposição ao risco oposta aos riscos da carteira atual, de modo a equilibrar a posição.}
\end{enumerate}

\subsection{\textit{Market making} simultâneo}
Ao considerarmos a situação em que o MM aplica a sua estratégia em diversos mercados simultaneamente observamos um aumento da complexidade, mas também das alternativas para lidar com riscos envolvidos - chamamos essa situação de MM \textbf{simultâneo} ou \textbf{multivariado}.

Para tal, a posição $V_{T}$ do agente passa a ser a soma de todas posições parciais $V_{T, k}$, onde $k$ é uma bolsa onde o agente possui ações:
\begin{eqnarray*}
    V_t &=& \sum_{k=1}^N V_{T, k}\\
\end{eqnarray*}

O objetivo principal (\ref{eq:target_fct}) continua o mesmo, e mantém-se a restrição (\ref{eq:eod_restriction}). Contudo, surgem novas alternativas para proteção da carteira durante a noite:

\begin{enumerate}
    \item O agente pode avaliar o risco global da carteira, e incluir um único ativo de proteção contra o risco global ao final do dia.
    \item Se o ativo for negociado em múltiplas bolsas (chamado também de ativo \textit{co-listed}), o agente pode continuar a negociação deste em outra bolsa caso uma delas esteja fechada.
\end{enumerate}

Considerando um cenário em que haja ações em \textit{co-listing}, surge a possibilidade de criação de estratégias mais sofisticadas, permitindo a implementação dos itens mencionados acima.
 