\subsection{Definição formal do \textit{market making}}

Os mercados normalmente definem uma quantidade mínima por lote ofertado, diferente em cada bolsa (e.g de 100 em 100 ações como no mercado padrão da B3). Um vendedor pode consequentemente enviar uma ordem de venda de 100 ações, mas não de 50. Se os preços ofertados no momento permitirem a execução de um negócio, a transação ocorrerá em cima da quantidade disponível, ou seja, se o mesmo vendedor que oferta a venda de 100 ações fecha um negócio com um comprador buscando 200 ações por exemplo, a quantidade negociada será de apenas 100. O restante da oferta de compra, i.e. 100 ações, continua no livro de oferta de compras, enquanto no lado da venda, a próxima oferta de maior preferência ficará no topo do livro. 

Se definirmos como $a^{0}_{t}$ a oferta de venda com menor preço\footnote{A melhor oferta está na posição 0 da fila, decrescente para vendas e crescente para compras.} no instante $t$ e $b^{0}_{t}$ como a oferta de compra com maior preço, então para um mercado sem oportunidade de negócio no mesmo instante $t$, tem-se que $a^{j}_{t} > b^{i}_{t}$. Além disso, a diferença entre as ofertas $a^{0}_{t} - b^{0}_{t} = \mathbf{bas}_t$ é chamado de \textit{spread} no momento $t$.

O objetivo de uma estratégia de \textit{market making} (MM) nesse contexto é inserir ordens no livro de ofertas no momento atual $t$, podendo ser: 

\begin{itemize}
    \item ofertas de venda representada por $a^{mm}_{t} < a^{0}_{t}$; ou 
    \item ofertas de compra com $b^{mm}_{t} > b^{0}_{t}$;
\end{itemize}
sendo que quaisquer ordens criadas não devem iniciam novas transações.

Para ilustrar melhor o funcionamento do livro de ordens, imagine que um agente qualquer de \textit{MM} tenha em aberto uma oferta de venda pelo preço $a^{mm}_{t}$. Caso surja uma nova oferta de compra com preço $b^{n}_{t} \geq a^{mm}_{t}$ (note que caso isso ocorra também valerá $b^{n}_{t} \geq b^{0}_{t}$, pois o preço da oferta do agente de \textit{MM} sempre está atrás da melhor oferta, que é a oferta $b_{t}^{0}$), isso acarretará em um negócio entre os dois agentes, e a remoção de ambas ofertas $b_{t}^{n} \text{ e } a_{t}^{mm}$ do livro de ordens. Em seguida, o agente tem sua posição $V_{t}$ ajustada:
\begin{equation}
    V_{t + 1} \leftarrow V_{t} + a^{mm}_{t}\cdot q,
\end{equation}
sendo $V_{t + 1}$ a nova posição do agente e $q$ a quantidade de ativos ofertada em questão.

O agente então escolhe esperar ou realizar uma das ações abaixo:
\begin{enumerate}
    \item inserir uma nova oferta de venda, substituindo a oferta $a_t^{mm}$ anterior;
    \item inserir ou ajustar uma oferta de compra existente no livro de ofertas de compras;
\end{enumerate}

A decisão do agente irá depender de sua expectativa sobre a evolução do mercado, incluindo, mas não limitado à
\begin{itemize}
    \item probabilidade de chegar uma oferta de compra com preço superior $b_{t + s}^{n} > b_{t}^{mm}$, $s > 0$; \footnote{\label{agent}\textit{n} é outro agente ativo no mercado}
    \item probabilidade de chegar uma oferta de venda com preço inferior $a_{t + s}^{n} \leq a_{t}^{mm}$, $s > 0$; \footref{agent}
    \item risco futuro da posição ultrapassar os limites estabelecidos pelas corretoras.
\end{itemize}

Por exemplo, um agente esperando a chegada de uma oferta de venda que possa casar com sua oferta de compra $b_{t}^{mm}$ decide não ajustar a sua ordem atual e aguardar a execução, lucrando assim o \textit{BAS} menos os custos de transação:
\begin{equation*}
    v_{t + s} = q \cdot \left(a_t^{mm} - b_t^{mm}\right) - c, s > 0,
\end{equation*}
sendo que $q > 0$ a quantidade de ações negociadas em questão, e $v_{t + s}, s > 0$ representa o resultado das operações do agente após a execução de uma ordem e $c$ o custo por transação.\footnote{Os custos de transação incluem tipicamente os custos da corretora e emolumentos das bolsas. Custos de custódio de terceiros não pertencem aos cálculos, já que o agente não pretende manter a posição depois do fechamento do pregão.} Além disso, nem sempre pode-se esperar que uma venda ocorre em sequência a uma compra (e vice-versa), ou que as execuções ocorram imediatamente.

Portanto, o agente de \textit{MM} não observa o resultado individual da última operação, mas sim a sua posição total $V_{t}$, que pode ser reconstruída a partir das operações de compra $b\in B_t$ e operações de venda $a\in A_t$ executadas até o momento \textit{t}: \footnote{Para fins expositivos, usamos a notação $P_s^{ask_0}$ para representar o melhor preço de venda e $P_s^{bid_0}$, mas o leitor deve se atender à possibilidade que a quantidade negociada ($q_b$, $q_a$) poderia ultrapassar a quantidade exposta no livro. Nessa situação, o preço executado seria um preço médio das operações realizadas até a execução total da quantidade. }
\begin{equation}
    V_{t} := \sum_{b \in B_t} q_{b_{t}} \cdot b_{t}^{0} - \sum_{a \in A_t} q_{a_{t}} \cdot a_t^{0} - c
\end{equation}
sendo que $\sum_{b=1}^{B_t} q_b \cdot b_{t}^{mm} $ representa o valor das posições compradas, avaliadas na data $t$  usando o preço de compra observado no instante; $\sum_{a=1}^{A_t} q_a \cdot a_{t}^{mm}$ representa o valor das posições vendidas na data $t$, usando o possível preço de recompra no instante, e $V_t$ é o processo de consumo, que se compõe pelos custos de compra, as receitas das vendas e os custos de transação. Note que $b_{t}^{mm}$ e $a_{t}^{mm}$ representam os preços de compra e venda, respetivamente, que foram executados nas diversas operações. 

\subsection{Definição do objetivo do agente de MM}
O objetivo principal do agente de \textit{MM} é maximizar o valor da sua carteira (portfólio). Isso é feito pelas etapas intermediárias de maximizar o \textit{BAS} e a quantidade de transações realizadas:

\begin{equation}
\begin{aligned}
\max_{b \in B_{t}, a \in A_{t}} \quad & V_T \label{eq:target_fct}\\
% \textrm{s.t.} \quad & y_{i}(w\phi(x_{i}+b))+\xi_{i}-1\\
  % &\xi\geq0    \\
\end{aligned}
\end{equation}

O agente também tem uma restrição adicional, de que ao final do dia não esteja exposto a nenhum risco de mercado. 
Existem algumas alternativas para formalizar essa restrição:
\begin{enumerate}
    \item No final do dia, o agente não pode ter nenhum ativo em posição: 
    \begin{equation}
        \sum_{b=1}^{B_t} q_b  - \sum_{a=1}^{A_t} q_a = 0\label{eq:eod_restriction}
    \end{equation}
    \item No final do dia, se houver alguma posição, o agente precisa \textit{headgear} a exposição ao risco, comprando (vendendo) futuros ou outros derivativos, dependendo da posição remanescente.\footnote{A atividade de \textit{hedge}, de forma informal, consiste em comprar ou vender ativos que tenham uma exposição ao risco de tal forma que compense os riscos da carteira atual.}
\end{enumerate}

\subsection{\textit{Market making} simultâneo}
Ao considerarmos a situação em que o MM aplica a sua estratégia em diversos mercados simultaneamente observamos um aumento da complexidade, mas também das alternativas para lidar com riscos envolvidos - chamamos essa situação de MM \textbf{simultâneo} ou \textbf{multivariado}.

O valor da posição do agente passa a ser a soma de todos $N$ ativos:
\begin{eqnarray*}
    V_t &:=& \sum_{i=1}^N V_{t,i}\\
    V_{t,i} &:=& \sum_{b_{i}=1}^{B_{i}} q_{b_{i}} \cdot b_{t}^{0} 
    \sum_{a_{i}=1}^{A_{i}} q_{a_{i}} \cdot a_{t}^{0} - c
\end{eqnarray*}

O objetivo principal (\ref{eq:target_fct}) continua o mesmo, e mantém-se a restrição (\ref{eq:eod_restriction}). Contudo, surgem novas alternativas para proteção da carteira durante a noite:

\begin{enumerate}
    \item O agente pode avaliar o risco global da carteira, e incluir um único ativo de proteção contra o risco global ao final do dia.
    \item Se o ativo for negociado em múltiplas bolsas (chamado também de ativo \textit{co-listed}), o agente pode continuar a negociação deste em outra bolsa caso uma delas esteja fechada.
\end{enumerate}

Considerando um cenário em que haja ações em \textit{co-listing}, surge a possibilidade de criação de estratégias mais sofisticadas, permitindo a implementação dos itens mencionados acima.
 