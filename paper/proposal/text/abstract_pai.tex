Este projeto de pesquisa tem como foco a análise de uma estratégia de \textit{Market-Making} (MM) no mercado financeiro usando técnicas de Aprendizado por Reforço (RL). 

Uma estratégia de MM insere uma oferta de compra no topo da lista das compras (\textit{Bid}) e uma oferta de venda no topo da lista de vendas (\textit{Ask}), sendo que o preço de compra sempre é menor que o preço de venda, i.e. o \textit{Bid-Ask-Spread} (BAS) é positivo.

Se as duas ordens fossem executadas sem movimentação de preço, o agente financeiro lucra o BAS menos os custos de transação. 
O agente financeiro tem como objetivo executar essa transação de compra-venda diversas vezes durante o dia e acabar o dia, no final do pregão, sem posição remanescente.

O risco associado com essa estratégia é que não há garantia que o agente consiga executar a ponta de compra e a de venda imediatamente.
Dependendo do processo de chegada de ofertas de compras e vendas, não só pode ser que o agente fecha um cíclo com um resultado negativo, muitas vezes pode ser impossível fechar os ciclos.

Ao longo do tempo, o agente financeiro corre o risco de acumular posições compradas ou vendidas, especialmente se o mercado se movimenta numa única direção. Para não perder dinheiro, ele vai ter que gerenciar seu estoque de posições, tomando ao longo do tempo decisões sobre como e com qual quantidade ele quer ofertar compras ou vendas.

A presente pequisa aplica RL para otimizar o controle do estoque do agente financeiro, levando em conta seu objetivo de maximizar lucro, e considerando a restrição que no final do dia o estoque tem que ser zero. 

Além disso, incluímos alguns aspectos adicionais que são relevantes para a aplicação real.
Primeiro, analisamos como um estratégia de MM pode ser calibrada e testada no mercado, já que o agente interfere no mercado - ou seja, ele não é um tomador de preço (\textit{price-taker}), mas o formador de preço (\textit{price-maker}). Portanto, ele muda a amostra histórica disponível para \textit{backtests}.
Segundo, o agente normalmente opera com diversos ativos em paralelo, ou seja ele controla um estoque de ativos que têm uma estrutura multi-dimensional, com alguma estrutura de dependência entre as distribuições de retornos. Essa estrutura pode ser usada para uma gestão mais eficiente do risco do estoque.

Ao final deste estudo, esperamos desenvolver um agente de RL capaz de definir uma estratégia dinâmica de oferta de compra e vendas em múltiplos ativos, que se adapte aos processos de chegada de ofertas e execuções do mercado, controlando o risco associado às posições mantidas após o fechamento do mercado.

Um aspecto desafiador adicional na nossa pesquisa é o fato que os dados do mercado para analisar são de alta frequência o que demanda um pré-tratamento dos dados históricos custos. A velocidade da chegada dos dados impacta também na solução da estratégia ótima de gestão de estoque, já que o intervalo entre chegada de um novo dado e tomada de decisão é limitado.

Este projeto representa uma contribuição valiosa para a área de finanças quantitativas, oferecendo uma abordagem inovadora para abordar um portfólio de operações de \textit{Market Making}. A aplicação de técnicas de RL nesse contexto pode fornecer soluções práticas e eficazes para a gestão de portfólio em ambientes de alto risco.