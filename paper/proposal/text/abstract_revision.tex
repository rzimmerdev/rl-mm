Nessa persquisa, esperamos criar uma estratégia de \textit{Market-Making} (MM) que ao ser aplicada ao mercado financeiro cria diversas ordens de limite com o intuito de lucrar com as diferenças de preços. Nosso objetivo é que essa estratégia gere um agente de MM capaz de maximizar o retorno, e ao mesmo tempo, reduzir o risco associado ao inventário  (em questão) mantido e as ordens existentes, principalmente após o período de fechamento do mercado. A literatura ( citar autores) mostra que há diversas políticas de controle de risco que reduzem o risco diário da estratégia associada, mas há poucas que tratam do risco noturno (risco que decorre do período após o fechamento do mercado). O presente projeto de pesquisa  tem como propóstio realizar uma análise das técnicas existentes de Aprendizado por Reforço, especialmente aquelas que possam ser utilizadas para otimizar as políticas de controle de inventário de um agente financeiro de MM. O principal objetivo é  maximizar o lucro sob a restrição de zerar seu inventário ao final do dia.


 O presente projeto de pesquisa tem como propósito realizar uma análise das técnicas existentes de Aprendizado por Reforço - especialmente aquelas que possam ser utilizadas para otimizar as políticas de controle de inventário de um agente financeiro de MM - com a finalidade de desenvolver estratégia de Market-Making (MM) para uso no mercado financeiro. 
O principal objetivo das estratégias de Market-Making (MM) é  maximizar o lucro sob a restrição de zerar um inventário ao final pregão nas bolsas de valores. Uma estratégia de MM busca criar ordens de limite, que proveem liquidez ao mercado, de modo a lucrar com a diferença entre o preço de bid (Melhor preço de compra) e o preço de ask (Melhor preço de venda) denominada Bid-Ask-Spread (BAS). Por definição, se uma ordem limite é executada, o agente que a criou lucra o resultado da diferença do valor do BAS e dos custos da transação. 
Um agente financeiro tem como objetivo maximizar o lucro do investimento na mesma medida que minimiza a risco a que está exposto no mercado, dadas as suas ordens e a sua posição atual. Ao longo do dia, várias transações de compra-venda podem ocorrer, mas ao final do pregão a negociação se encerra. Por isso, a expectativa é que possamos desenvolver uma estratégia que permita ao agente criar ordens prévias de modo a finalizar o dia sem posição remanescente, zerando assim seu risco noturno. Visamos uma estratégia que elimine a possibilidade de que o processo de chegada de ofertas de compra e venda se encerre antes que o agente tenha  tenha suas ordens executadas, ou até mesmo, que feche o dia com um retorno negativo. O valor da estratégia de MM proposta nesse projeto de pesquisa de dá pela pretensão de eliminar o risco associado com a dificuldade de executar as ordens de limite nas pontas de compra e venda, antes do fechamento do pregão.
Nessa persquisa, esperamos criar uma estratégia de \textit{Market-Making} (MM) que ao ser aplicada ao mercado financeiro cria diversas ordens de limite com o intuito de lucrar com as diferenças de preços. Nosso objetivo é que essa estratégia gere um agente de MM capaz de maximizar o retorno, e ao mesmo tempo, reduzir o risco associado ao inventário  (em questão) mantido e as ordens existentes, principalmente após o período de fechamento do mercado. A relevância dessa estratégia se sustenta na literatura ( citar autores) que mostra haver diversas políticas de controle de risco que reduzem o risco diário da estratégia associada, mas haver poucas que tratem do risco noturno (risco que decorre do período após o fechamento do mercado). 

