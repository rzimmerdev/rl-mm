
Espera-se que a estratégia de Market Making (MM) apresente os seguintes resultados:

A estratégia deve maximizar o valor da carteira do agente ao final do dia, levando em conta todas as transações, custos de transação e flutuações nos preços dos ativos. Ao mesmo tempo, a estratégia deve minimizar o risco overnight, mantendo exposições limitadas a movimentos adversos de preços após o fechamento do mercado.

A capacidade de adaptação às expectativas de mercado é fundamental, e o agente MM deve ser capaz de inserir e ajustar ofertas com base em probabilidades percebidas de eventos futuros.

Além disso, a estratégia deve contribuir positivamente para a liquidez do mercado, facilitando a execução de negócios para outros participantes. Isso, no entanto, deve ser equilibrado com considerações de custos de transação, incluindo comissões de corretagem e outros custos associados à execução de negócios.

Na segunda parte da pesquisa, a estratégia de MM em mercados multidimensionais deve otimizar o valor da carteira considerando múltiplos ativos e diferentes possibilidades de alocação de recursos, enquanto também gerencia o risco global da carteira.



2 Metodologia
Para atingir os objetivos deste projeto, definiram-se 5 etapas. Cada uma das etapas
utilizou diferentes métodos científicos, como o protocolo de mapeamento sistemático, análise
temática, estudos de campo e o Evidence-based Timeline Retrospectives. Para garantir
o rigor da revisão de literatura, decidiu-se adotar etapas do protocolo de mapeamento
sistemático da literatura (MSL), por ser apropriado para lidar com áreas amplas e mal
definidas (PETERSEN et al., 2008; KITCHENHAM; CHARTERS, 2007), sendo este o
caso da literatura relacionada a UX em startups. Para analisar os artigos encontrados
na literatura, adotou-se a análise temática (BRAUN; CLARKE, 2006), por ser uma
abordagem flexível para analisar dados qualitativos. Em relação aos estudos de campo,
30 Capítulo 1. Introdução
devido ao cenário de pandemia de Covid-19 que acometeu o mundo no início de 2020,
a maior parte das empresas de tecnologia começaram a atuar remotamente. Visto que
não era possível fazer observações presenciais, decidiu-se utilizar um método que não se
baseasse apenas em entrevistas, mas que trouxesse uma maior profundidade na coleta dos
dados. Chegou-se então ao “Evidence-based Timeline Retrospective“ (EBTR) proposto por
(BJARNASON et al., 2014), um método com diversas etapas de conversas e interação
com os membros da empresa e que possui como diferencial as retrospectivas baseadas em
evidências coletadas pelo pesquisador através de entrevistas com os membros das empresas.
A metodologia proposta para alcançar o objetivo deste projeto é ilustrada na Figura
1, descrevendo em detalhes cada uma das etapas definidas:
Figura 1: Visão geral da metodologia utilizada para guiar o projeto de mestrado.
Etapa A
Revisão bibliográca
Etapa E
Comunicação dos
resultados
Etapa D
Panorama entre
literatura e prática
Literatura Prática
Estudos de campo
Análise dos estudos
Etapa C
Mapeamento da
literatura
Análise temática
Etapa B
Fonte: elaborado pelo autor.
Etapa A - Investigação na literatura: realização de um estudo bibliográfico inicial
de forma a obter uma visão geral dos tópicos User eXperience e startups e também dos
estudos que relacionam ambos. Esta atividade foi incremental, já que novas referências
importantes surgiram ao longo do projeto.
Etapa B - Análise temática da literatura: condução de uma investigação do estado
da arte através do mapeamento da literatura (PETERSEN et al., 2008), analisado pelas
lentes da Análise Temática (CRUZES; DYBA, 2011). O objetivo foi identificar os principais
temas relacionados a prática de UX em startups de software relatadas pela literatura. Os
resultados obtidos nesta etapa foram submetidos ao journal Information and Software
Technology1
.
Etapa C - Estudos de campo: condução de observações em startups de software através
de tecnologias de comunicação virtual. O objetivo dos estudos de campo foi entender como
1 <https://www.journals.elsevier.com/information-and-software-technology>
1.3. Contribuições 31
o trabalho de UX é feito na prática em startups e quais são suas necessidades de UX.
Nesta etapa, também foi elaborado e enviado um projeto ao Comitê de Ética2
(ver no
Apêndice G), de forma a validar as medidas tomadas para a integridade dos participantes
envolvidos nesta pesquisa.
Etapa D - Paralelo entre literatura e prática: os temas da literatura da prática
de UX em startups foram comparados de modo a entender se os tópicos tratados na
literatura estão alinhados com a prática. Além disso, também foi possível entender o que
está acontecendo na prática dentro das startups que a literatura ainda não abordou.
Etapa E - Comunicação dos resultados: os resultados obtidos tanto pela Análise
Temática da literatura, como pelos estudos de campo e análises, foram divulgados através
de um artigo científico submetido a revista Information and Software Technology (IST),
planeja-se a escrita de dois artigos científicos nos próximos meses, comunicando os resultados
dos estudos de campo e sobre a metodologia utilizada. Alguns outros potenciais veículos
para essas publicações são: Journal of Systems and Software (JSS) e Simpósio Brasileiro
de Engenharia de Software (SBES)