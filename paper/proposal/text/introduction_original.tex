A tomada de decisões nos mercados, especialmente em atividades como market making e operações de alta frequência, exige a consideração de múltiplos fatores e a capacidade de se adaptar rapidamente a mudanças nas condições do mercado. 
No entanto, tais operações frequentemente expõem os participantes a riscos substanciais, particularmente durante o período noturno, quando os mercados estão fechados e as informações disponíveis são limitadas, assim como há necessidade de esperar até a abertura do mercado para que sejam refletidas. % Explicar melhor como surge e por que existe o risco overnight.

Este projeto de pesquisa propõe abordar o desafio de minimizar o risco overnight na negociação financeira, especificamente no contexto de market making. O market making é uma estratégia que envolve a postagem contínua de ordens de compra e venda em um livro de ordens de limite (\textit{limit order book}), ou seja, fora do melhor preço de compra ou venda para fornecer liquidez ao mercado e gerar lucros a partir da diferença entre os preços de compra e venda.

Inspirado por avanços recentes em Aprendizado por Reforço (RL), como nos estudos de  \cite{Gasperov2021} e \cite{Ganesh2019},  nossa pesquisa busca desenvolver um agente de RL capaz de otimizar a alocação de recursos de forma a zerar o portfólio antes do fechamento do mercado. Essa abordagem representa uma mudança significativa em relação às estratégias tradicionais de manutenção de posições durante a noite.

O agente de RL será treinado em ambientes simulados, permitindo-lhe aprender a tomar decisões que minimizem o risco overnight, ao mesmo tempo em que busca oportunidades para lucrar com transações ao longo do dia. A formulação de recompensas adequadas assim como dos parâmetros do ambiente de simulação desempenhará um papel fundamental nesse processo, incentivando o agente a tomar ações que estejam alinhadas com o objetivo de proteger o capital investido e que reflitam o comportamento adequado para cenários reais.

A importância deste projeto reside na sua capacidade de contribuir para a gestão de risco em operações financeiras, oferecendo uma abordagem inovadora que busca equilibrar a busca de lucros com a redução do risco. A aplicação de técnicas de RL nesse contexto pode oferecer soluções práticas e eficazes para investidores e instituições financeiras que buscam melhorar a eficiência de suas estratégias atuais. Este projeto é relevante para a comunidade de finanças quantitativas, uma vez que tem o potencial de fornecer uma abordagem sobre a aplicação do Aprendizado por Reforço em ambientes financeiros de alta frequência, ajudando a moldar estratégias de negociação mais robustas e eficazes.
