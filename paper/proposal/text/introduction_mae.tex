\textbf{Os Market Makers automatizados são agentes algorítmicos que fornecem liquidez nos mercados eletrônicos (Abraham Othman, 2012).http://reports-archive.adm.cs.cmu.edu/anon/2012/CMU-CS-12-123.pdf} O uso de algoritmos computacionais para negociações financeiras é uma prática crescente no mercado financeiro, existindo diversas técnicas e modelos que podem ser implementados com essa finalidade e para outros propósitos, entre eles maximizar retornos, minimizar riscos. A presente proposta de pesquisa busca estudar o uso de agentes \textbf{de tomada de decisão para \textit{market making} aplicados a mercados financeiros}, que operam sobre diversos ativos em paralelo considerando um universo multi-variado.  


Há diversos trabalhos na área de computação financiera que discutem e propõem estratégias intra-diárias que buscam \textbf{lucrar com a diferença do \textit{bid-ask-spread} (BAS)}. Comumente, as técnicas de gerenciamento de risco usadas por tais estratégias são aplicáveis a posições mantidas durante o funcionamento do mercado. Quando um agente de \textit{MM} oferece simultaneamente compras e vendas no mercado, ele corre o risco de somente ser executado numa das pontas. No caso mais simples, pode acontecer que ele somente execute uma venda e nunca mais aconteça outra operação de compra. Consequentemente, o \textit{MM} ficaria com uma posição vendida no estoque até o final do dia e, após o fechamento do mercado, os participantes não tenham mais a possibilidade de negociar ordens.

A literatura sobre finanças (\textbf{citar textos para esse argumento) }aponta que qualquer posição mantida durante a noite é consequentemente vulnerável a eventos inesperados - tais como notícias financeiras, eventos macroeconômicos e outros - pois o mercado não permitiria negociações enquanto estiver fechado. Consequentemente, podem ocorrer aberturas de mercado altamente voláteis que afetem negativamente as posições do Agente. Nesse cenário, encontrar o equilíbrio entre minimizar o risco de posições noturnas e maximizar o retorno das operações diárias é um desafio crítico para agentes de MM. 

Considerando essa dificuldade do mercado em evitar o risco noturno, buscamos desenvolver uma proposta de agente MM que venha sanar essa vulnerabilidade existente no período de inatividade do mercado. Com esse propósito, visamos desenvolver um Agente MM capaz de maximizar seu retorno intra-diário, sob a restrição de haver uma posição zerada até o momento de fechamento do mercado. 


Nosssa proposta consiste em considerar diferentes ativos em paralelo nesse cenário de risco noturno de mercado introduzindo uma novidade no campo de pesquisa financeiro-computacional. Sabemos que em um cenário típico de mercados financeiro, os agentes que hoje trabalham com uma estratégia MM (tipicamente \textit{hedge-funds}) raramente se restringem a um único ativo. Dessa forma, uma visão de portfólio diversificado na gestão do risco das posições intra diárias é um avanço extremamente importante para o tratamento do risco noturno. ( é importante pra quem ou quem)

Além do foco em minimizar o risco \textit{overnight}, buscamos pesquisar e obter soluções práticas para sanar outras dificuldades que se somam a esse problema principal, os quais contribuem para aumentar o risco e estão intrinsecamente relacionados a ele. A saber:
\begin{itemize}
    \item Os dados histórico \underline{estáticos} usados para calibrar e avaliar as estratégias de MM são estáticos e não permitem atualizar as transformações de mercado no momento em que ocorrem;
    \item OS dados de alta frequência \underline{alta frequência} influenciam de forma não previsível as decisões do agente.
\end{itemize}


Pesquisas relacionadas na área de trading algorítmico usam, em muitos casos, dados históricos para calibrar e testar hipóteses - o que é chamado de \textit{backtesting}. Assim sendo, temos como primeiro desafio determinar uma metodologia capaz de permitir a validação das nossas hipóteses e políticas de agente. Nesse caso, observamos que as ações (nesse caso as transações) de um agente de MM geram mudanças no estado do mercado, pois um agente de MM não é apenas um consumidor de preços (\textit{price-taker}), mas um fornecedor de preços (\textit{price-maker}). Isso significa que sua interação com o mercado altera a amostra histórica, impedindo o uso de técnicas de \textit{backtest} regulares.

De fato, não se pode descartar
a possibilidade que os outros agentes reajam às ações do agente original: imaginemos o cenário em que outro agente de \textit{MM} já esteja atuando no mercado, some a isto a possibilidade que esse agente vá continuar tentando se manter no nível das melhores ofertas, de modo a ter suas ordens executadas primeiro. Assim sendo, as ordens que nosso agente enviar afetarão os valores subsequentes de outros agentes de forma que não poderíamos prever os resultados dessa interação e suas consequências, sendo possível que se configurem situações de perda para os agentes.

O segundo aspecto principal proposto nesse projeto de pesquisa é o fato de que os dados do mercado em análise são de alta frequência, o que demanda um pré-tratamento dos dados históricos e dos dados atuais que vão chegando no transcorrer dos dias de negociação. Por isso, cada instante de tempo pede que seja construído o livro de ofertas, refletindo novas ofertas de compra e venda, excluindo cancelamentos e eventuais ajustes de preços de ofertas pré-existentes. Assim, dessa forma, \textbf{\textit{Quando uma ordem é executada, observa-se no mercado a chegada de uma oferta que ultrapassa o melhor preço do outro lado do livro imediatamente levando ao desaparecimento das ordens afetadas.}}
Dito isto, conclui-se que a velocidade da chegada dos dados impacta na solução da estratégia ótima de gestão de estoque, já que o intervalo entre a chegada de um novo dado e a tomada de decisão é curto.

Ao final deste estudo, esperamos obter um agente que execute uma estratégia de \textit{MM} ótima em múltiplos ativos, que se adapte aos processos de chegada de ofertas e execuções do mercado, assim controlando o risco associado às posições mantidas após o fechamento do mercado.

