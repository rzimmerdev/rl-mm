A pesquisa tem como objetivo central a criação de uma estratégia de market making para mercados financeiros de alta frequência. Nesse contexto, planejamos obter tanto um método de otimização para a política de preços de compra e venda, como também a criação de um ambiente de simulação em tempo real do livro de ofertas limite e de outros agentes. 

A pesquisa focará em aplicações de técnicas de aprendizado de reforço para modelar o comportamento do agente de market making e obter a política ótima. Isso inclui a definição de cadeias de Markov de estado, de recompensa e de decisão. 
Em cenários onde a calibragem de parâmetros é necessária, serão realizados ajustes dos mesmos ao longo do tempo, de modo a levar em consideração as condições do mercado e as expectativas do agente.

Como conclusão da pesquisa, pretendemos realizar uma avaliação abrangente do desempenho da estratégia de market making, incluindo mas não limitado a análise do retorno da carteira ao longo do tempo, levando em consideração custos de transação e flutuações nos preços dos ativos. Avaliaremos também a eficácia da estratégia sob a restrição de risco \textit{overnight} máximo e o impacto das ordens geradas para zerar a carteira na liquidez do mercado. Após o  treinamento do agente, realizaremos também uma análise comparativa entre a estratégia de market making desenvolvida com técnicas de aprendizado por reforço e estratégias tradicionais de mercado, especificamente de \textit{price-taking}. Isso nos permitirá destacar as vantagens e desvantagens da abordagem de aprendizado de reforço e comparar os resultados obtidos quantitativamente.
