A pesquisa tem como objetivo central a criação de uma estratégia específica de \textit{market making} ótima. Nesse contexto, planejamos obter tanto um método para otimização da política de escolha de preços, como também da criação do ambiente de simulação para o livro de ofertas limite e de outros agentes participantes do mercado.

O foco principal é a aplicação de técnicas de aprendizado por reforço para modelar o comportamento do agente de market making e inserir uma restrição essencial para minimizar o risco do agente. Em cenários onde a calibragem de parâmetros é necessária, serão realizados ajustes dos mesmos ao longo do tempo, de modo a levar em consideração as condições do mercado e as expectativas do agente.

Como conclusão da pesquisa, pretendemos realizar uma avaliação abrangente do desempenho da estratégia de market making, incluindo mas não limitado a análise do retorno da carteira ao longo do tempo, levando em consideração custos de transação e flutuações nos preços dos ativos. Avaliaremos também a eficácia da estratégia sob a restrição de risco \textit{overnight} máximo e o impacto das ordens geradas para zerar a carteira na liquidez do mercado. Após o treinamento do agente, realizaremos também uma análise comparativa entre a estratégia desenvolvida e estratégias tradicionais de mercado, especificamente de \textit{price-taking}. Isso nos permitirá destacar as vantagens e desvantagens da abordagem de aprendizado de reforço e comparar estatisticamente os resultados obtidos.
