Confira essa referencia bibliografica

https://www.cs.cmu.edu/~sandholm/liquidity-sensitive%20automated%20market%20maker.teac.pdf

http://reports-archive.adm.cs.cmu.edu/anon/2012/CMU-CS-12-123.pdf

https://dblp.org/pid/91/6962.html


https://repositorio.ufscar.br/handle/ufscar/629

Os mercados normalmente definem uma quantidade mínima por lote ofertado (e.g de 100 em 100 ações como no mercado padrão da B3), cujas quantidades ofertadas podem variar. Dessa forma é possível que um vendedor quer vender 100 ações e um comprador quer comprar 200 ações. Se os preços ofertados permitem a execução de um negócio, a transação se limitaria à máxima quantidade possível, ou seja 100 ações nesse caso. O restante da oferta de compra, i.e. 100 ações, continua no livro de oferta de compras, enquanto no lado da venda, a próxima oferta de maior preferência ficará no topo do livro.

Como o \textit{matching engine} casa todas as ofertas de compra e venda que tenham preços cruzados (preços de compra maiores que preços de vendas), ele somente acaba de criar negócios quanda todos os preços de compras estiverem menores que todos os preços de venda. 
Se definimos como $P_t^{ask_1}$ o preço da oferta de venda com menor preço \footnote{Ou seja, essa oferta está na posição 1 da fila de venda (\textit{ask}).} no instante $t geq 0$ e $P_t^{bid_1}$ como o mais alto preço ofertado na compra, então valeria num estado do mercado sem oportunidade de negócio que $P_t^{ask_j} > P_t^{bid_i} \forall i \in I, j \in J$ onde $J$ e $I$ representam os conjuntos de ofertas de venda e compra, respetivamente.

A diferença entre as melhores ofertas, $P_t^{ask_1} - P_t^{bid_1} = \mathbf{bas}_t$ é chamado de \textit{bid-ask-spread} na data $t$.

O objetivo da estratégia de \textit{market making} (MM) é inserir nesse livro de ofertas em $t$ uma oferta de venda com $P_t^{ask_mm} < P_t^{ask_1}$ e uma outra oferta de compra com $P_t^{bid_mm} > P_t^{bid_1}$, mas de tal forma que não haja execução de negócio, ou seja:
\begin{equation*}
    \forall{i \in I}, \forall{j \in J}:  P_t^{bid_i} < P_t^{bid_mm} < P_t^{ask_mm} < P_t^{ask_j}
\end{equation*}
    
Se nessa situação aparecer uma oferta de compra com preço $P_t^{compra} \geq P_t^{ask_mm}$, haverá um negócio fechado e as duas ofertas serão retiradas do livro de ofertas. Além disso, o agente de MM terá sua posição de ativos $V_t, t> 0$ ajustada:
\begin{equation*}
    V_t:= V_{t-} + P_t^{ask_mm}\cdot q_t^{ask_mm}, t > 0,
\end{equation*}
sendo $V_{t-}$ a posição do agente antes do instante $t$ e $q_t^{ask_mm}$ a quantidade de ativos ofertada pelo preço $P_t^{ask_mm}$.

Depois desse evento, o agente tem que tomar uma decisão o que fazer em seguida:
\begin{enumerate}
    \item insere uma nova oferta de venda, substituindo a oferta $P_t^{ask_mm}$ anterior;
    \item ajusta a oferta de compra ainda existente no livro de oferta de compras;
    \item faz nada.
\end{enumerate}

A tomada de decisão depende da expectativa do agente sobre a evolução futura do mercado, incluindo, entre outros:
\begin{itemize}
    \item qual a probabilidade de chegar uma oferta de compra com preço $P_s^{compra} > P_t^{bid_mm}$, $s >t$;
    \item qual a probabilidade de chegar uma oferta de venda com preço $P_s^{venda} \leq P_t^{bid_mm}$,
    \item qual o risco futuro da posição e se esse risco ultrapassa os limites estabelecidos pelas instituições controlando a negociação.
\end{itemize}

Por exemplo, assume que o agente espera a chegada de uma oferta de venda que possa ser executada na oferta de compra dele ( $P_t^{bid_mm}$). A decisão dele seria não ajustar a sua ordem atual, mas sim aguardar a execução e assim lucrar o BAS menos os custos de transação:
\begin{equation*}
    V_s := q_t^{ask_mm} \cdot \left(P_t^{ask_mm} - P_t^{bid_mm}\right) - tc_t, s> t,
\end{equation*}
sendo que $q_t, t>0$ é a quantidade executada no instante $t$, quando foi executada a oferta de venda, e $V_s, s\geq t$ representa o valor da carteira do agente na data $s$ e $tc_t$ é o custo de transação.\footnote{Os custos de transação incluem tipicamente os custos da corretora e emolumentos das bolas. Custos de custódio do outro lado não entrariam na nossa consideração, já que o agente não pretende manter a posição depois do fechamento do pregão. }

O objetivo do agente é maximizar o valor da sua carteira no final do dia, $T>t$ ao longo de todos os negócios $e=1,...,N$ ocorridos no dia
\begin{equation*}
    \max V_T := \sum_{e=1}^N q_e \cdot P_e, 0<e\leq T, 
\end{equation*}
sub a restrição que a quantidade final em estoque esteja zero: $q_T:= \sum_{e=1}^N q_e = 0$

O problema do MM foi tratado por [Alevaneda] sob algumas hipóteses sobre os processos de chegada de ofertas de compra e venda. Além disso, os autores não incluíram a restrição de estar sem estoque no final do dia, $q_T \neq 0$.
Nesse cenário, eles conseguiram definir uma estratégia ótima da seguinte forma:[incluir aqui resumo do paper]

Ao longo do tempo, diversos autores começaram a reduzir as hipóteses, deixando o cenário mais genérico.
[incluir aqui um pequeno resuma da literatura]


Em situações menos restritivos, como nos trabalhos de [xxx] ou [xxx], não há como achar uma estratégia ótima de forma analítica. As estratégias propostas têm parâmetros que precisam ser calibrados aos dados observados. 
Comparado com pesquisas que precisam calibrar parâmetros de estratégias, mas nas quais as estratégias são de forma "tomadora de preço" (\textit{price-taking}) enfrentamos um problema extra na formação de preço pelo agente.
Colocando dinamicamente ofertas no LOB, o agente alteraria o histórico das observações - especialmente das execuções. Portanto, o \textit{backtest} da estratégia não reflete como seria o resultado real com interferência do agente.

Além de focar no processamento de dados, como a criação em tempo real dos livros de oferta, a primeira parte do nosso projeto visa sugerir uma forma de \textit{backtest} e calibragem de parâmetros.

Na segunda parte da pesquisa vamos estender a situação do MM para um mercado multidimensional - ou seja, o agente aplica a estratégia de MM em diversos ativos ao mesmo tempo. A função-objetivo do agente com $k=1,...K$ ativos vira a ser:
\begin{eqnarray*}
    V_t^k := \sum_{e_k=1}^{N_k} q_{e_k} \cdot P_{e_k}, 0<e_k\leq T, \\
    \max V_T &:=&\sum_{k=1}^K   V_t^k
\end{eqnarray*}
Esse problema multidimensional abre algumas novas oportunidades para o MM, como:
\begin{itemize}
    \item ao invés de fechar a posição no final do dia, $q_T=0$ o agente pode manter uma posição, sob a condição que o risco global da carteira esteja dentro dos limites;
    \item se negociar ativos em diversos mercados com moedas diferentes o agente pode escolher em qual instante fazer a compra de câmbio; um exemplo seria a negociação em paralelo de ADRs numa bolsa americana e de ações na B3;
    \item as posições tomadas nos diversos mercados não necessariamente precisam ter uma fungibilidade física na custódia, mas podem simplesmente criar um saldo financeiro global das posições em todos os ativos.
\end{itemize}