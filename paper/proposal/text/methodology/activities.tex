
A execução do projeto será dividida nas seguintes etapas sequenciais:

\begin{enumerate}

\item Pesquisa Bibliográfica \label{item:bibliography_review}

Realizar uma revisão abrangente da literatura relacionada a Aprendizado por Reforço (RL), market making e estratégias de minimização de risco em finanças quantitativas.
Identificar os principais conceitos, algoritmos e abordagens utilizados em pesquisas anteriores.
Compreender as tendências atuais e os desafios enfrentados na aplicação do \textit{RL} em estratégias de \textit{trading}.

\item Coleta e Processamento de Dados \label{item:data}

Coletar dados históricos do mercado financeiro relevantes para o estudo.
Garantir a qualidade, assim como o formato correto para uso nos modelos e a consistência.

\item Definição do Ambiente de Simulação \label{item:environment}

Criar um ambiente de simulação que represente com precisão as condições do mercado, incluindo a dinâmica do livro de ordens, a volatilidade e outros fatores relevantes. Introduzir a noção de risco overnight no ambiente de simulação. Analisar o framework mais adequado à tarefa (\textit{OpenAI Gym}, \textit{RLlib}, \textit{garage}, etc.).

\item Escolha e Implementação de Algoritmos de Aprendizado por Reforço \label{item:agent}

Selecionar algoritmos de \textit{RL} adequados para a tarefa de market making e minimização de risco overnight, com base na revisão da literatura. Implementar os algoritmos escolhidos em um ambiente de desenvolvimento.

\item Treinamento do Agente de \textit{RL} \label{item:training}

Iniciar o treinamento do agente de \textit{RL} em ambiente simulado. Definir as recompensas e as métricas de desempenho relevantes para a minimização do risco overnight. Monitorar o progresso do treinamento e ajustar os hiperparâmetros conforme necessário.

\item Avaliação do Desempenho \label{item:evaluation}

Avaliar o desempenho do agente de \textit{RL} em cenários de simulação.
Realizar análises estatísticas para medir a eficácia na minimização de risco e na geração de lucro.
Comparar o desempenho do agente com estratégias de mercado tradicionais.

\item Ajuste e Otimização \label{item:adjustments}

Realizar ajustes no agente de \textit{RL} com base nos resultados da avaliação de desempenho. Otimizar a estratégia de market making para alcançar um equilíbrio entre lucro e minimização de risco.

\item Documentação e Relatórios \label{item:reports}

Documentar todo o processo, desde a escolha dos algoritmos até os resultados do treinamento. Escrever relatórios técnicos e científicos que descrevam os métodos e resultados obtidos durante a pesquisa.

\item Divulgação e Publicação \label{item:publishing}

- Apresentar os resultados em conferências acadêmicas relevantes. Preparar artigos científicos para submissão em conferências e revistas da área de finanças quantitativas e aprendizado de máquina.

- Disponibilizar os algoritmos, dados de simulação e documentação publicamente para que outros pesquisadores possam utilizar e reproduzir os resultados. Os resultados e o código desenvolvido será publicado usando a ferramenta de versionamento \textit{Git}, assim como instruções para uso e detalhes de implementação do agente no formato \textit{Markdown}.
\end{enumerate}
