
A execução do projeto será dividida nas seguintes etapas:

\begin{enumerate}

\item Pesquisa Bibliográfica \label{item:bibliography_review}

Realizar uma revisão abrangente da literatura relacionada a Aprendizado por Reforço (RL), market making e estratégias de minimização de risco em finanças quantitativas além das referências iniciais, usando bancos de dados de periódicos, como o CAFe e Arxiv.
Realizar uma análise estatística dos principais algoritmos, políticas de aprendizado e bancos de dados utilizados em pesquisas anteriores.
Compreender as tendências atuais e os desafios enfrentados na aplicação do \textit{RL} em estratégias de \textit{trading}.

\item Coleta e Processamento de Dados \label{item:data}

Coletar dados históricos do mercado financeiro relevantes para o estudo tendo como referência as bases de dados divulgadas na literatura.
Adaptar os dados para o formato do paradigma de Aprendizado por Reforço.

\item Definição do Ambiente de Simulação \label{item:environment}

A partir dos dados obtidos e bibliotecas de simulação, preparar um ambiente de que represente com precisão as condições do mercado de alta frequência, incluindo a dinâmica do motor do livro de ordens, a volatilidade e o processo de chegada de ofertas. Analisar o framework de simulação mais adequado à tarefa (\textit{OpenAI Gym}, \textit{RLlib}, \textit{garage}, etc.).

\item Escolha do algoritmo de Aprendizado por Reforço \label{item:agent}

Selecionar algoritmos de \textit{RL} adequados para a tarefa de market making e restrições adicionais (especificamente a minimização do risco), com base na revisão da literatura e na taxonomia dos algoritmos existentes (baseado em modelo, livre de modelo, entre outros).

\item Treinamento do Agente de \textit{RL} \label{item:training}

Iniciar o treinamento do agente de \textit{RL} no ambiente simulado. Definir as funções de recompensa e as métricas de desempenho relevantes para a minimização do risco overnight. Monitorar o progresso do treinamento e ajustar os hiperparâmetros conforme necessário.

\item Avaliação do Desempenho \label{item:evaluation}

Avaliar o desempenho do agente de \textit{RL} em cenários de simulação, e realizar análises estatísticas em cima dos hiperparâmetros usados. Comparar o desempenho do agente com estratégias de mercado tradicionais.
Realizar ajustes no agente de \textit{RL} com base nos resultados da avaliação de desempenho. Otimizar a estratégia de market making para alcançar um equilíbrio entre lucro e minimização de risco.

\item Divulgação e Publicação \label{item:publishing}

- Apresentar os resultados em conferências acadêmicas relevantes. Preparar artigos científicos para submissão em conferências e revistas da área de finanças quantitativas e aprendizado de máquina.

- Disponibilizar os algoritmos, dados de simulação e documentação publicamente para que outros pesquisadores possam utilizar e reproduzir os resultados. Os resultados e o código desenvolvido será publicado usando a ferramenta de versionamento \textit{Git}, assim como instruções para uso e detalhes de implementação do agente no formato \textit{Markdown}.
\end{enumerate}
