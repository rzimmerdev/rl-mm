
Com base nas tarefas enumeradas na Seção \ref{section:activities}, é mostrado na Tabela \ref{tab:chronogram} o cronograma a ser executado durante a realização deste projeto.

\begin{table}[ht]
\centering
\caption{Cronograma das atividades.}
\begin{tabular}{|c|c|c|c|c|c|c|c|c|c|c|c|c|}
\hline
\multirow{2}{*}{{\bf Fases}} & \multicolumn{12}{c|}{{\bf Meses}}
\\ \cline{2-13}
    & 1 & 2 & 3 & 4 & 5 & 6 & 7 & 8 & 9 & 10 & 11 & 12
\\ \hline
    {\bf \hyperref[item:bibliography_review]{Pesquisa Bibliográfica} } 
    & x & x & x & x & & & & & & & &
\\ \hline
    {\bf \hyperref[item:data]{Coleta de Dados}} 
    &  &  & x & x & & & & & & & &
\\ \hline
    {\bf \hyperref[item:environment]{Ambiente de Simulação}} 
    &  &  & x & x & x &  &  & & & & &
\\ \hline
    {\bf \hyperref[item:agent]{Definição do Agente}} 
    &  &  &  &  & x & x & x & x &  &  &  & 
\\ \hline
    {\bf \hyperref[item:training]{Treinamento do Agente}} 
    & & & & & & & x & x & & & & 
\\ \hline
    {\bf \hyperref[item:evaluation]{Métricas de Desempenho}} 
    &  &  &  &  &  &  &  & x & x & & & 
\\ \hline    
    {\bf \hyperref[item:adjustments]{Ajustes e Otimização}} 
    & & & & & x & x & & & x & x & x &
\\ \hline
    {\bf \hyperref[item:reports]{Produção do Artigo}} 
    & & & & & & & & & x & x & x & x
\\ \hline
\end{tabular}
\label{tab:chronogram}
\end{table}
