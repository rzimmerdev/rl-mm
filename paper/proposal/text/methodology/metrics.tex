
A análise dos resultados da estratégia de Market Making (MM) envolve uma avaliação abrangente das decisões tomadas pelo agente ao longo do tempo e seus impactos na carteira de ativos. Algumas das formas de análise incluem:

\textbf{Desempenho Financeiro}: Avaliar o valor da carteira ao longo do tempo, considerando todas as transações, incluindo custos de transação e flutuações nos preços dos ativos.

\textbf{Minimização do Risco Overnight}: Avaliar a eficácia da estratégia em minimizar a exposição a movimentos adversos de preços após o fechamento do mercado.

\textbf{Decisões de Inserção e Ajuste}: Analisar as decisões do agente MM em relação à inserção e ajuste das ofertas de compra e venda, considerando a adaptação às expectativas de mercado.

\textbf{Impacto na Liquidez do Mercado}: Examinar como a estratégia MM afeta a liquidez do mercado, incluindo a capacidade de execução de outros participantes.

\textbf{Comparação com Estratégias Tomadoras de Preço}: Comparar o desempenho da estratégia MM com estratégias que não inserem ofertas no livro de ofertas (price-taking), destacando vantagens ou desvantagens.

\textbf{Calibragem de Parâmetros}: Em cenários onde a calibragem de parâmetros é necessária, analisar como esses parâmetros são ajustados ao longo do tempo e como isso afeta o desempenho global da estratégia.

\textbf{Mercado Multidimensional}: Na segunda parte da pesquisa, estender a análise para estratégias de MM em mercados multidimensionais, considerando a alocação de recursos entre diferentes ativos e a gestão do risco global da carteira.
