
O andamento da pesquisa buscará seguir a seguinte metodologia, especificamente o plano de atividades descrito — dentro dos prazos estipulados — visando concluir o objetivo comentado acima. Uma série de métricas serão utilizadas para estabelecer a eficácia do agente na tarefa proposta, assim como também comparações à outras estratégias de \textit{price-taking}.

\subsection{Planejamento de Atividades}
\label{section:activities}

A execução do projeto será dividida nas seguintes etapas sequenciais:

\begin{enumerate}

\item Pesquisa Bibliográfica \label{item:bibliography_review}

Realizar uma revisão abrangente da literatura relacionada a Aprendizado por Reforço (RL), market making e estratégias de minimização de risco em finanças quantitativas.
Identificar os principais conceitos, algoritmos e abordagens utilizados em pesquisas anteriores.
Compreender as tendências atuais e os desafios enfrentados na aplicação do \textit{RL} em estratégias de \textit{trading}.

\item Coleta e Processamento de Dados \label{item:data}

Coletar dados históricos do mercado financeiro relevantes para o estudo.
Garantir a qualidade, assim como o formato correto para uso nos modelos e a consistência.

\item Definição do Ambiente de Simulação \label{item:environment}

Criar um ambiente de simulação que represente com precisão as condições do mercado, incluindo a dinâmica do livro de ordens, a volatilidade e outros fatores relevantes. Introduzir a noção de risco overnight no ambiente de simulação. Analisar o framework mais adequado à tarefa (\textit{OpenAI Gym}, \textit{RLlib}, \textit{garage}, etc.).

\item Escolha e Implementação de Algoritmos de Aprendizado por Reforço \label{item:agent}

Selecionar algoritmos de \textit{RL} adequados para a tarefa de market making e minimização de risco overnight, com base na revisão da literatura. Implementar os algoritmos escolhidos em um ambiente de desenvolvimento.

\item Treinamento do Agente de \textit{RL} \label{item:training}

Iniciar o treinamento do agente de \textit{RL} em ambiente simulado. Definir as recompensas e as métricas de desempenho relevantes para a minimização do risco overnight. Monitorar o progresso do treinamento e ajustar os hiperparâmetros conforme necessário.

\item Avaliação do Desempenho \label{item:evaluation}

Avaliar o desempenho do agente de \textit{RL} em cenários de simulação.
Realizar análises estatísticas para medir a eficácia na minimização de risco e na geração de lucro.
Comparar o desempenho do agente com estratégias de mercado tradicionais.

\item Ajuste e Otimização \label{item:adjustments}

Realizar ajustes no agente de \textit{RL} com base nos resultados da avaliação de desempenho. Otimizar a estratégia de market making para alcançar um equilíbrio entre lucro e minimização de risco.

\item Documentação e Relatórios \label{item:reports}

Documentar todo o processo, desde a escolha dos algoritmos até os resultados do treinamento. Escrever relatórios técnicos e científicos que descrevam os métodos e resultados obtidos durante a pesquisa.

\item Divulgação e Publicação \label{item:publishing}

- Apresentar os resultados em conferências acadêmicas relevantes. Preparar artigos científicos para submissão em conferências e revistas da área de finanças quantitativas e aprendizado de máquina.

- Disponibilizar os algoritmos, dados de simulação e documentação publicamente para que outros pesquisadores possam utilizar e reproduzir os resultados. Os resultados e o código desenvolvido será publicado usando a ferramenta de versionamento \textit{Git}, assim como instruções para uso e detalhes de implementação do agente no formato \textit{Markdown}.
\end{enumerate}


\subsection{Cronograma}
\label{section:chronogram}

Com base nas tarefas enumeradas na Seção \ref{section:activities}, é mostrado na Tabela \ref{tab:chronogram} o cronograma a ser executado durante a realização deste projeto.

\begin{table}[ht]
\centering
\caption{Cronograma das atividades.}
\begin{tabular}{|c|c|c|c|c|c|c|c|c|c|c|c|c|}
\hline
\multirow{2}{*}{{\bf Fases}} & \multicolumn{12}{c|}{{\bf Meses}}
\\ \cline{2-13}
    & 1 & 2 & 3 & 4 & 5 & 6 & 7 & 8 & 9 & 10 & 11 & 12
\\ \hline
    {\bf \hyperref[item:bibliography_review]{Pesquisa Bibliográfica} } 
    & x & x & x & x & & & & & & & &
\\ \hline
    {\bf \hyperref[item:data]{Coleta de Dados}} 
    &  &  & x & x & & & & & & & &
\\ \hline
    {\bf \hyperref[item:environment]{Ambiente de Simulação}} 
    &  &  & x & x & x & x &  & & & & &
\\ \hline
    {\bf \hyperref[item:agent]{Definição do Agente}} 
    &  &  &  &  & x & x & x & x &  &  &  & 
\\ \hline
    {\bf \hyperref[item:training]{Treinamento do Agente}} 
    & & & & & & & x & x & x & x & & 
\\ \hline
    {\bf \hyperref[item:evaluation]{Métricas de Desempenho}} 
    &  &  &  &  &  &  &  & x & x & x & & 
\\ \hline
    {\bf \hyperref[item:publishing]{Produção do Artigo}} 
    & & & & & & & & x & x & x & x & x
\\ \hline
\end{tabular}
\label{tab:chronogram}
\end{table}
%

\subsection{Resultados Esperados}
\label{section:results}
\section{Realized Experiments and Result Discussion}
\label{sec:realized-experiments-and-result-discussion}

\subsection{Experiment Setup}
\label{subsec:experiment-setup}

The experiments were conducted using a Red-black tree implementation for the limit order book, while new orders follow the event dynamics described in
~\hyperref[subsec:market-model-description-and-environment-dynamics]{Section~\ref*{subsec:market-model-description-and-environment-dynamics}}.
Both restricted agents were implemented through the Proximal Policy Optimization (PPO) algorithm, which is a model-free, on-policy algorithm that optimizes the policy directly.
A max episode value of 10,000 episodes was used, with each episode consisting on average of 390 observations (or 1 event per corresponding market minute).
Per gathered trajectory, 500 epochs were used for the policy improvement step, with a batch size of 64.
We used the Adam optimizer with a learning rate of $10^{-4}$ and a discount factor of 0.99.
For comparison metrics, we used the Sharpe ratio, the average daily return, and the average daily volatility.

\subsection{Experiment Results}
\label{subsec:experiment-results}

% Graphs:
% average financial return + confidence interval (+- volatility) x episode number
% average financial return (+- volatility) x current timestep (per 100x trajectories after training)
% average inventory x current timestep (per 100x trajectories after training)
% average reward moving average x episode number

% Table:
% Cols: agent with restriction, agent without restriction, benchmark
% Training
% Rows: Training time +- std
%       Time per episode +- std
%       Mean processing time actor +- std per episode
%       Mean processing time critic +- std per episode
%       Mean financial return +- std

% Test (after last episode or convergence)
% Rows: Mean financial return +- std
%       Mean Sharpe ratio +- std
%       Agent action latency +- std
%       Mean inventory at market close


\subsection{Forma de Análise dos Resultados}
\label{section:metrics}
A análise dos resultados da estratégia de \textit{market making} será  realizada diretamente em cima dos retornos obtidos pelo agente, diários e acumulados, assim como a volatilidade do portfólio e impacto de mercado das operações para zerar a carteira até o fechamento do pregão. A avaliação das decisões tomadas pelo agente ao longo do tempo e seus impactos na carteira de ativos incluem mas não se limitam à:

\textbf{Desempenho}: Avaliar o valor da carteira ao longo do tempo, considerando todas as transações, incluindo custos de transação e flutuações nos preços dos ativos.
A estratégia deve maximizar o valor da carteira do agente ao final do dia, levando em conta todas as transações, custos de transação e flutuações nos preços dos ativos.

\textbf{Minimização do Risco Overnight}: Ao mesmo tempo, a estratégia deve minimizar o risco overnight, mantendo exposições limitadas a movimentos adversos de preços após o fechamento do mercado.

Além disso, a estratégia deve contribuir positivamente para a liquidez do mercado, facilitando a execução de negócios para outros participantes buscando lucro. Isso, no entanto, deve ser equilibrado com considerações de custos de transação, incluindo comissões de corretagem e outros custos associados à execução de negócios.

\textbf{Decisões de Inserção e Ajuste}: Analisar as decisões do agente MM em relação à inserção e ajuste das ofertas de compra e venda, considerando a adaptação às expectativas de mercado. O agente deve levar em consideração as distribuições de quantidades executadas dado determinados preços. Serão utilizados testes estatísticos para verificar se o grafo de transições obtido pelo agente efetivamente reflete os processos do mercado real.


\textbf{Comparação com Estratégias Tomadoras de Preço}: Comparar as métricas de desempenho mencionadas da estratégia MM com estratégias que não inserem ofertas limite no livro de ordens (também chamadas de estratégias de \textit{price-taking}), destacando desempenho melhora ou não com relação ao agente desenvolvido.



% Precisa disso afinal? Cabe questionamentos...
% \subsection{Exequibilidade}
