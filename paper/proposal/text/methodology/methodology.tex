
O andamento da pesquisa buscará seguir a seguinte metodologia, especificamente o plano de atividades descrito — dentro dos prazos estipulados — visando concluir o objetivo comentado acima. Uma série de métricas serão utilizadas para estabelecer a eficácia do agente na tarefa proposta, assim como também comparações à outras estratégias de \textit{price-taking}.

\subsection{Planejamento de Atividades}
\label{section:activities}

A execução do projeto será dividida nas seguintes etapas sequenciais:

\begin{enumerate}

\item Pesquisa Bibliográfica \label{item:bibliography_review}

Realizar uma revisão abrangente da literatura relacionada a Aprendizado por Reforço (RL), market making e estratégias de minimização de risco em finanças quantitativas.
Identificar os principais conceitos, algoritmos e abordagens utilizados em pesquisas anteriores.
Compreender as tendências atuais e os desafios enfrentados na aplicação do \textit{RL} em estratégias de \textit{trading}.

\item Coleta e Processamento de Dados \label{item:data}

Coletar dados históricos do mercado financeiro relevantes para o estudo.
Garantir a qualidade, assim como o formato correto para uso nos modelos e a consistência.

\item Definição do Ambiente de Simulação \label{item:environment}

Criar um ambiente de simulação que represente com precisão as condições do mercado, incluindo a dinâmica do livro de ordens, a volatilidade e outros fatores relevantes. Introduzir a noção de risco overnight no ambiente de simulação. Analisar o framework mais adequado à tarefa (\textit{OpenAI Gym}, \textit{RLlib}, \textit{garage}, etc.).

\item Escolha e Implementação de Algoritmos de Aprendizado por Reforço \label{item:agent}

Selecionar algoritmos de \textit{RL} adequados para a tarefa de market making e minimização de risco overnight, com base na revisão da literatura. Implementar os algoritmos escolhidos em um ambiente de desenvolvimento.

\item Treinamento do Agente de \textit{RL} \label{item:training}

Iniciar o treinamento do agente de \textit{RL} em ambiente simulado. Definir as recompensas e as métricas de desempenho relevantes para a minimização do risco overnight. Monitorar o progresso do treinamento e ajustar os hiperparâmetros conforme necessário.

\item Avaliação do Desempenho \label{item:evaluation}

Avaliar o desempenho do agente de \textit{RL} em cenários de simulação.
Realizar análises estatísticas para medir a eficácia na minimização de risco e na geração de lucro.
Comparar o desempenho do agente com estratégias de mercado tradicionais.

\item Ajuste e Otimização \label{item:adjustments}

Realizar ajustes no agente de \textit{RL} com base nos resultados da avaliação de desempenho. Otimizar a estratégia de market making para alcançar um equilíbrio entre lucro e minimização de risco.

\item Documentação e Relatórios \label{item:reports}

Documentar todo o processo, desde a escolha dos algoritmos até os resultados do treinamento. Escrever relatórios técnicos e científicos que descrevam os métodos e resultados obtidos durante a pesquisa.

\item Divulgação e Publicação \label{item:publishing}

- Apresentar os resultados em conferências acadêmicas relevantes. Preparar artigos científicos para submissão em conferências e revistas da área de finanças quantitativas e aprendizado de máquina.

- Disponibilizar os algoritmos, dados de simulação e documentação publicamente para que outros pesquisadores possam utilizar e reproduzir os resultados. Os resultados e o código desenvolvido será publicado usando a ferramenta de versionamento \textit{Git}, assim como instruções para uso e detalhes de implementação do agente no formato \textit{Markdown}.
\end{enumerate}


\subsection{Cronograma}
\label{section:chronogram}

Com base nas tarefas enumeradas na Seção \ref{section:activities}, é mostrado na Tabela \ref{tab:chronogram} o cronograma a ser executado durante a realização deste projeto.

\begin{table}[ht]
\centering
\caption{Cronograma das atividades.}
\begin{tabular}{|c|c|c|c|c|c|c|c|c|c|c|c|c|}
\hline
\multirow{2}{*}{{\bf Fases}} & \multicolumn{12}{c|}{{\bf Meses}}
\\ \cline{2-13}
    & 1 & 2 & 3 & 4 & 5 & 6 & 7 & 8 & 9 & 10 & 11 & 12
\\ \hline
    {\bf \hyperref[item:bibliography_review]{Pesquisa Bibliográfica} } 
    & x & x & x & x & & & & & & & &
\\ \hline
    {\bf \hyperref[item:data]{Coleta de Dados}} 
    &  &  & x & x & & & & & & & &
\\ \hline
    {\bf \hyperref[item:environment]{Ambiente de Simulação}} 
    &  &  & x & x & x & x &  & & & & &
\\ \hline
    {\bf \hyperref[item:agent]{Definição do Agente}} 
    &  &  &  &  & x & x & x & x &  &  &  & 
\\ \hline
    {\bf \hyperref[item:training]{Treinamento do Agente}} 
    & & & & & & & x & x & x & x & & 
\\ \hline
    {\bf \hyperref[item:evaluation]{Métricas de Desempenho}} 
    &  &  &  &  &  &  &  & x & x & x & & 
\\ \hline
    {\bf \hyperref[item:publishing]{Produção do Artigo}} 
    & & & & & & & & x & x & x & x & x
\\ \hline
\end{tabular}
\label{tab:chronogram}
\end{table}
%

\subsection{Resultados Esperados}
\label{section:results}
\section{Realized Experiments and Results}
\label{sec:realized-experiments-and-results}

\subsection{Experiment Setup}
\label{subsec:experiment-setup}

A max episode value of 10,000 episodes was used, with each episode consisting on average of 390 observations (or 1 event per corresponding market minute).
Per gathered trajectory, 10 epochs were used for the policy improvement step.
For comparison metrics, we used the Sharpe ratio, the daily return, and the daily volatility,
averaged over 50 episodes after training using the same simulator hyperparameters used for training.

% hyperparam optimization
% --gamma=0.9 --epsilon=0.25 --lambd=0.85 --entropy=0.0012 --lr_policy=3e-4 --lr_value=3e-4  --batch_size=256
We used the Adam optimizer with a learning rate of $3 \times 10^{-4}$ for both the policy and value networks.
The discount factor $\gamma$ was set to 0.9, the GAE parameter $\lambda$ was set to 0.85, and the PPO clipping parameter $\epsilon$ was set to 0.25.
The entropy coefficient was set to 0.0012, and the batch size set to 256 samples per episode/update.
We optimized the hyperparameters using a simple grid search approach.

\subsection{Experiment Results}
\label{subsec:experiment-results}

% Graphs:
% average financial return + confidence interval (+- volatility) x episode number
% average financial return (+- volatility) x current timestep (per 100x trajectories after training)
% average inventory x current timestep (per 100x trajectories after training)
% average reward moving average x episode number

% Table:
% Cols: rl-agent, benchmark agent
% Training
% Rows: Training time
%       3:08:54, , -
%       Time per episode +- std
%       0.8458 \pm 0.1044
%       Mean processing time actor +- std per episode
%       Mean processing time critic +- std per episode
%       Mean financial return +- std

% Mean PnL: -1.174902750662374e-05, Std PnL: 4.761125180139532e-05, Sortino Ratio: -0.5046819237780464
% Mean Stoikov PnL: -0.000423402308920138, Std Stoikov PnL: 0.0010584297592714442, Sortino Ratio: -0.6399344498955829

% Test (after last episode or convergence)
% Rows: Mean financial return +- std
%       Mean Sharpe ratio +- std
%       Agent action latency +- std
%       Mean inventory at market close

To evaluate the financial performance of the trained reinforcement learning agent, we analyzed key metrics such as
financial return, return volatility, and the Sortino ratio.
The results were averaged over 50 episodes after training.

As shown in Table~\ref{tab:test-results}, the reinforcement learning agent exhibited a mean financial return of $-1.174 \times 10^{-5}$
(annualized return of about $-0.3\%$.), an almost neutral performance under adverse market conditions,
while the benchmark agent had a mean financial return of $-0.0004$ (annualized return of about $-10.5\%$),
a clear underperformance under the same conditions.
The return volatility for the RL-agent was lower at $4.7611 \times 10^{-5}$ (annualized volatility of about $1.2\%$),
compared to $0.001$ for the benchmark agent (annualized volatility of about $25.2\%$), indicating more stable financial returns.
Additionally, the Sortino ratio of the RL-agent was $-0.5046$, also outperforming the benchmark's ratio of $-0.6399$.

\begin{table}
    \centering
    \centering
    \small
    \begin{tabular}{|c|c|c|}
        \hline
        \textbf{Training}      & \textbf{Metric}                           \\
        \hline
        Training Time          & $3\text{h}08\text{m}54\text{s}$           \\
        Time per Episode       & $0.8458 \pm 0.1044$ \text{ (s)}           \\
        Processing Time Actor  & $0.0029 \pm 0.001 \text{ (s)}$            \\
        Processing Time Critic & $0.0002 \pm 3 \times 10^{-5} \text{ (s)}$ \\
        \hline
    \end{tabular}
    \caption{Test Results}
    \label{tab:test-results}
    \centering
    \vspace{0.5cm}
    \small
    \begin{tabular}{|c|c|c|}
        \hline
        \textbf{Test}     & \textbf{RL-Agent}       & \textbf{Benchmark} \\
        \hline
        Financial Return  & $-1.174 \times 10^{-5}$ & $-0.0004$          \\
        Return Volatility & $4.7611 \times 10^{-5}$ & $0.0010$           \\
        Sortino Ratio     & $-0.5046$               & $-0.6399$          \\
        \hline
    \end{tabular}
    \caption{Training Results}
    \label{tab:training-results}
\end{table}


% reward.png and returns.png

\begin{figure}
    \centering
    \begin{minipage}{0.45\textwidth}
        \centering
        \includegraphics[width=1\textwidth]{images/reward}
        \caption{Exponential moving average of the training reward per episode, with a linear trend line.}
        \label{fig:average-reward-moving-average}
    \end{minipage}
    \hspace{0.04\textwidth} % Adjust horizontal space between figures
    \begin{minipage}{0.45\textwidth}
        \centering
        \includegraphics[width=1\textwidth]{images/returns}
        \caption{Financial return, averaged over 100 trajectories with a 1 standard deviation confidence interval.}
        \label{fig:average-financial-return}
    \end{minipage}
\end{figure}



\subsection{Forma de Análise dos Resultados}
\label{section:metrics}
A análise dos resultados da estratégia de \textit{market making} será  realizada diretamente em cima dos retornos obtidos pelo agente, diários e acumulados, assim como a volatilidade do portfólio e impacto de mercado das operações para zerar a carteira até o fechamento do pregão. A avaliação das decisões tomadas pelo agente ao longo do tempo e seus impactos na carteira de ativos incluem mas não se limitam à:

\textbf{Desempenho}: Avaliar o valor da carteira ao longo do tempo, considerando todas as transações, incluindo custos de transação e flutuações nos preços dos ativos. Comparar as métricas de desempenho mencionadas da estratégia MM com estratégias que não inserem ofertas limite no livro de ordens (chamadas de estratégias de \textit{price-taking}), destacando desempenho melhora ou não com relação ao agente desenvolvido.

\textbf{Minimização do Risco Overnight}: Ao mesmo tempo, a estratégia deve minimizar o risco overnight, mantendo exposições limitadas a movimentos adversos de preços após o fechamento do mercado.
A estratégia deve contribuir positivamente para a liquidez do mercado, facilitando a execução de negócios para outros participantes buscando lucro. Isso, no entanto, deve ser equilibrado com considerações de custos de transação, incluindo comissões de corretagem e outros custos associados à execução de negócios.

\textbf{Explicabilidade e Interpretabilidade do Agente}: Analisar as decisões do agente MM em relação à inserção e ajuste das ofertas de compra e venda, considerando a adaptação às expectativas de mercado. O agente deve levar em consideração as distribuições de quantidades executadas dado determinados preços. Serão utilizados testes estatísticos para verificar se o grafo de transições obtido pelo agente efetivamente reflete os processos do mercado real.




% Precisa disso afinal? Cabe questionamentos...
% \subsection{Exequibilidade}
