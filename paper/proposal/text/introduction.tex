Há um crescente foco na pesquisa para computação financeira voltado à estratégias intra-diárias de \textit{MM} \cite{} (advances in mm), que consistem em criar ordens de limite de modo a lucrar em cima da diferença entre o preço de bid (Melhor preço de compra) e o preço de ask (Melhor preço de venda), chamada de Bid-Ask-Spread (BAS). Por serem estratégias intra-diárias, utilizam técnicas de gerenciamento de risco que funcionam estritamente para os horários em que o mercado está aberto e permite novas ordens de compra e venda. Durante esses horários, um agente de \textit{MM} oferta simultaneamente ordens de compra e venda ao mercado, sob o risco de poder não ter uma ordem executada até o fechamento do mercado\cite{} (risks in MM).

Esse tipo de risco, chamado de risco \textit{overnight} existe devido ao fato de o mercado estar fechado durante a noite e não processar nenhuma ordem existente até sua abertura. Logo, qualquer posição mantida durante a noite está vulnerável a eventos inesperados, como notícias financeiras, mudanças macroeconômicas e outros, resultando em aberturas de mercado altamente voláteis que podem afetar negativamente as posições do agente. Portanto, um agente que esteja sob a restrição de zerar sua posição até o momento de fechamento e que seja capaz de \textbf{maximizar o retorno das operações diárias e equilibrar com a minimização do risco de posições noturnas} caracteriza um avanço crítico para estratégias de \textit{MM} usadas por fundos de investimento e corretoras.

Pesquisas prévias relacionadas na área de trading algorítmico usam em muitos casos dados históricos para calibrar e gerar análises numéricas sobre seus agentes - processo chamado de \textit{backtesting}. Com o objetivo de obter um agente de \textit{MM} condicionado à restrição mencionada, teremos como primeiro desafio determinar uma metodologia para obter dados financeiros e fazer a validação das nossas hipóteses sobre o agente, considerando que as ordens de um agente de MM geram mudanças no estado do mercado. Isso significa que sua interação com o mercado altera a amostra histórica, impedindo o uso de técnicas de \textit{backtest} regulares, pois um agente de MM não é um consumidor de preços (\textit{price-taker}), mas sim um fornecedor de preços (\textit{price-maker}). Num cenário real, não pode ser descartada o efeito das ações do agente: imagine o cenário em que um agente de \textit{MM} externo já esteja atuando no mercado. Tal agente continuará tentando se manter no nível das melhores ofertas de modo a ter suas ordens executadas primeiro. Assim, as ordens que nosso agente enviar afetarão os valores subsequentes deste outro agente.

Outro detalhe importante sobre o formato de dados que o agente irá tratar é o fato de que são dados de alta frequência, ou seja a cada instante curto de tempo (geralmente na escala de segundos ou milissegundos) é necessário construir ou atualizar o livro de ofertas, de modo a refletir novas ofertas de compra e venda, cancelamentos de ordens, e eventuais ajustes de preços de ofertas existentes.
Quando uma ordem de qualquer agente é executada, observa-se no mercado uma chegada de uma oferta que ultrapassa o melhor preço do outro lado do livro e causa, imediatamente, o desaparecimento das ordens afetadas. A diferença de tempo entre as ordens impactará portanto na solução da estratégia ótima de gestão de estoque, já que o intervalo entre chegada e tomada de decisão é curto em comparação com dados usados para estratégias de longo prazo ou de \textit{price-taking}.

Portanto, antes de idealizar e treinar quaisquer agentes, buscamos obter alternativas voltadas ao \textit{MM} para os seguintes itens: 
\begin{itemize}
    \item uso de dados históricos \underline{estáticos} para calibrar e avaliar estratégias;
    \item uso de dados de \underline{alta frequência}, especialmente quando a frequência influencia nas decisões do agente.
\end{itemize}

Como conclusão da pesquisa, esperamos obter um agente que execute uma estratégia de \textit{MM} que maximize o retorno diário esperado em múltiplos ativos e que se adapte aos processos de chegada de ofertas e execuções do mercado, controlando o risco associado às posições mantidas após o fechamento do mercado. 

% O aspecto de considerar diferentes ativos em paralelo nesse cenário é uma novidade nesse campo de pesquisa, além de ser um cenário típico nos mercados financeiros. Os agentes do mercado que hoje trabalham com uma estratégia \textit{MM} (tipicamente \textit{hedge-funds}) raramente se restringem a um único ativo. Dessa forma, uma visão de portfólio na gestão do risco das posições intra diárias é um avanço extremamente importante.

% Em suma, temos duas etapas adicionais a serem consideradas que se somam ao problema principal de conceitualização e treino do agente: 