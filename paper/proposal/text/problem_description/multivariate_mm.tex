Ao considerarmos a situação em que o MM aplica a sua estratégia em diversos mercados simultaneamente observamos um aumento da complexidade, mas também das alternativas para lidar com riscos envolvidos - chamamos essa situação de MM \textbf{simultâneo} ou \textbf{multivariado}.

Para tal, consideramos a possibilidade do agente negociar um ativo $i$ dentre um total de $k$ ativos diferentes, em possivelmente diversos mercados. Os elementos do espaço de observações e de ações do agente passam a ser os vetores 
\[ 
\mathcal{S} = \{(\mathbf{P}, \mathbf{W}, \mathbf{I}) \mid \mathbf{P} \in \mathbb{R}^{k}, \mathbf{W} \in \mathbb{R}^{k}, \mathbf{I} \in \mathbb{Z}^{k}\}
\]
\[
\mathcal{A} = \left\{ (\mathbf{p}^a, \mathbf{p}^b, \mathbf{Q}^a, \mathbf{Q}^b) \mid \mathbf{p} \in \mathbb{R}^{k},  \mathbf{Q} \in \mathbb{N}^{k} \right\}
\].

O objetivo principal (\ref{eq:objective}) continua a mesmo, e remove-se a restrição (\ref{eq:inventory_restriction}). Contudo, surgem novas alternativas para proteção da carteira durante a noite:

\begin{enumerate}
    \item O agente pode avaliar o risco global da carteira, e incluir um único ativo de proteção contra o risco global ao final do dia.
    \item Se o ativo for negociado em múltiplas bolsas (chamado também de ativo \textit{co-listed}), o agente pode continuar a negociação deste em outra bolsa caso uma delas esteja fechada.
\end{enumerate}

Considerando um cenário em que haja ações em \textit{co-listing}, surge a possibilidade de criação de estratégias mais sofisticadas, permitindo a implementação dos itens mencionados acima.
