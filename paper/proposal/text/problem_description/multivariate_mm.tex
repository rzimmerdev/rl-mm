Ao considerarmos a situação em que o MM aplica a sua estratégia em diversos mercados simultaneamente observamos um aumento da complexidade, mas também das alternativas para lidar com riscos envolvidos - chamamos essa situação de MM \textbf{simultâneo} ou \textbf{multivariado}.

Para tal, a função valor $V_{T}$ do agente passa a ser a soma de todas posições parciais $V_{T, k}$, onde $k$ é uma bolsa onde o agente possui ações:
\begin{eqnarray*}
    V_t &=& \sum_{k=1}^N V_{T, k}\\
\end{eqnarray*}

O objetivo principal (\ref{objective_equation}) continua o mesmo, e mantém-se a restrição (\ref{overnight_restriction}). Contudo, surgem novas alternativas para proteção da carteira durante a noite:

\begin{enumerate}
    \item O agente pode avaliar o risco global da carteira, e incluir um único ativo de proteção contra o risco global ao final do dia.
    \item Se o ativo for negociado em múltiplas bolsas (chamado também de ativo \textit{co-listed}), o agente pode continuar a negociação deste em outra bolsa caso uma delas esteja fechada.
\end{enumerate}

Considerando um cenário em que haja ações em \textit{co-listing}, surge a possibilidade de criação de estratégias mais sofisticadas, permitindo a implementação dos itens mencionados acima.
