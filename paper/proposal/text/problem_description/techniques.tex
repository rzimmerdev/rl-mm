Em síntese, a otimização do retorno financeiro em operações de \textit{market-making} envolve desafios complexos, abordados por métodos analíticos tradicionais e soluções analíticas para tempo contínuo, mas que frequentemente enfrentam limitações na dinâmica do ambiente financeiro \citep{Avellaneda2008, rao2020stochastic, Gasperov2021}.

O aprendizado por reforço, notadamente com o algoritmo Soft Actor-Critic (SAC) e Deep Q-Learning (DQL), destaca-se pela sua adaptabilidade contínua às mudanças na dinâmica do sistema do mercado, oferecendo uma abordagem eficaz e flexível \citep{Ganesh2019, bakshaev2020marketmaking, Sutton2018}. A aplicação de técnicas como redes neurais de memória de curto prazo (LSTM) no aprendizado por reforço demonstra a capacidade de lidar com complexidades e padrões temporais nos mercados financeiros \citep{WOS:000747190900001}.

Embora as distintas abordagens apresentem vantagens e desvantagens, a escolha pelo aprendizado por reforço visa superar desafios e adaptar-se à complexidade computacional do ambiente \citep{WOS:000963297000001}. Assim, o uso do paradigma de aprendizado por reforço foi escolhido para obter a solução da equação de otimalidade de Bellman e da política do agente. Finalmente, a contribuição proposta nesta pesquisa consiste em integrar a aversão ao risco noturno à estratégia do agente, adicionando uma restrição que limita a exposição ao mercado no final do dia, ampliando assim a compreensão e gestão dos riscos associados ao \textit{market-making} \citep{almgren2000}. 

Existem algumas alternativas para formalizar matematicamente essa restrição:
\begin{enumerate}
    \item De forma simplificada, no final do dia o agente não pode ter nenhum ativo em posição: 
    \begin{equation}
        I_{T} = 0
        \label{eq:inventory_restriction}
    \end{equation}
    \item Com uma abordagem mais complexa, se houver alguma posição restante, o agente precisa \textit{headgear}\footnote{De maneira simplificada, o \textit{hedge} consiste em comprar ou vender ativos que tenham uma exposição ao risco oposta aos riscos da carteira atual, de modo a equilibrar a posição.} sua exposição ao risco ao participar em outros mercados abertos no momento, abordagem que chamamos de \textit{market making} \textbf{simultâneo}.
\end{enumerate}
