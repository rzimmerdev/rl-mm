

Atenção a esta parte. Eu puxei para frente a definição de objetivo do agente

A pesquisa tem como objetivo central a criação de uma estratégia eficaz de market making em mercados financeiros de alta frequência. Tem com foco na minimização do risco overnight e na maximização do desempenho financeiro da carteira do agente. 

Nosso objetivo central é desenvolver uma estratégia de market making que insira ofertas de compra e venda de forma eficiente no livro de ofertas, considerando as restrições do mercado, como o bid-ask spread e as limitações de tempo e liquidez. Esta estratégia busca equilibrar a minimização do risco overnight com a maximização do potencial de lucro.

A pesquisa também envolverá a aplicação de técnicas de aprendizado de reforço para modelar o comportamento do agente de market making. Definimos como o objetivo do agente, maximizar o valor da sua carteira no final do dia, $T>t$ ao longo de todos os negócios $e=1,...,N$ ocorridos no dia
\begin{equation*}
    \max V_T := \sum_{e=1}^N q_e \cdot P_e, 0<e\leq T, 
\end{equation*}
sub a restrição que a quantidade final em estoque esteja zero: $q_T:= \sum_{e=1}^N q_e = 0$
Para alcançarmos nossos obbjetivos será necessário incluir a definição de estados, ações e recompensas, bem como a formulação de uma política que orientará as decisões do agente ao longo do tempo. Por fim, pretendemos testar e realizar uma avaliação abrangente do desempenho da estratégia de market making. Isso incluirá a análise do valor da carteira ao longo do tempo, levando em consideração custos de transação e flutuações nos preços dos ativos. Avaliaremos também a eficácia da estratégia na minimização do risco overnight e seu impacto na liquidez do mercado.

Uma das finalidades da pesquisa é realizar uma análise comparativa entre a estratégia de market making desenvolvida com aprendizado de reforço e estratégias tradicionais de mercado. Isso nos permitirá destacar as vantagens e desvantagens da abordagem de aprendizado de reforço.

Em cenários onde a calibragem de parâmetros for necessária, investigaremos como esses parâmetros podem ser ajustados dinamicamente ao longo do tempo, levando em consideração as condições do mercado e as expectativas do agente. Isso contribuirá para a adaptação contínua da estratégia. O sucesso desta pesquisa será medido pela capacidade de desenvolver uma estratégia de market making que seja capaz de superar estratégias tradicionais, minimizar o risco \textit{overnight} e se adaptar dinamicamente às condições de mercado, contribuindo para a eficiência e aprimoramento das atividades de market making em mercados financeiros de alta frequência.

Definimos como o objetivo do agente, maximizar o valor da sua carteira no final do dia, $T>t$ ao longo de todos os negócios $e=1,...,N$ ocorridos no dia
\begin{equation*}
    \max V_T := \sum_{e=1}^N q_e \cdot P_e, 0<e\leq T, 
\end{equation*}
sub a restrição que a quantidade final em estoque esteja zero: $q_T:= \sum_{e=1}^N q_e = 0$
