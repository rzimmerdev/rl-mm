Neste projeto de pesquisa propomos realizar uma análise de estratégias de \textit{market-making} (\textit{MM}) existentes no mercado financeiro, assim como o uso de técnicas de Aprendizado por Reforço (\textit{AR}) para otimização do retorno de uma estratégia de \textit{MM}. Recentes avanços literários na área de \textit{AR} para otimização de agentes de negociação focam na busca por políticas que maximizem o retorno diário desses agentes e minimizam o risco das ordens e do inventário gerenciado pelos mesmos. Tal risco associado a uma estratégia de \textit{market-making} vem da falta de garantia de conseguir executar ordens criadas, e dependendo do processo de chegada de ofertas de compras e vendas, pode ser que o agente não tenha nenhuma de suas ordens executadas, ou até mesmo feche o dia com um retorno negativo. Considerando essa definição de risco e uma lacuna na literatura voltada à minimização do risco pós-fechamento de mercado (\textit{overnight}), buscamos realizar uma pesquisa bibliográfica sobre técnicas de \textit{AR} para otimização de políticas de negociação, assim como a conceitualização e treino de um agente de \textit{MM} que maximize o retorno diário sob a restrição de finalizar o dia sem posição remanescente, ou seja, zerar o risco noturno associado. 
