Neste projeto de pesquisa propomos realizar uma análise de estratégias financeiras existentes para criação de mercados (\textit{market-making} em inglês), assim como o possível uso de técnicas de Aprendizado por Reforço (\textit{AR}) para maximização do retorno esperado de tais estratégias. Recentes avanços literários na área de \textit{AR} para otimização de agentes financeiros focam na busca por políticas que maximizem o retorno diário e minimizem o risco das ordens e do inventário gerenciado por tais agentes. O risco associado a uma estratégia de \textit{market-making} vem da falta de garantia para execução das ordens criadas, e dependendo do processo de chegada das ofertas de compra e venda, pode ser que o agente não tenha nenhuma de suas ordens executadas, ou até mesmo feche o dia com um retorno negativo. Considerando essa definição de risco e uma lacuna na literatura voltada à minimização do risco pós-fechamento de mercado — chamado de risco \textit{overnight} — realizaremos uma pesquisa bibliográfica sobre técnicas de \textit{AR} para otimização de políticas de negociação, assim como a conceitualização e treino de um agente de \textit{MM} que maximize o retorno diário esperado sob a restrição de finalizar o dia sem posição remanescente, ou seja, zerar o risco noturno associado. 
%