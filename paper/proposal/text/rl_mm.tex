\subsection{Aprendizado por Reforço}
O Aprendizado por Reforço (RL) é um paradigma de aprendizado de máquina que se baseia em princípios da psicologia comportamental e se concentra em formalizar agentes autônomos capazes de tomar decisões em ambientes dinâmicos. Simplificadamente, trata-se de uma política de controle que interage com um ambiente de modo a maximizar uma recompensa cumulativa ao longo do tempo.

O processo de RL é análogo ao modo como os seres humanos aprendem por tentativa e erro. O agente explora diferentes ações, observa as consequências dessas ações no ambiente e ajusta sua estratégia com base nas recompensas obtidas. O objetivo final é desenvolver uma política de decisão que leve a ações que maximizem a recompensa esperada.

Essencialmente, um agente é composto por três elementos principais:

Política (\textit{Policy}): A política define a estratégia do agente, ou seja, como ele escolhe ações em resposta às observações do ambiente. Pode ser uma estratégia determinística ou estocástica.

Recompensa (\textit{Reward}): A recompensa é uma medida numérica que informa ao agente o quão boa ou ruim foi uma ação específica em um determinado estado do ambiente. O objetivo do agente é maximizar a recompensa cumulativa ao longo do tempo.

Modelo do Ambiente (\textit{Environment Model}): O modelo do ambiente representa as interações entre o agente e o ambiente. Ele descreve como as ações do agente afetam o estado do ambiente e como o ambiente responde às ações.

O RL é aplicado em diversos domínios, dá robótica até jogos e finanças. No contexto deste projeto, o RL desempenha um papel central na criação de um agente de market making capaz de tomar decisões inteligentes e adaptáveis em tempo real, minimizando o risco associado às operações de market making durante o período noturno.

\subsection{Market Making}
O princípio mais simples para obter lucro em atividades de negociação (\textit{trading}) consiste em tentar comprar um ativo num preço baixo e vendê-lo num preço maior.
Como os preços nos mercados são definidos por demanda e oferta e cada agente no mercado tem sua utilidade subjetiva em relação ao valor de um ativo, pode-se observar um livro de ofertas de compras e vendas nos mercados organizados, o \textit{limit order book} (LOB). 

Neste chamado "livro" são registradas todas as ofertas por prioridade de preço e tempo: as ofertas de compra de melhor preço ficam no topo do livro de compras, e se houver ofertas com o mesmo preço, a oferta que chegou primeiro tem prioridade. No lado das vendas, o livro é organizada com a mesma regra, porém, visualizada de forma invertida: a oferta de venda com o menor preço está no topo de livro, as ofertas de maior preço, no fundo.

Em caso de haver uma ou mais ofertas de compras com preço maior ou igual às ofertas de venda anunciadas, uma transação ocorrerá. Hoje em dia, nas bolsas de valores digitais a execução de uma transação é automática e executado por um sistema chamado de \textit{matching engine} - ou seja, um motor de pareamento de ordens.
Como consequência, a bolsa anuncia o fechamento de uma ordem e as ofertas relacionadas são removidas dos livros.
O market making é uma estratégia frequentemente usada nos mercados financeiros, fornecendo liquidez ao mercado.
O market maker é um participante do mercado que se fornece preços de compra (bid) e venda (ask) contínuos para um ativo financeiro específicopor meio de ordens de limite, acima ou abaixo do melhor preço de compra ou venda, respectivamente. Sua função é comprar ativos dos vendedores e vender ativos aos compradores, garantindo que sempre haja liquidez disponível no mercado.

Os principais elementos do market making incluem, mas não se limitam à:

Spread: É a diferença entre o preço de compra (bid) e o preço de venda (ask) oferecidos pelo market maker. O market maker lucra com a diferença entre esses preços.

Livro de Ordens de Limite: O market maker gerencia um livro de ordens de limite, onde as ordens de compra e venda dos participantes são registradas. Ele ajusta os preços de compra e venda com base nas mudanças nas condições do mercado para equilibrar a oferta e a demanda.

Gestão de Risco: O market maker enfrenta riscos, incluindo risco de inventário e risco de mercado. O risco de inventário ocorre quando o market maker mantém uma posição desequilibrada entre ativos comprados e vendidos, enquanto o risco de mercado está relacionado às flutuações nos preços dos ativos.

No contexto deste projeto, o market making é o foco da pesquisa, especificamente a minimização do risco associado às posições mantidas durante o período noturno. Isso envolve o desenvolvimento de estratégias inteligentes e adaptáveis para gerenciar o livro de ordens de limite e otimizar a alocação de recursos, enquanto se equilibra a busca de lucros com a redução do risco. O Aprendizado por Reforço (RL) é a abordagem escolhida para treinar um agente capaz de realizar essa tarefa de forma ótima.