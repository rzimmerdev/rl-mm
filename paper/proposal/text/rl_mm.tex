\subsection{Market Making}
O market-making é uma forma de negociação (\textit{trading} em inglês)  que de forma simplificada consiste em comprar certo ativo a um determinado preço e vender-lo a um preço maior.
Como os preços de compra e venda dos ativos são afetados pela demanda e oferta no momento, cada agente no mercado exerce certa influência sobre o valor de um ativo, a depender das ofertas que o mesmo mantém para tal ativo no livro de ordens limite (\textit{limit order book}, ou \textit{LOB}). 

No \textit{LOB} os registros se dão primeiramente por ordem de preço, e em segundo por data de criação: as ofertas com melhor preço (tanto para compra como para venda) ficam no topo do livro, e caso tenham o mesmo valor entre si tem sua posição desempatada pela ordem temporal de chegada. No livro de compras, a melhor oferta é a cuja ordem oferece o maior preço, e o contrário vale para o livro de vendas.

Nas bolsas de valores digitais a execução de uma transação é automática e auxiliada por um sistema chamado de \textit{matching engine} - ou seja, um motor para pareamento de ordens. Esse sistema verifica se para a melhor oferta de compra (ou venda) existe outra correspondente no livro de venda (ou compra) com valor menor ou igual (ou maior ou igual para venda). Após o pareamento, a bolsa anuncia o a execução da ordem e as ofertas relacionadas são removidas dos livros.

Em suma, os principais elementos do market making incluem, mas não se limitam à:

Spread: É a diferença entre o preço de compra (\textit{bid} em inglês) e o preço de venda (\textit{ask} em inglês) entre duas ofertas. O market maker busca lucrar com a diferença entre esses preços.

Livro de Ordens Limite: É o conjunto ordenado onde as ofertas de compra e venda são registradas. Também é permitido o ajuste dos preços de compra e venda de ofertas existentes por parte dos agentes.

Gestão de Risco: Todos \textit{market-makers} enfrentam riscos em suas negociações, entre eles o risco de inventário e risco de mercado. O risco de inventário ocorre quando o market maker mantém uma posição desequilibrada entre ativos comprados e vendidos, enquanto o risco de mercado está relacionado às flutuações nos preços dos ativos.

No contexto deste projeto, o foco da pesquisa será a otimização de uma estratégia de \textit{market-making} que minimize o risco associado às posições mantidas durante a noite. A estratégia será composta por um agente responsável pela interação com o mercado e alocação de recursos sob políticas para redução de risco. O Aprendizado por Reforço (RL) foi a abordagem escolhida para obter um agente ótimo capaz de realizar essa tarefa.

\subsection{Aprendizado por Reforço}
O Aprendizado por Reforço (RL) é um paradigma de aprendizado de máquina baseado em princípios da psicologia comportamental e em otimização estocástica, especificamente tarefas de controle ótimo. Simplificadamente, trata-se de uma técnica para modelagem, simulação e treino de um agente capaz de interagir com um ambiente dinâmico de modo a maximizar uma recompensa cumulativa ao longo do tempo.

O processo de RL é análogo ao modo como os seres humanos aprendem por tentativa e erro. Um agente explora diferentes ações, observa as consequências dessas ações no ambiente e ajusta sua estratégia com base nas recompensas obtidas e no novo estado do ambiente. O objetivo final é a obtenção de uma política que maximize a recompensa esperada.

Essencialmente, um agente consiste de três componentes:

Política (ou \textit{Policy}): Define o processo de tomada de decisões do agente, ou seja, como ele escolhe ações em resposta às observações do ambiente. Pode ser uma estratégia determinística ou estocástica.

Recompensa (ou \textit{Reward}): É uma medida numérica que permite ao agente interpretar quão boa ou ruim foi uma ação específica em função do estado atual do ambiente. O objetivo do agente é maximizar a recompensa cumulativa ao longo do tempo.

Modelo do Ambiente (ou \textit{Environment}): Representa e simula o estado do ambiente real, assim como as possíveis ações que o agente possa realizar. É uma descrição quantitativa de elementos do ambiente e como as ações tomadas pela política afetam o ambiente em si.

O RL pode ser aplicado à diversos domínios, da robótica, jogos à finanças e no contexto deste projeto será o paradigma central na idealização e modelagem da estratégia de \textit{market-making}.
