%% Copyright 2019 Matheus H. J. Saldanha <mhjsaldanha@gmail.com>
%
% This work may be distributed and/or modified under the
% conditions of the LaTeX Project Public License, either version 1.3
% of this license or (at your option) any later version.
% The latest version of this license is in
%   http://www.latex-project.org/lppl.txt
% and version 1.3 or later is part of all distributions of LaTeX
% version 2005/12/01 or later.
%
% This work has the LPPL maintenance status `maintained'.

\documentclass[12pt,a4paper]{article}

% Pacotes para o português.
\usepackage[utf8]{inputenc}
\usepackage[T1]{fontenc}
\usepackage{enumitem}
\usepackage{natbib}
\usepackage[titleref,user]{zref}
\ztitlerefsetup{}

\usepackage{graphicx}     % Comando \includegraphics
\usepackage{xcolor}       % Comando de cores \textcolor
\usepackage{indentfirst}  % Indenta o primeiro parágrafo de cada seção
\usepackage{url}          % Comandos \url e \href
\usepackage[top=2cm, bottom=2cm, left=2cm, right=2cm]{geometry} % Define as margens do documento
\usepackage{multirow}     % Permite criar tabelas com uma célula ocupando várias linhas
\usepackage{amssymb}      % Símbolos matemáticos
\usepackage{amsmath}      % Ambientes para escrever fórmulas, \begin{align} por exemplo.
\usepackage{caption}      % Para definir o estilo das legendas de figuras e tabelas.
\usepackage{setspace}     % Para definir espaçamento entre linhas. (\onehalfspacing, \singlespacing, \doublespacing)
\usepackage{breakcites}   % Para permitir quebra de linha no meio de citações.
\usepackage{times}        % Fonte Times New Roman
\usepackage{lipsum}       % Para gerar texto temporário. Exemplo: \lipsum \lipsum[1] \lipsum[4-5].
\usepackage{inconsolata}  % Fonte boa para códigos e URLs. Use \texttt{}
\usepackage{hyperref}     % Faz os links ficarem azuis e clicáveis. Facilita a navegação pelo PDF.

\usepackage[pdf]{graphviz}
\usepackage{pgfplots}
\usepackage{standalone}
\usepackage{tikz}
\usepackage{float}
\usepackage{pdfpages}

\pgfplotsset{compat=1.17}

\makeatletter
\hypersetup{
	pdfkeywords={research project},
	colorlinks=true,       		% false: boxed links; true: colored links
	linkcolor=blue,          	% color of internal links
	citecolor=blue,        		% color of links to bibliography
	filecolor=magenta,      	% color of file links
	urlcolor=blue,
	bookmarksdepth=4,
}
\makeatother

\makeatletter
\renewcommand\tableofcontents{%         % Redefine table of contents to our taste
    \section*{\huge\centering\contentsname
        \@mkboth{%
           \MakeUppercase\contentsname}{\MakeUppercase\contentsname}}%
           \vspace{24pt}%
    \@starttoc{toc}%
    \newpage%
}

% Comando para marcar o texto para revisão.
\newcommand{\rev}[1]{\textcolor{red}{#1}}

\newcommand*\itemlabel[1]{%
        \label{#1}%
        \zlabel{#1}%
}
\newcommand*\itemref[1]{\ref{#1}}
\newcommand*\secref[1]{``\ztitleref{#1}''}

% Permite escrever aspas normais "text" em vez de ``text''
\usepackage[autostyle]{csquotes}
\MakeOuterQuote{"}

\begin{document}

\doublespacing

\begin{titlepage}
    \begin{center}
        {\large \sc UNIVERSIDADE DE SÃO PAULO} \\
        {\large \sc INSTITUTO DE CIÊNCIAS MATEMÁTICAS E DE COMPUTAÇÃO}\\[0.7cm]
        % {\small \sc DEPARTAMENTO DE SISTEMAS DE COMPUTAÇÃO}
        
        \vspace{4cm}

        % Título.
        {\large \sc Projeto de Pesquisa FAPESP}\\
        \rule{\linewidth}{2pt}
        
        \vspace{0.7em} % Ajuste ao seu gosto
        {\Large \bfseries Aprendizado por Reforço para otimização de uma estratégia de Market-Making }
        \vspace{0.2em} % Ajuste ao seu gosto
        
        \rule{\linewidth}{2pt} \\
        {\small \sc Linha de Fomento: Bolsa no País - Regular - Iniciação Científica}
    \end{center}
    
    \vspace{2.8cm}

    % Assinaturas
    \begin{minipage}{0.43\textwidth}
        \emph{Candidato:}\\[2.08cm]
        \rule{0.9\linewidth}{0.3mm}\\
        Rafael Zimmer
    \end{minipage}
    \hspace{1cm}
    \begin{minipage}{0.43\textwidth}
        \emph{Orientador:}\\[2.08cm]
        \rule{0.9\linewidth}{0.3mm}\\
    \end{minipage}

    \vfill

    % Data
    \begin{center}
        \makeatletter
        São Paulo \\
        \@date
        \makeatother
    \end{center}
\end{titlepage}


\pagestyle{empty}
\begin{center}
    {\bf \huge Resumo} \\[3em] % Dá um pulo de cerca de 3 linhas
\end{center}
% Neste projeto de pesquisa propomos realizar uma análise de estratégias de \textit{market-making} (\textit{MM}) existentes no mercado financeiro, assim como o uso de técnicas de Aprendizado por Reforço (\textit{AR}) para otimização do retorno de uma estratégia de \textit{MM}. Recentes avanços literários na área de \textit{AR} para otimização de agentes de negociação focam na busca por políticas que maximizem o retorno diário desses agentes e minimizam o risco das ordens e do inventário gerenciado pelos mesmos. Tal risco associado a uma estratégia de \textit{market-making} vem da falta de garantia de conseguir executar ordens criadas, e dependendo do processo de chegada de ofertas de compras e vendas, pode ser que o agente não tenha nenhuma de suas ordens executadas, ou até mesmo feche o dia com um retorno negativo. Considerando essa definição de risco e uma lacuna na literatura voltada à minimização do risco pós-fechamento de mercado (\textit{overnight}), buscamos realizar uma pesquisa bibliográfica sobre técnicas de \textit{AR} para otimização de políticas de negociação, assim como a conceitualização e treino de um agente de \textit{MM} que maximize o retorno diário sob a restrição de finalizar o dia sem posição remanescente, ou seja, zerar o risco noturno associado. 

Neste projeto de pesquisa propomos realizar uma análise de estratégias de \textit{market-making} (\textit{MM}) existentes no mercado financeiro, assim como o uso de técnicas de Aprendizado por Reforço (\textit{AR}) para otimização do retorno de uma estratégia de \textit{MM}. Recentes avanços literários na área de \textit{AR} para otimização de agentes de negociação focam na busca por políticas que maximizem o retorno diário desses agentes e minimizam o risco das ordens e do inventário gerenciado pelos mesmos. Tal risco associado a uma estratégia de \textit{market-making} vem da falta de garantia de conseguir executar ordens criadas, e dependendo do processo de chegada de ofertas de compras e vendas, pode ser que o agente não tenha nenhuma de suas ordens executadas, ou até mesmo feche o dia com um retorno negativo. Considerando essa definição de risco e uma lacuna na literatura voltada à minimização do risco pós-fechamento de mercado (\textit{overnight}), buscamos realizar uma pesquisa bibliográfica sobre técnicas de \textit{AR} para otimização de políticas de negociação, assim como a conceitualização e treino de um agente de \textit{MM} que maximize o retorno diário sob a restrição de finalizar o dia sem posição remanescente, ou seja, zerar o risco noturno associado. 



\noindent{}
\newpage
\pagestyle{empty} % Não numera a página
\tableofcontents
\newpage
\setcounter{page}{1}
\pagestyle{plain} % Agora passa a numerar as páginas

\section{Introdução e justificativa}
\label{section:introduction}
\section{Introduction}
\label{sec:introduction}

Market making in financial markets involves continuously quoting buy and sell prices and dealing with supply and demand imbalances.
Market makers are essential, as they narrow the bid-ask spreads, reduce price volatility, and maintain market stability,
particularly in times of uncertainty by providing liquidity, which helps to maintain a fair and efficient market~\cite{Glosten1985, OHara1995}.
With the advent of electronic trading, placing optimal bid and ask prices as a market maker is becoming an almost completely automatized task,
but such a transition comes with the hardships of dealing with additional caveats, including slippage, market impact, adverse market regimes~\cite{Cont2010, Bouchaud2018}
and non-stationary conditions~\cite{Gasperov2021}.

The Reinforcement Learning (RL) paradigm has shown promising results for optimizing market making strategies,
where agents learn to adapt their quote placing policies through trial and error given a numerical outcome reward score\footnote{Not to be confused with financial returns.}.
The RL approach is based on the Bellman equation for state values or state-action pair values,
which recursively define the state-value of a policy as the expected return of discounted future rewards.
State-of-the-art RL algorithms, such as Proximal Policy Optimization (PPO) and Deep Q-Learning (DQN),
solve the Bellman equation by approximating the optimal policy using neural networks to learn either the policy, the value function~\cite{Sutton2018},
or both~\cite{Schulman2015, Mnih2015}, aiming to allow the agent to its quoted prices based adaptively on the observed market conditions~\cite{He2023, Bakshaev2020}.

Using historical data to train RL agents is a common practice in the literature, but it has some limitations,
as it is computationally expensive and requires a large amount of data,
besides not including the effects of market impact and inventory risk~\cite{Frey2023, Ganesh2019} during the training process.
An additional approach to creating realistic environments are agent-based simulations,
where generative agents are trained against observed market messages and used to simulate the limit order book environment~\cite{Frey2023, Ganesh2019},
but this approach has the disadvantage of trading off control over the market dynamics for replicating observed market flow,
as well as limiting the agent's adaptability to unseen market regimes, which can lead to underfitting~\cite{Jerome2022, Selser2021}
and suboptimal decision-making under scenarios where market impact and slippage have a more prominent effect on the agent's performance.
On the other hand, using stochastic models to simulate the limit order book environment is computationally cheaper and faster to train~\cite{Gasperov2021, Sun2022},
but might not generalize well to unseen market conditions.

In the context of this paper, we implement a RL-based market making agent and test its robustness and generalization capabilities under non-stationary environments.
The agent is trained on a crafted simulator that models the dynamics of a limit order book (LOB) according to a set of parameterizable stochastic processes.
This approach aims to demonstrate the effects of non-stationary environments and adverse market conditions on
a RL-based market making agent, and how the agent's decision-making policies are affected by market impact and inventory risk.
Our main contribution is implementing multiple non-stationary dynamics into a single limit order book simulator,
and how using fine-controlled non-stationary environments can enhance the agent's performance under adverse market conditions
and provide a more realistic training environment for RL agents in market-making scenarios.
We also perform a comparative analysis of the agent's performance under changing market conditions during the
trading day and benchmark it against a closed-expression optimal solution (under a simplified market model),
aiming to validate the usage of RL-based market making agents in markets with more complex dynamics.



\section{Das áreas a serem abordadas}
\label{section:rl_mm}
\subsection{Aprendizado por Reforço}
O Aprendizado por Reforço (RL) é um paradigma de aprendizado de máquina que se baseia em princípios da psicologia comportamental e se concentra em formalizar agentes autônomos capazes de tomar decisões em ambientes dinâmicos. Simplificadamente, trata-se de uma política de controle que interage com um ambiente de modo a maximizar uma recompensa cumulativa ao longo do tempo.

O processo de RL é análogo ao modo como os seres humanos aprendem por tentativa e erro. O agente explora diferentes ações, observa as consequências dessas ações no ambiente e ajusta sua estratégia com base nas recompensas obtidas. O objetivo final é desenvolver uma política de decisão que leve a ações que maximizem a recompensa esperada.

Essencialmente, um agente é composto por três elementos principais:

Política (\textit{Policy}): A política define a estratégia do agente, ou seja, como ele escolhe ações em resposta às observações do ambiente. Pode ser uma estratégia determinística ou estocástica.

Recompensa (\textit{Reward}): A recompensa é uma medida numérica que informa ao agente o quão boa ou ruim foi uma ação específica em um determinado estado do ambiente. O objetivo do agente é maximizar a recompensa cumulativa ao longo do tempo.

Modelo do Ambiente (\textit{Environment Model}): O modelo do ambiente representa as interações entre o agente e o ambiente. Ele descreve como as ações do agente afetam o estado do ambiente e como o ambiente responde às ações.

O RL é aplicado em diversos domínios, dá robótica até jogos e finanças. No contexto deste projeto, o RL desempenha um papel central na criação de um agente de market making capaz de tomar decisões inteligentes e adaptáveis em tempo real, minimizando o risco associado às operações de market making durante o período noturno.

\subsection{Market Making}
O princípio mais simples para obter lucro em atividades de negociação (\textit{trading}) consiste em tentar comprar um ativo num preço baixo e vendê-lo num preço maior.
Como os preços nos mercados são definidos por demanda e oferta e cada agente no mercado tem sua utilidade subjetiva em relação ao valor de um ativo, pode-se observar um livro de ofertas de compras e vendas nos mercados organizados, o \textit{limit order book} (LOB). 

Neste chamado "livro" são registradas todas as ofertas por prioridade de preço e tempo: as ofertas de compra de melhor preço ficam no topo do livro de compras, e se houver ofertas com o mesmo preço, a oferta que chegou primeiro tem prioridade. No lado das vendas, o livro é organizada com a mesma regra, porém, visualizada de forma invertida: a oferta de venda com o menor preço está no topo de livro, as ofertas de maior preço, no fundo.

Em caso de haver uma ou mais ofertas de compras com preço maior ou igual às ofertas de venda anunciadas, uma transação ocorrerá. Hoje em dia, nas bolsas de valores digitais a execução de uma transação é automática e executado por um sistema chamado de \textit{matching engine} - ou seja, um motor de pareamento de ordens.
Como consequência, a bolsa anuncia o fechamento de uma ordem e as ofertas relacionadas são removidas dos livros.
O market making é uma estratégia frequentemente usada nos mercados financeiros, fornecendo liquidez ao mercado.
O market maker é um participante do mercado que se fornece preços de compra (bid) e venda (ask) contínuos para um ativo financeiro específicopor meio de ordens de limite, acima ou abaixo do melhor preço de compra ou venda, respectivamente. Sua função é comprar ativos dos vendedores e vender ativos aos compradores, garantindo que sempre haja liquidez disponível no mercado.

Os principais elementos do market making incluem, mas não se limitam à:

Spread: É a diferença entre o preço de compra (bid) e o preço de venda (ask) oferecidos pelo market maker. O market maker lucra com a diferença entre esses preços.

Livro de Ordens de Limite: O market maker gerencia um livro de ordens de limite, onde as ordens de compra e venda dos participantes são registradas. Ele ajusta os preços de compra e venda com base nas mudanças nas condições do mercado para equilibrar a oferta e a demanda.

Gestão de Risco: O market maker enfrenta riscos, incluindo risco de inventário e risco de mercado. O risco de inventário ocorre quando o market maker mantém uma posição desequilibrada entre ativos comprados e vendidos, enquanto o risco de mercado está relacionado às flutuações nos preços dos ativos.

No contexto deste projeto, o market making é o foco da pesquisa, especificamente a minimização do risco associado às posições mantidas durante o período noturno. Isso envolve o desenvolvimento de estratégias inteligentes e adaptáveis para gerenciar o livro de ordens de limite e otimizar a alocação de recursos, enquanto se equilibra a busca de lucros com a redução do risco. O Aprendizado por Reforço (RL) é a abordagem escolhida para treinar um agente capaz de realizar essa tarefa de forma ótima.

\section{Compreendendo o Problema Inicial}
\label{section:problem_description}
\subsection{Definição formal do \textit{market making}}

Os mercados normalmente definem uma quantidade mínima por lote ofertado, diferente em cada bolsa (e.g de 100 em 100 ações como no mercado padrão da B3). Um vendedor pode consequentemente enviar uma ordem de venda de 100 ações, mas não de 50. Se os preços ofertados no momento permitirem a execução de um negócio, a transação ocorrerá em cima da quantidade disponível, ou seja, se o mesmo vendedor que oferta a venda de 100 ações fecha um negócio com um comprador buscando 200 ações por exemplo, a quantidade negociada será de apenas 100. O restante da oferta de compra, i.e. 100 ações, continua no livro de oferta de compras, enquanto no lado da venda, a próxima oferta de maior preferência ficará no topo do livro. 

Se definirmos como $a^{0}_{t}$ a oferta de venda com menor preço\footnote{A melhor oferta está na posição 0 da fila, decrescente para vendas e crescente para compras.} no instante $t$ e $b^{0}_{t}$ como a oferta de compra com maior preço, então para um mercado sem oportunidade de negócio no mesmo instante $t$, tem-se que $a^{j}_{t} > b^{i}_{t}$. Além disso, a diferença entre as ofertas $a^{0}_{t} - b^{0}_{t} = \mathbf{bas}_t$ é chamado de \textit{spread} no momento $t$.

O objetivo de uma estratégia de \textit{market making} (MM) nesse contexto é inserir ordens no livro de ofertas no momento atual $t$, podendo ser: 

\begin{itemize}
    \item ofertas de venda representada por $a^{mm}_{t} < a^{0}_{t}$; ou 
    \item ofertas de compra com $b^{mm}_{t} > b^{0}_{t}$;
\end{itemize}
sendo que quaisquer ordens criadas não devem iniciam novas transações.

Para ilustrar melhor o funcionamento do livro de ordens, imagine que um agente qualquer de \textit{MM} tenha em aberto uma oferta de venda pelo preço $a^{mm}_{t}$. Caso surja uma nova oferta de compra com preço $b^{n}_{t} \geq a^{mm}_{t}$ (note que caso isso ocorra também valerá $b^{n}_{t} \geq b^{0}_{t}$, pois o preço da oferta do agente de \textit{MM} sempre está atrás da melhor oferta, que é a oferta $b_{t}^{0}$), isso acarretará em um negócio entre os dois agentes, e a remoção de ambas ofertas $b_{t}^{n} \text{ e } a_{t}^{mm}$ do livro de ordens. Em seguida, o agente tem sua posição $V_{t}$ ajustada:
\begin{equation}
    V_{t + 1}:= V_{t} + a^{mm}_{t}\cdot q,
\end{equation}
sendo $V_{t + 1}$ a nova posição do agente e $q$ a quantidade de ativos ofertada em questão.

O agente então escolhe esperar ou realizar uma das ações abaixo:
\begin{enumerate}
    \item inserir uma nova oferta de venda, substituindo a oferta $a_t^{mm}$ anterior;
    \item inserir ou ajustar uma oferta de compra existente no livro de ofertas de compras;
\end{enumerate}

A decisão do agente irá depender de sua expectativa sobre a evolução do mercado, incluindo, mas não limitado à
\begin{itemize}
    \item probabilidade de chegar uma oferta de compra com preço superior $b_{t + s}^{n} > b_{t}^{mm}$, $s > 0$; \footnote{\label{agent}\textit{n} é outro agente ativo no mercado}
    \item probabilidade de chegar uma oferta de venda com preço inferior $a_{t + s}^{n} \leq a_{t}^{mm}$, $s > 0$; \footref{agent}
    \item risco futuro da posição ultrapassar os limites estabelecidos pelas corretoras.
\end{itemize}

Por exemplo, um agente esperando a chegada de uma oferta de venda que possa casar com sua oferta de compra $b_{t}^{mm}$ decide não ajustar a sua ordem atual e aguardar a execução, lucrando assim o \textit{BAS} menos os custos de transação:
\begin{equation*}
    v_{t + s} = q \cdot \left(a_t^{mm} - b_t^{mm}\right) - c, s > 0,
\end{equation*}
sendo que $q > 0$ a quantidade de ações negociadas em questão, e $v_{t + s}, s > 0$ representa o resultado das operações do agente após a execução de uma ordem e $c$ o custo por transação.\footnote{Os custos de transação incluem tipicamente os custos da corretora e emolumentos das bolsas. Custos de custódio de terceiros não pertencem aos cálculos, já que o agente não pretende manter a posição depois do fechamento do pregão.} Além disso, nem sempre pode-se esperar que uma venda ocorre em sequência a uma compra (e vice-versa), ou que as execuções ocorram imediatamente.

Portanto, o agente de \textit{MM} não observa o resultado individual da última operação, mas sim a sua posição total $V_{t}$, que pode ser reconstruída a partir das operações de compra $b\in B_t$ e operações de venda $a\in A_t$ executadas até o momento \textit{t}: \footnote{Para fins expositivos, usamos a notação $P_s^{ask_0}$ para representar o melhor preço de venda e $P_s^{bid_0}$, mas o leitor deve se atender à possibilidade que a quantidade negociada ($q_b$, $q_a$) poderia ultrapassar a quantidade exposta no livro. Nessa situação, o preço executado seria um preço médio das operações realizadas até a execução total da quantidade. }
\begin{equation}
    V_{t} := \sum_{b \in B_t} q_{b_{t}} \cdot b_{t}^{0} - \sum_{a \in A_t} q_{a_{t}} \cdot a_t^{0} - c
\end{equation}
sendo que $\sum_{b=1}^{B_t} q_b \cdot b_{t}^{mm} $ representa o valor das posições compradas, avaliadas na data $t$  usando o preço de compra observado no instante; $\sum_{a=1}^{A_t} q_a \cdot a_{t}^{mm}$ representa o valor das posições vendidas na data $t$, usando o possível preço de recompra no instante, e $V_t$ é o processo de consumo, que se compõe pelos custos de compra, as receitas das vendas e os custos de transação. Note que $b_{t}^{mm}$ e $a_{t}^{mm}$ representam os preços de compra e venda, respetivamente, que foram executados nas diversas operações. 

\subsection{Definição do objetivo do agente de MM}
O objetivo principal do agente de \textit{MM} é maximizar o valor da sua carteira (portfólio). Isso é feito pelas etapas intermediárias de maximizar o \textit{BAS} e a quantidade de transações realizadas:

\begin{equation}
\begin{aligned}
\max_{b \in B_{t}, a \in A_{t}} \quad & V_T \label{eq:target_fct}\\
% \textrm{s.t.} \quad & y_{i}(w\phi(x_{i}+b))+\xi_{i}-1\\
  % &\xi\geq0    \\
\end{aligned}
\end{equation}

O agente também tem uma restrição adicional, de que ao final do dia não esteja exposto a nenhum risco de mercado. 
Existem algumas alternativas para formalizar essa restrição:
\begin{enumerate}
    \item No final do dia, o agente não pode ter nenhum ativo em posição: 
    \begin{equation}
        \sum_{b=1}^{B_t} q_b  - \sum_{a=1}^{A_t} q_a = 0\label{eq:eod_restriction}
    \end{equation}
    \item No final do dia, se houver alguma posição, o agente precisa \textit{headgear} a exposição ao risco, comprando (vendendo) futuros ou outros derivativos, dependendo da posição remanescente.\footnote{A atividade de \textit{hedge}, de forma informal, consiste em comprar ou vender ativos que tenham uma exposição ao risco de tal forma que compense os riscos da carteira atual.}
\end{enumerate}

\subsection{\textit{Market making} simultâneo}
Ao considerarmos a situação em que o MM aplica a sua estratégia em diversos mercados simultaneamente observamos um aumento da complexidade, mas também das alternativas para lidar com riscos envolvidos - chamamos essa situação de MM \textbf{simultâneo} ou \textbf{multivariado}.

O valor da posição do agente passa a ser a soma de todos $N$ ativos:
\begin{eqnarray*}
    V_t &:=& \sum_{i=1}^N V_{t,i}\\
    V_{t,i} &:=& \sum_{b_{i}=1}^{B_{i}} q_{b_{i}} \cdot b_{t}^{0} 
    \sum_{a_{i}=1}^{A_{i}} q_{a_{i}} \cdot a_{t}^{0} - c
\end{eqnarray*}

O objetivo principal (\ref{eq:target_fct}) continua o mesmo, e mantém-se a restrição (\ref{eq:eod_restriction}). Contudo, surgem novas alternativas para proteção da carteira durante a noite:

\begin{enumerate}
    \item O agente pode avaliar o risco global da carteira, e incluir um único ativo de proteção contra o risco global ao final do dia.
    \item Se o ativo for negociado em múltiplas bolsas (chamado também de ativo \textit{co-listed}), o agente pode continuar a negociação deste em outra bolsa caso uma delas esteja fechada.
\end{enumerate}

Considerando um cenário em que haja ações em \textit{co-listing}, surge a possibilidade de criação de estratégias mais sofisticadas, permitindo a implementação dos itens mencionados acima.
 

\section{Trabalhos prévios}
\label{section:previous_works}
A pesquisa sobre a minimização de risco overnight usando Aprendizado por Reforço (RL) para estratégias de market making está inserida em um contexto mais amplo de estudos que exploram a aplicação do RL em finanças quantitativas. Nesta seção, levantamos uma lista inicial de trabalhos relacionados que ajudaram a moldar e fundamentar a proposta deste projeto.

A pesquisa formal do problema do \textit{Market Making} (MM) foi iniciada pelo estudo de  Marco Avellaneda e Sasha Stoikov, em seu trabalho seminal de 2008 \citep{Avellaneda2008}. Eles abordaram o problema do MM sob certas suposições relacionadas aos processos de chegada de ordens de compra e venda. No entanto, é importante observar que, nesse cenário, eles não impuseram a restrição de que o inventário do \textit{market maker} no final do dia fosse diferente de zero, ou seja, $q_T \neq 0$.

Como um dos resultados, Avellaneda e Stoikov conseguiram definir uma estratégia ótima para o MM, que se baseia em um cálculo cuidadoso das cotações de compra e venda em resposta às chegadas de ordens de mercado. Esta estratégia foi derivada dentro de um quadro teórico e matemático bem definido, oferecendo uma proposta clara sobre como um \textit{market maker} pode otimizar seu desempenho em um livro de ordens.

Ao longo do tempo, diversos autores começaram a relaxar algumas das hipóteses feitas por Avellaneda e Stoikov, tornando o cenário de MM mais genérico e realista. Esse avanço na literatura expandiu as possibilidades de modelagem e análise de estratégias de MM em ambientes mais complexos e dinâmicos. Alguns exemplos são:
\begin{itemize}
    \item ``Optimal Market Making with Limited Risk'' \citep{Gueant2017}: Este estudo aborda especificamente o problema do market making sob a perspectiva da minimização do risco. Os autores desenvolvem um modelo de market making que leva em consideração restrições de risco e investigam como otimizar a estratégia de market making enquanto limitam o risco associado.
    \item ``High-frequency trading in a limit order book'' \citep{Avellaneda2008}: Este estudo investiga as estratégias de trading de alta frequência em um livro de ordens de limite. Embora não aborde diretamente o uso do RL, fornece insights valiosos sobre o funcionamento de mercados eletrônicos e os desafios enfrentados pelos market makers, incluindo a gestão de risco e a necessidade de ajustar os preços rapidamente.
\end{itemize}


Nos últimos anos, o uso do paradigma de Aprendizado por Reforço se tornou mostrou extremamente útil para tarefas mais complexas, de jogos à medicina \citep{Kaelbling1996}. 
Aplicações específicas do método de \textit{RL} no estudo de problemas de \textit{MM} receberam grande atenção por alguns autores recentemente, notavelmente em:
\begin{itemize}
    \item "Reinforcement Learning Approaches to Optimal Market Making" \citep{Gasperov2021}: Este estudo fornece uma visão abrangente das aplicações do Aprendizado por Reforço em market making. Os autores demonstram como o RL pode ser usado para ajustar dinamicamente os preços de compra e venda em resposta às condições do mercado. Eles destacam a eficácia do RL em otimizar o retorno ajustado ao risco em comparação com estratégias tradicionais.
    \item "Reinforcement Learning for Market Making in a Multi-agent Dealer Market" \citep{Ganesh2019}: Este artigo oferece uma visão detalhada de como o RL pode ser aplicado em um ambiente de mercado com vários agentes, semelhante ao cenário do mundo real. Os autores demonstram que um agente de RL pode aprender a adaptar suas estratégias de market making em resposta às ações de outros agentes e às condições do mercado, incluindo a gestão de risco.
\end{itemize}


Ao revisar esses trabalhos relacionados, podemos observar que, apesar dos tratamentos teóricos bem elaborados, a situação real do uso do MM não se encaixa nas limitações impostas pelas pesquisa:

\begin{itemize}
    \item nos mercados financeiros reais, os agentes operam de uma forma mais complexa que assumido nas pesquisas: quase todos usam estratégias MM simultâneas;
    \item as alternativas de proteção e as restrições de posicionamento, especialmente numa estratégia simultânea, são bem mais abrangentes que na literatura atual. 
\end{itemize}

Em situações reais, diferente da proposta de \citet{Avellaneda2008}, não é possível encontrar uma estratégia ótima de forma analítica. O uso de técnicas de Aprendizado por Reforço em estratégias de market making oferece um potencial significativo para melhorar a eficiência das operações financeiras e mitigar os riscos associados, especialmente o risco overnight. 



\section{Objetivos}
\label{section:objectives}
A pesquisa tem como objetivo central a criação de uma estratégia específica de \textit{market making} ótima. Nesse contexto, planejamos obter tanto um método para otimização da política de escolha de preços, como também da criação do ambiente de simulação para o livro de ofertas limite e de outros agentes participantes do mercado.

O foco principal é a aplicação de técnicas de aprendizado por reforço para modelar o comportamento do agente de market making e inserir uma restrição essencial para minimizar o risco do agente. Em cenários onde a calibragem de parâmetros é necessária, serão realizados ajustes dos mesmos ao longo do tempo, de modo a levar em consideração as condições do mercado e as expectativas do agente.

Como conclusão da pesquisa, pretendemos realizar uma avaliação abrangente do desempenho da estratégia de market making, incluindo mas não limitado a análise do retorno da carteira ao longo do tempo, levando em consideração custos de transação e flutuações nos preços dos ativos. Avaliaremos também a eficácia da estratégia sob a restrição de risco \textit{overnight} máximo e o impacto das ordens geradas para zerar a carteira na liquidez do mercado. Após o treinamento do agente, realizaremos também uma análise comparativa entre a estratégia desenvolvida e estratégias tradicionais de mercado, especificamente de \textit{price-taking}. Isso nos permitirá destacar as vantagens e desvantagens da abordagem de aprendizado de reforço e comparar estatisticamente os resultados obtidos.


\section{Metodologia}
\label{section:methodology}
\section{Methodology}
\label{sec:methodology}

\subsection{Problem Definition}
\label{subsec:problem-definition}

The market-making problem addressed in this work involves designing an optimal trading policy for an agent using reinforcement learning.
The policy should be able to balance profit and risk, particularly inventory risk under the restriction of having zero inventory at market close,
and interacts with the environment by quoting bid and ask prices and adjusting offered quantities.
The environment dynamics are modeled by an underlying limit order book (LOB) and its behaviour will be described in the following sections.
The agent's objective is to maximize cumulative rewards over time, while minimizing inventory risk and market impact costs.

\subsection{A Formal Description of the Chosen RL Environment}
\label{subsec:formal-description-of-the-rl-environment}
In modeling the environment, we initially utilize a continuous-time, continuous-state Markov Chain framework,
and later transition to a discrete implementation to address computational space constraints.
The specific case in which a Markov Chain also has an associated reward distribution $R$ for each state transition is called a Markov Reward Process
and given that the agent's decisions also affect the transition probabilities due to market impact it is therefore called a Markov Decision Process (MDP) in control literature.
A Markov Decision Process can generically be defined as a 4-tuple $ (\mathcal{S}, \mathcal{A}, \mathbb{P}, R) $, where:

\begin{itemize}
    \item $\mathcal{S}$ is a set of states called the state space.
    \item $\mathcal{A}$ is a set of actions called the action space.
    \item $P: \mathcal{S} \times \mathcal{A} \times \mathcal{S} \to [0, 1]$ is the transition probability function for the MDP.
    \item $R: \mathcal{S} \times \mathcal{A} \times \mathcal{S}' \rightarrow \mathbb{R}$ is the reward function associated with each state transition.
\end{itemize}

Then, the agent's interaction with the environment is made through actions chosen from the action space $\mathcal{A}$ in response to observed states $s \in \mathcal{S}$,
according to a policy $\pi (s, a)$.
The agent's goal is to maximize cumulative rewards over time, which are obtained from the reward function $R$.
The return $G_t$ is used to estimate the value of a state-action pair by summing the rewards obtained from time $t$ to the end of the episode (usually, as well as in our case,
discounted by a factor $\gamma$ to favor immediate rewards), where $R_{t}$ is the observed reward at time $t$ and $T$ is the time of the episode's end:

\[
    G_t = \int_{t}^{T} \gamma^{k-t} R_{k} dk
\]

%\[
%    \pi^{*} = \arg \max_{\pi} \mathbb{E} \left[ G_t | s_t, \pi \right]
%\]

\subsubsection{Chosen State Space}
We choose a state space that tries to best incorporate the historical events of the limit order book into a single observable state using commonly used indicators and LOB levels,
as well as intrinsic features to the agent deemed relevant and chosen through our initial bibliography research, as well as ~\cite{Gasperov2021}.
Given our performed bibliographical research, we chose the agent's current inventory for the intrinsic feature and a set of indicators for the extrinsic features:
the Relative Strength Index (RSI); order imbalance (O); and micro price (MP).
Additionally, for a fixed number $D$ of LOB price levels the pair $(\delta^d, Q^d)$, where $\delta^d$ is the half-spread distance for the level $d \leq D$,
and $Q^d$ the amount of orders posted at that level is added to the state as a set of tuples, for both the ask and bid sides of the book.
The state space can therefore be represented by the following expression:

\[
    s_{t} \in \mathcal{S} = \left\{ \text{RSI}_t, \text{OI}_t, MP_{t}, \text{Inventory}_t, \left( \delta_t^{d, ask}, Q_t^{d, ask} \right)_{d=1}^{D}, \left( \delta_t^{d, bid}, Q_t^{d, bid} \right)_{d=1}^{D} \right\}
\]
where $0 < t < T$.

The indicators for our chosen market simulation framework are defined individually by values directly obtained from the observed LOB,
and serve as market state summaries for the agent to use:

\begin{itemize}
    \item \textbf{Order Imbalance (OI):} Order imbalance measures the relative difference between buy and sell orders at a given time.
    It is defined as:
    \[
        \text{OI}_t = \frac{Q_t^{\text{bid}} - Q_t^{\text{ask}}}{Q_t^{\text{bid}} + Q_t^{\text{ask}}},
    \]
    where \( Q_t^{\text{bid}} \) and \( Q_t^{\text{ask}} \) represent the total bid and ask quantities at time \( t \), respectively.
    \( \text{OI}_t \in [-1, 1] \), with \( \text{OI}_t = 1 \) indicating complete dominance of bid orders, and \( \text{OI}_t = -1 \) indicating ask order dominance.

    \item \textbf{Relative Strength Index (RSI):} The RSI is a momentum indicator that compares the magnitude of recent gains to recent losses to evaluate overbought or oversold conditions. It is given by:
    \[
        \text{RSI}_t = 100 - \frac{100}{1 + \frac{\text{Average Gain}}{\text{Average Loss}}},
    \]
    where the \textit{Average Gain} and \textit{Average Loss} are computed over a rolling window (in our case, fixed 5 minute observation intervals).
    Gains are the price increases during that window, while losses are the price decreases.

    \item \textbf{Micro Price (\( P_{\text{micro}} \)):} The micro price is a weighted average of the best bid and ask prices, weighted by their respective quantities:
    \[
        P_{\text{micro},t} = \frac{P_t^{\text{ask}} Q_t^{\text{bid}} + P_t^{\text{bid}} Q_t^{\text{ask}}}{Q_t^{\text{bid}} + Q_t^{\text{ask}}},
    \]
    where \( P_t^{\text{ask}} \) and \( P_t^{\text{bid}} \) represent the best ask and bid prices at time \( t \).

\end{itemize}

\subsubsection{Chosen Action Space}

The control, or agent, interacts with the environment choosing actions from the set of possible actions,
such that $a \in \mathcal{A}$ in response to observed states $s \in \mathcal{S}$ according to a policy $\pi (s, a)$ which we will define shortly,
and the end goal is to maximize cumulative rewards over time.
The agent's chosen action impacts the evolution of the system's dynamics by inserting orders into the LOB that might move the observed midprice,
to introduce features of market impact into our model.

The action space $\mathcal{A}$ includes the decisions made by the agent at time $t$, specifically the desired bid and ask spreads pair
$\delta^{\text{ask}}, \delta^{\text{bid}}$ and the corresponding posted order quantities $Q^{\text{ask}}, Q^{\text{bid}}$:
$$
\mathcal{A} = \left\{ (\delta^{\text{ask}}, \delta^{\text{bid}}, Q^{\text{ask}}, Q^{\text{bid}}), \forall \delta \in \mathbb{R}^+, \forall Q \in \mathbb{Z}\} \right.
$$

\subsubsection{Episodic Reward Function and Returns}

The episode reward function $R_t \in \mathbb{R}$ reflects the agent's profit and inventory risk obtained during a specific time in the episode.
It depends on the spread and executed quantities, as well as the inventory cost and was choosen according to commonly used reward structures taken from the literature review.

The overall objective is to maximize cumulative utility while minimizing risk associated with inventory positions,
and later insert restrictions so the risk for inventory is either limited at zero at market close, or incurring in larger penalties on the received rewards.
For our model the utility chosen is based on a running Profit and Loss (PnL) score while still managing inventory risk.
The choosen reward function is based on a risk-aversion enforced utility function, specifically the \textit{constant absolute risk aversion (CARA)}~\cite{Arrow1965, Pratt1964}
and depends on the realized spread $\delta$ and the realized quantity $q$\footnote{Differs from the agent's posted order quantity $Q$, as $q$ is a stochastic variable dependent on the underlying market dynamics.}.
The running PnL at time $t$ is computed as follows, where a penalty for holding large inventory positions is discounted from the \textit{PnL} score:

\begin{gather*}
    \text{Running PnL}_t = \delta_t^{\text{ask}} q_t^{\text{ask}} - \delta_t^{\text{bid}} q_t^{\text{bid}} + \text{I}_t \cdot \Delta M_t, \\
    \text{Penalty}_t = \eta \left( \text{Inventory}_t \cdot \Delta M_t \right)^+,\\
    \text{PnL}_t \coloneqq \text{Running PnL}_t - \text{Penalty}_t\\
\end{gather*}
where \( \eta \) is the penalty factor applied to positive inventory changes.

Finally, the reward function is defined as the negative of the exponential of the running PnL, a common choice for risk-averse utility functions according to literature~\cite{Gueant2022, Selser2021a, FalcesMarin2022}.
The chosen CARA utility function is defined as follows, where \( \gamma \) is the risk aversion parameter:
\[
    R_t = U(\text{PnL}_t) = -e^{-\gamma \cdot \text{PnL}_t},
\]

\subsection{State Transition Distribution}
\label{subsec:state-transition-distribution}

The previously mentioned transition probability density $P$ is given by a Stochastic Differential Equation expressed by the Kolmogorov forward equation for Markov Decission Processes:

\begin{equation}
    \label{eq:equation2}
    \frac{\partial P(s', t | s, a)}{\partial t}  = \int_{\mathcal{S}} \mathcal{L}(x | s, a, t) P(s'| x, a, t) dx
\end{equation}

for all $s, s' \in \mathcal{S}$ and all times $t$ before market close $T$, that is, $t \le T$,
where $a$ is choosen by our control agent according to a policy $\pi (s)$.
$\mathcal{L}$ is the generator operator and governs the dynamics of the state transitions given the current time.

In continuous-time and state MDPs, the state dynamics is reflected by $\mathcal{L}$ and modern approaches to optimal control
solve analytically by obtaining a closed-form expression for the model's evolution equations, as in~\citet{Avellaneda2008, Gueant2017}.
or numerically by approximating its transition probabilities, as in~\citet{Gueant2022, Selser2021a, FalcesMarin2022}.
Closed-form expressions for $\mathcal{L}$ are obtainable for simple models, by usually disconsidering market impact,
which is not the case for our proposed model, and solving for the generator operator is therefore outside the scope of this paper.
A numerical approach will be used furthermore as an approximation for the policy distribution according to observed environment trajectories.
We will use a neural network as the approximator function for the actor's policy through the Proximal Policy Optimization (PPO) algorithm
for the optimization of the policy distribution, as described in~\hyperref[sec:implementation-and-models]{Section~\ref*{sec:implementation-and-model-description}}.


\subsection{Market Model Description and Environment Dynamics}
\label{subsec:market-model-description-and-environment-dynamics}
Our approach to the problem of market making leverages \textbf{online reinforcement learning} by means of a simulator that models the dynamics of
a limit order book (LOB) according to a set of stylized facts observed in real markets.
For our model of the limit order book the timing of events follows a \textit{Hawkes process}
to represent a continuous-time MDP that captures the observed stylized fact of clustered order arrival times.
The Hawkes process is a \textit{self-exciting process}, where the intensity \( \lambda(t) \) depends on past events.
Formally, the intensity \( \lambda(t) \) evolves according to the following equation:
\begin{gather*}
    \lambda(t) = \mu + \sum_{t_i < t} \phi(t - t_i),\\
    \phi(t - t_i) = \alpha e^{-\beta (t - t_i)},\\
\end{gather*}
where \( \mu > 0 \) is the baseline intensity, and \( \phi(t - t_i) \) is the \textit{kernel function} that governs the decay of influence from past events \( t_i \).
A common choice for \( \phi \) is the exponential decay function, where \(\alpha\) controls the magnitude of the self-excitation and \(\beta\) controls the rate of decay.
\newline

The bid and ask prices for each new order are modeled by two separate \textit{Geometric Brownian Motion} processes to capture the normally distributed returns observed in real markets.
The underlying partial differential equation governing the ask and bid prices are given by:
\begin{gather*}
    dX_{t}^{ask} = (\mu_t + s_t) X_{t}^{mid} dt + \sigma dW_t,\\
    dX_{t}^{bid} = (\mu_t - s_t) X_{t}^{mid} dt + \sigma dW_t,
\end{gather*}

where $\mu$ is the price drift, $s_t$ is the mean spread, and $\sigma$ the price volatility.
The drift rate process follows a \textit{Ornstein-Uhlenbeck} process, which is a mean-reverting process,
while the spread rate similarly follows a \textit{Cox-Ingersoll-Ross} process, accounting for non-negative spreads.

Whenever a new limit order that narrows the bid-ask spread or a market order arrive the midprice is updated to reflect the top-of-book orders.
The midprice $X_{t}^{mid}$ is therefore obtained by averaging the current top-of-book bid and ask prices:

\[
    X_{t}^{mid} = \frac{X_{t}^{ask} + X_{t}^{bid}}{2}
\]
and while there are no orders on both sides of the book, the midprice is either the last traded price,
observed or an initial value set for the simulation, in the given order of priority.
By analyzing the midprice process, we can make sure the returns are distributed according to $\mathcal{N}\left(\mu_t, \frac{\sigma^2}{2}\right)$,
and are therefore normally distributed, assuring the model reflects a stylized fact of LOBs commonly observed in markets~\cite{Gueant2022}.

Finally, the order quantities $q_t^{\text{ask}}$ and $q_t^{\text{bid}}$ are modeled as Poisson random variables, where the arrival rate $\lambda_q$ is a constant parameter.
\[
    q_t^{\text{ask}}, q_t^{\text{bid}} \sim \text{Poisson}(\lambda_q),
\]

Our simulator was implemented using the Red-black tree structure for the limit order book, while new orders follow the event dynamics described
by the Geometric Brownian Motion and Poisson processes for order details.
The individual market regime variables, specifically the spread, order arrival density and return drift,
are sampled from the aforementioned set of pre-defined distributions, and inserted into the simulator at each event time step.

\subsection{Decision Process}
\label{subsec:decision-process}

In reinforcement learning, the Bellman equation is a fundamental recursive relationship that expresses the value of a
state in terms of the immediate reward and the expected value of subsequent states.
For a given policy $\pi$, the Bellman equation for the value function $V(s)$ with respect to the chosen reward function is:
\begin{equation*}
    \begin{aligned}
        V^\pi(s) &= \sum_{a \in \mathcal{A}} \pi(a \mid s) \sum_{s' \in \mathcal{S}} P(s'_{t+1} \mid s_{t}, a) \left[ R_t + \gamma V^\pi(s'_{t+1}) \right]\\
        V^\pi(s) &= \mathbb{E}_\pi \left[ R_t + \gamma V^\pi(s_{t+1}) \mid s_t = s \right]\\
    \end{aligned}
\end{equation*}

where $V(s)$ is the value function, representing the expected return (cumulative discounted rewards) starting from state $s$,
going to state $s'$ by taking action $a$ and continuing the episode, $R_t$ is the reward obtained from the action taken,
and $\gamma$ is the discount factor, used to weigh the value favorably towards immediate rather than future rewards.
The Bellman equation underpins the process of optimal policy derivation, where the goal is to find the optimal policy $\pi^*$,
that is, the control for our action space $\mathcal{A}$ that maximizes the expected return,
and can solved by maximizing for the value function:
\begin{equation*}
    \begin{aligned}
        V^*(s) &= \max_a \mathbb{E} \left[ R_t + \gamma V^*(s_{t+1}) \mid s_t = s, a_t = a \right]\\
        \pi^*(s) &= \arg \max_{a \in \mathcal{A}} \mathbb{E} \left[ R_t + \gamma V^*(s_{t+1}) \mid s_t = s, a_t = a \right]\\
    \end{aligned}
\end{equation*}
The Bellman equation is used to recursively define state values in terms of immediate rewards and expected value of subsequent states,
and for simpler environments, the equation can be solved directly by dynamic programming methods.
Such methods, like value iteration or policy iteration, although effective, become computationally expensive and sometimes unfeasible for large state spaces,
or require a model of the environment, which is not always available in practice.
Neural networks have been successfully used to approximate both the value function or the optimal actions directly,
demonstrating state-of-the-art results in various domains, including games, robotics, and finance
~\cite{He2023, Bakshaev2020, Patel2018, Ganesh2019, Sun2022, Gasperov2021a}.
Our simulation environment is complex enough that the state space is too large to store all possible state-action pairs,
and since we assume no knowledge of the underlying dynamics of the environment when training the agent, a model-free approach is necessary.

\subsubsection{Generalized Policy Iteration and Policy Gradient}
\label{subsubsec:gpi}
Generalized Policy Iteration algorithms are a family of algorithms that combines policy evaluation and policy improvement steps iteratively,
so that the policy is updated based on the value function, and the value function is updated based on the policy.
While simpler methods directly estimate state values or state-action values, such as Q-learning, more complex methods
such as Actor-Critic algorithms, estimate both the policy and the value function.
The critic estimates the value function $V^\pi$, or $A^{\pi} = Q^{\pi} - V^{\pi}$, if using the advantage function, while the actor learns the policy $\pi$.
We use a \textbf{policy gradient} approach, where the agent optimizes the policy directly by maximizing the expected return,
through consecutive gradient ascent steps on the policy parameters given episodes of experience.
For policy gradient methods, the following equations express the gradient ascent step for the
policy parameters $\theta$ and the value function parameters $\phi$, using a $\alpha$ learning rate and the gradient operator $\nabla$:
\begin{gather*}
    \nabla_{\theta} J(\theta) = \mathbb{E}_{\tau \sim \pi_{\theta}} \left[ \sum_{t=0}^{T} \nabla_{\theta} \log \pi_{\theta}(a_t \mid s_t) A^{\pi}(s_t, a_t) \right]\\
    \nabla_{\phi} J(\phi) = \mathbb{E}_{\tau \sim \pi_{\theta}} \left[ \sum_{t=0}^{T} \nabla_{\phi} \left( V^{\pi}(s_t) - R_t \right)^2 \right]\\
    \pi_{\theta} \leftarrow \pi_{\theta} + \alpha \nabla_{\theta} J(\theta), \phi \leftarrow \phi + \alpha \nabla_{\phi} J(\phi)
\end{gather*}
\begin{figure}
    \centering
    \includegraphics[width=.8\columnwidth]{images/gpi}
    \caption{Diagram of the Generalized Policy Iteration loop for the implemented agent.}
    \label{fig:gpi}
\end{figure}

In the context of \textbf{Proximal Policy Optimization (PPO)}
the model is trained to improve its policy and value function by maximizing the expected return,
according to a clipped surrogate objective function, which ensures that the policy does not deviate excessively from the previous policy,
promoting stability during training.
For \textbf{PPO}, the loss function used for the policy gradient ascent is defined as:
\begin{gather*}
    L^{\text{actor}}(\theta) = \mathbb{E}_t \left[ \min \left( r_t(\theta) \hat{A}_t, \text{clip}(r_t(\theta), 1 - \epsilon, 1 + \epsilon) \hat{A}_t \right) \right],\\
    \hat{A}_t = \sum_{l=0}^{\infty} (\gamma \lambda)^l \delta_{t+l}.
\end{gather*}
where \( r_t(\theta) \) is the probability ratio between the new and old policies and $\hat{A}_t$ is the advantage function.
The critic loss is the mean square error between the predicted value $V(s_t)$ and the actual return $R_t$, defined as:
\begin{gather*}
    L^{\text{critic}} = \mathbb{E}_t \left[ \left( V(s_t) - R_t \right)^2 \right],\\
\end{gather*}
Algorithmically, this is implemented by gathering trajectories from the environment and estimating the expression above using finite differences
as shown in \hyperref[alg:ppo]{Algorithm~\ref{alg:ppo}}.
\begin{algorithm}
    \begin{algorithmic}[1]
        \Require Policy $\pi_\theta$, value function $V_\phi$, trajectories $\tau$, epochs $K$, batch size $B$
        \State \textbf{Compute Advantage Estimation:}
        \State $\hat{A}_t = \sum_{l=0}^{\infty} (\gamma \lambda)^l \delta_{t+l}$
        \State \textbf{Compute Returns:} $R_t = \hat{A}_t + V(s_t)$
        \State \textbf{Construct dataset} $\mathcal{D} = (s_t, a_t, \hat{A}_t, R_t, \log \pi_\theta(a_t | s_t))$
        \For{epoch $k = 1$ to $K$}
            \For{minibatch $(s, a, \hat{A}, R, \log \pi_{\theta_{\text{old}}}(a | s))$ in $\mathcal{D}$}
                \State \textbf{Policy Update:}
                \State Compute probability ratio:
                \(
                r_t(\theta) = \frac{\pi_\theta(a | s)}{\pi_{\theta_{\text{old}}}(a | s)}
                \)
                \State Compute loss separately for actor and critic:
                \begin{equation*}
                    \begin{aligned}
                        L^{\text{actor}} &= \mathbb{E} \left[ \min(r_t(\theta) \hat{A}_t, \text{clip}(r_t(\theta), 1 - \epsilon, 1 + \epsilon) \hat{A}_t) \right]\\
                        L^{\text{critic}} &= \mathbb{E} \left[ (R_t - V(s_t))^2 \right]\\
                    \end{aligned}
                \end{equation*}
                \State Perform backpropagation and gradient update:
                \[
                    \theta \gets \theta + \alpha \nabla_\theta L^{\text{actor}}, \quad \phi \gets \phi + \alpha \nabla_\phi L^{\text{critic}}
                \]
            \EndFor
        \EndFor
    \end{algorithmic}
    \caption{Actor-Critic with PPO Updates}
    \label{alg:ppo}
\end{algorithm}

Each pass updates both the policy parameters $\theta$ and the value function parameters $\phi$
by applying the usual backward propagation pass for neural networks to minimize the loss function according to a chosen optimizer.
A more detailed explanation of the PPO algorithm can be found in the original paper by Schulman et al. (2017)~\cite{Schulman2017}.
Our implementation of the algorithm is a distributed-parallel version of the original version,
where multiple learners interact with the environment in parallel and share the experience to update the policy and value function.
Short of the V-trace algorithm, this approach is similar to the IMPALA algorithm~\cite{Espeholt2018}.
Besides the proposed approach for simulating dynamic limit order book environments,
we also implement a hybrid neural network for the agent's policy and value functions,
composed of a sequence of self-attention layers to capture the spatial dependencies between the different levels of the LOB,
concatenated with dense layers to process the market features, as discussed in~\autoref{sec:implementation-and-model-description}.

\subsubsection{Benchmark Closed-Form Expression for Simplified Model}
To ensure the agent's performance is able to capture the complexity of the environment, we use a benchmark closed-form expression
to compare the agent's performance against.
We chose the closed-form expression for the optimal bid-ask spread pair as proposed by Avellaneda et al. (2008)~\cite{Avellaneda2008},
which is given by the following expression:
\begin{equation}
    \begin{aligned}
        \delta^* &= \frac{\sigma}{\sqrt{2}} \text{erf}^{-1} \left( \frac{1}{2} \left( 1 + \frac{\mu}{\sigma} \right) \right),\\
        &\text{erf}(x) = \frac{2}{\sqrt{\pi}} \int_{0}^{x} e^{-t^2} dt.
    \end{aligned}
    \label{eq:avellaneda}
\end{equation}
where $\sigma$ is the volatility, $\mu$ is the mean spread, and $\text{erf}^{-1}$ is the inverse error function.
The error function $\text{erf}(x)$ serves simply as a measure of the spread of the normal distribution.

The closed-form expression is derived from a simple market model which assumes normally distributed non-mean reverting spreads,
constant executed order size and exponentially distributed event time dynamics.
As it depends on a simpler model for the market to be implemented, our expectations are that the agent will under-perform in a more complex environment.
The optimal bid-ask spread pair according to \autoref{eq:avellaneda} is then given by $p_\text{bid} = \mu - \delta^*$ and $p_\text{ask} = \mu + \delta^*$.
A fixed quantity of $1$ quoted share as originally proposed is used.



\newpage

\bibliographystyle{apalike}
\bibliography{reinforcement_learning.bib}

\end{document}