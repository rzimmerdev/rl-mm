\section{Methodology}
\label{sec:methodology}

\subsection{Problem Definition}
\label{subsec:problem-definition}
The market-making problem addressed in this work involves designing an optimal trading policy for an agent using reinforcement learning (RL).
The agent aims to maximize profit while managing risks, particularly inventory risk under the restriction of having zero inventory at market close,
and interacts with the environment by quoting bid and ask prices and adjusting offered quantities.
The environment dynamics are modeled by an underlying limit order book (LOB) and its behaviour will be described in the following sections.
As discussed previously, the main challenge for choosing an adequate agent and its policy lies in balancing profitability with risk management,
especially regarding inventory at the close of the market, where overnight positions can expose the agent to significant risks.

\subsection{A Formal Description of the Chosen RL Environment}
\label{subsec:formal-description-of-the-rl-environment}
In modeling the RL environment, we initially utilize a continuous-time, continuous-state Markov Chain framework,
and later transition to a discrete representation to address computational space constraints.
The specific case in which a Markov Chain also has an associated reward distribution $R$ for each state transition is called a Markov Reward Process
and given that the MM problem also has a decision process that affects the transition probabilities it is therefore called a Markov Decision Process (MDP) in control literature.
A Markov Decision Process can generically be defined as a 4-tuple $ (\mathcal{S}, \mathcal{A}, \mathbb{P}, R) $, where:

\begin{itemize}
    \item $\mathcal{S}$ is a set of states called the state space.
    \item $\mathcal{A}$ is a set of actions called the action space.
    \item $P: \mathcal{S} \times \mathcal{A} \times \mathcal{S} \to [0, 1]$ is the transition probability function for the MDP.
    \item $R: \mathcal{S} \times \mathcal{A} \times \mathcal{S}' \rightarrow \mathbb{R}$ is the reward function associated with each state transition.
\end{itemize}

Then, the agent interacts with the environment by choosing actions from the action space $\mathcal{A}$ in response to observed states $s \in \mathcal{S}$ according to a policy $\pi (s, a)$.
The agent's goal is to maximize cumulative rewards over time, which are obtained from the reward function $R$.
An adjusted version of the Bellman equation is used to derive the optimal policy for the agent, and the episode's return is used to evaluate the agent's performance.

\[
    \pi^{*} = \arg \max_{\pi} \mathbb{E} \left[ G_t | s_t, \pi \right]
\]

The return $G_t$ is defined as the sum of rewards obtained from time $t$ to the end of the episode (usually, as well as in our case, discounted by a factor $\gamma$ to favor immediate rewards):

\[
    G_t = \int_{t}^{T} \gamma^{k-t} R_{k} dk
\]
where $R_{t}$ is the observed reward at time $t$ and $T$ is the time of the episode's end.

\subsubsection{Chosen State Space}
We choose a state space that tries to best incorporate the historical events of the limit order book into a single observable state using commonly used indicators and LOB levels,
as well as intrinsic features to the agent deemed relevant and chosen through our initial bibliography research, as well as ~\cite{Gasperov2021}.
Given our performed bibliographical research, we chose the agent's current inventory for the intrinsic feature and a set of indicators for the extrinsic features:
the Relative Strength Index (RSI); order imbalance (O); and micro price (MP).
Additionally, for a fixed number $D$ of LOB price levels the pair $(\delta^d, Q^d)$, where $\delta^d$ is the half-spread distance for the level $d \leq D$,
and $Q^d$ the amount of orders posted at that level is added to the state as a set of tuples, for both the ask and bid sides of the book.
The state space can therefore be represented by the following expression:

\[
    s_{t} \in \mathcal{S} = \left\{ \text{RSI}_t, \text{OI}_t, MP_{t}, \text{Inventory}_t, \left( \delta_t^{d, ask}, Q_t^{d, ask} \right)_{d=1}^{D}, \left( \delta_t^{d, bid}, Q_t^{d, bid} \right)_{d=1}^{D} \right\}
\]
where $0 < t < T$.

The indicators for our chosen market simulation framework are defined individually by values directly obtained from the observed LOB,
and serve as market state summaries for the agent to use:

\begin{itemize}
    \item \textbf{Order Imbalance (OI):} Order imbalance measures the relative difference between buy and sell orders at a given time.
    It is defined as:
    \[
        \text{OI}_t = \frac{Q_t^{\text{bid}} - Q_t^{\text{ask}}}{Q_t^{\text{bid}} + Q_t^{\text{ask}}},
    \]
    where \( Q_t^{\text{bid}} \) and \( Q_t^{\text{ask}} \) represent the total bid and ask quantities at time \( t \), respectively.
    \( \text{OI}_t \in [-1, 1] \), with \( \text{OI}_t = 1 \) indicating complete dominance of bid orders, and \( \text{OI}_t = -1 \) indicating ask order dominance.

    \item \textbf{Relative Strength Index (RSI):} The RSI is a momentum indicator that compares the magnitude of recent gains to recent losses to evaluate overbought or oversold conditions. It is given by:
    \[
        \text{RSI}_t = 100 - \frac{100}{1 + \frac{\text{Average Gain}}{\text{Average Loss}}},
    \]
    where the \textit{Average Gain} and \textit{Average Loss} are computed over a rolling window (in our case, fixed 5 minute observation intervals).
    Gains are the price increases during that window, while losses are the price decreases.

    \item \textbf{Micro Price (\( P_{\text{micro}} \)):} The micro price is a weighted average of the best bid and ask prices, weighted by their respective quantities:
    \[
        P_{\text{micro},t} = \frac{P_t^{\text{ask}} Q_t^{\text{bid}} + P_t^{\text{bid}} Q_t^{\text{ask}}}{Q_t^{\text{bid}} + Q_t^{\text{ask}}},
    \]
    where \( P_t^{\text{ask}} \) and \( P_t^{\text{bid}} \) represent the best ask and bid prices at time \( t \).

\end{itemize}

\subsubsection{Chosen Action Space}

The control, or agent, interacts with the environment choosing actions from the set of possible actions,
such that $a \in \mathcal{A}$ in response to observed states $s \in \mathcal{S}$ according to a policy $\pi (s, a)$ which we will define shortly,
and the end goal is to maximize cumulative rewards over time.
The agent's chosen action impacts the evolution of the system's dynamics by inserting orders into the LOB that might move the observed midprice,
to introduce features of market impact into our model.

The action space $\mathcal{A}$ includes the decisions made by the agent at time $t$, specifically the desired bid and ask spreads pair
$\delta^{\text{ask}}, \delta^{\text{bid}}$ and the corresponding posted order quantities $Q^{\text{ask}}, Q^{\text{bid}}$:
$$
\mathcal{A} = \left\{ (\delta^{\text{ask}}, \delta^{\text{bid}}, Q^{\text{ask}}, Q^{\text{bid}}), \forall \delta \in \mathbb{R}^+, \forall Q \in \mathbb{Z}\} \right.
$$

\subsubsection{Episodic Reward Function and Returns}

The episode reward function $R_t \in \mathbb{R}$ reflects the agent's profit and inventory risk obtained during a specific time in the episode.
It depends on the spread and executed quantities, as well as the inventory cost and was choosen according to commonly used reward structures taken from the literature review.

The overall objective is to maximize cumulative utility while minimizing risk associated with inventory positions,
and later insert restrictions so the risk for inventory is either limited at zero at market close, or incurring in larger penalties on the received rewards.
For our model the utility chosen is based on a running Profit and Loss (PnL) score while still managing inventory risk.
The choosen reward function is based on a risk-aversion enforced utility function, specifically the \textit{constant absolute risk aversion (CARA)}~\cite{Arrow1965, Pratt1964}
and depends on the realized realized spread $\delta$ and the realized quantity $q$ (not to confuse with the agent's posted order quantity $Q$).

The running PnL at time $t$ is computed as follows, and a penalty for holding large inventory positions is discounted from the \textit{PnL} score, as follows:
\begin{gather*}
    \text{Running PnL}_t = \delta_t^{\text{ask}} q_t^{\text{ask}} - \delta_t^{\text{bid}} q_t^{\text{bid}} + \text{I}_t \cdot \Delta M_t, \\
    \text{Penalty}_t = \eta \left( \text{Inventory}_t \cdot \Delta M_t \right)^+,\\
    \text{PnL}_t \coloneqq \text{Running PnL}_t - \text{Penalty}_t\\
\end{gather*}
where \( \eta \) is the penalty factor applied to positive inventory changes.

Finally, the reward function is defined as the negative of the exponential of the running PnL, which is a common choice for risk-averse utility functions:
The constant absolute risk aversion (CARA) utility function is defined as follows, where \( \gamma \) is the risk aversion parameter:
\[
    R_t = U(\text{PnL}_t) = -e^{-\gamma \cdot \text{PnL}_t},
\]

\subsection{State Transition Distribution}
\label{subsec:state-transition-distribution}

The previously mentioned transition probability density $P$ is given by a Stochastic Differential Equation expressed by the Kolmogorov forward equation for Markov Decission Processes:

\begin{equation}
    \label{eq:equation2}
    \frac{\partial P(s', t | s, a)}{\partial t}  = \int_{\mathcal{S}} \mathcal{L}(x | s, a, t) P(s'| x, a, t) dx
\end{equation}

for all $s, s' \in \mathcal{S}$ and all times $t$ before market close $T$, that is, $t \le T$,
where $a$ is choosen by our control agent according to a policy $\pi (s)$.
$\mathcal{L}$ is the generator operator and governs the dynamics of the state transitions given the current time.

In continuous-time and state MDPs, the state dynamics is reflected by $\mathcal{L}$ and modern approaches to optimal control
solve analytically by obtaining a closed-form expression for the model's evolution equations, as in~\citet{Avellaneda2008, Gueant2017}.
or numerically by approximating its transition probabilities, as in~\citet{Gueant2022, Selser2021a, FalcesMarin2022}.
Closed-form expression for $\mathcal{L}$ are obtainable for simple models, that usually require that market order impact be not disconsidered,
which is not the case for our proposed model, and solving for the generator operator is therefore outside the scope of this paper.
A numerical approach will be used furthermore when we define a neural network approximator for the actor's policy using the
Proximal Policy Optimization (PPO) algorithm for model weight optimization in~\hyperref[sec:implementation-and-models]{Section~\ref*{sec:implementation-and-model-description}}.
as well as both restriction formulations.

\subsection{Market Model Description and Environment Dynamics}
\label{subsec:market-model-description-and-environment-dynamics}

For our model of the limit order book the timing of events follows a \textit{Hawkes process}
to represent a continuous-time MDP that captures the observed stylized fact of clustered order arrival times.

The Hawkes process is a \textit{self-exciting process}, where the intensity \( \lambda(t) \) depends on past events.
Formally, the intensity \( \lambda(t) \) evolves according to the following equation:
\[
    \lambda(t) = \mu + \sum_{t_i < t} \phi(t - t_i),
\]
where \( \mu > 0 \) is the baseline intensity, and \( \phi(t - t_i) \) is the \textit{kernel function} that governs the decay of influence from past events \( t_i \).
A common choice for \( \phi \) is an exponential decay:
\[
    \phi(t - t_i) = \alpha e^{-\beta (t - t_i)},
\]
where \(\alpha\) controls the magnitude of the self-excitation and \(\beta\) controls the rate of decay.

The bid and ask prices for each new order are modeled by two separate \textit{Ornstein-Uhlenbeck (OU) processes} to capture the mean-reversion behavior of spreads over the midprice:
\[
    ds_t = \theta(\mu - s_t) dt + \sigma dW_t,
\]
where $s_t$ is the market spread at time $t$, $\theta$ is the rate of mean reversion, $\mu$ is the long-term spread mean and $\sigma$ its volatility.
The Wiener process $W_t$ is used to represent random market fluctuations.
Both $\mu$ and $\sigma$ will be estimated using historical datasets of market LOBs in Section 5 <experimentos> to replicate real observed market conditions.

The bid and ask spreads $\delta_t^{\text{bid}}$ and $ \delta_t^{\text{ask}}$ for orders conditioned on their arrival follow normal distributions:
\[
    \delta_{t+1}^{\text{ask}} \sim \mathcal{N}(\mu + M_{t} + s_t, \sigma^2), \quad a_{t+1}^{\text{bid}} \sim \mathcal{N}(\mu + M_{t} - s_t, \sigma^2),
\]

Whenever a new limit order that narrows the bid-ask spread or a market order arrive the midprice is updated to reflect the top-of-book orders.
The midprice $M_{t+1}$ is therefore obtained by averaging the current top-of-book bid and ask prices:

\[
    M_{t+1} \coloneqq \frac{2M_t + \delta^{ask}_{t} - \delta^{bid}_{t}}{2}
    \sim \mathcal{N}\left(\mu + M_{t}, \frac{\sigma^2}{2}\right)
\]
and at $t = 0$, the midprice is defined according to some starting point of the simulation, usually set to observed historical prices at market open.

We can obtain the distribution for the midprice by analyzing the dynamics of the top of book bid and ask prices, resulting in the following \textit{Brownian Motion} process:
\[
    dM_t = \mu dt + \frac{\sigma^2}{2} dW_t,
\]
where \( W_t \) is a Wiener process, and therefore means our choosen midprice process reflects a stylized fact of LOBs commonly observed in markets <inserir referencia book HFT>, that is, having normally distributed returns.

Finally, the order quantities \( q_t^{\text{ask}} \) and \( q_t^{\text{bid}} \) are modeled as Poisson random variables:
\[
    q_t^{\text{ask}}, q_t^{\text{bid}} \sim \text{Poisson}(\lambda_q),
\]
where \( \lambda_q \) is the average order size.

\subsection{Decision Process and Steps to Maximize the Agent's Objective}
\label{subsec:decision-process-and-steps-to-maximize-the-agent's-objective}

In reinforcement learning, the Bellman equation is a fundamental recursive relationship that expresses the value of a state in terms of the immediate reward and the expected value of subsequent states. For a given policy $\pi$, the Bellman equation for the value function $V(s)$ with respect to the chosen reward function is:

\[
    V(s) = \mathbb{E}_{\pi} \left[ R(s', s, a) + \gamma V(s') \right],
\]

where $V(s)$ is the value function, representing the expected return (cumulative discounted rewards) starting from state $s$, going to state $s'$ by taking action $a$ and continuing the episode, $R(s', s, a)$ is the reward obtained from taking action $a$ in state $s$ and ending in state $s'$,
and $\gamma$ is the discount factor, which weighs the value favorably towards immediate in comparison to future rewards.

The Bellman equation underpins the process of optimal policy derivation, where the goal is to find the policy $\pi^*$, that is, the control for our action space $\mathcal{A}$ that maximizes the expected return, $V^*(s)$. The optimal value function satisfies the equation:

\[
    V^*(s) = \max_{\pi} \mathbb{E}_{\pi} \left[ R(s, a) + \gamma V^*(s') \right].
\]

Thus, the RL approach is a way of obtaining the optimal control $\pi^*$ given an agent's \textbf{objective} that maximizes the expected return for all states. For our approach, a numerical approximation of the policy is taken due to computational space limitations, the use of neural networks for the underlying approximator is a viable approach and the current State-of-the-art, as proposed by <ref>. The design of a \textbf{loss function}, a neural network architecture and choice of optimizer will be discussed in the following section shortly after a small overview of the considered algorithms.

\subsubsection{Deep Reinforcement Learning and Chosen Loss Function}

In the context of deep reinforcement learning the loss function is a derivation from the usual return function \( G(\tau) \).
In the context of \textbf{Proximal Policy Optimization (PPO)} and Actor Critic algorithms overall,
this loss function can be interpreted as the negative of the expected return, which we aim to minimize during training.
The model is trained to improve its policy such that the chosen actions maximize the cumulative rewards over time,
directly aligning with the Bellman equation's goal of maximizing the value function.

For \textbf{PPO}, the loss function is:

\[
    L^{\text{PPO}}(\theta) = \mathbb{E}_t \left[ \min \left( r_t(\theta) \hat{A}_t, \text{clip}(r_t(\theta), 1 - \epsilon, 1 + \epsilon) \hat{A}_t \right) \right],
\]

where \( r_t(\theta) \) is the probability ratio between the new and old policies, \( \hat{A}_t \) is the advantage function,
and the clipped term ensures that updates to the policy do not deviate excessively from the previous policy, promoting stability during training.
The critic loss is defined as:

\[
    L_{\text{critic}} = \mathbb{E}_t \left[ \left( R_t - V(s_t) \right)^2 \right],
\]

which minimizes the error between the predicted value \( V(s_t) \) and the actual return \( R_t \).
Both losses are minimized simultaneously during training to improve the policy and the value function,
through multiple sequential backward passes over batches of gathered episodes.
Each pass updates the model's weights by applying the usual backward propagation pass for neural networks to minimize the loss function according to the Adam optimizer.

\subsubsection{Benchmark Closed-Form Expression for Simplified Model}

To guide the agent's learning, we incorporate a closed-form expression for the best bid-ask pair prices derived from a simplified version of the environment.
This analytical solution offers an initial policy for the agent to start from,
specifically by assuming normally distributed non-mean reverting spreads and constant order size and exponentially distributed order flow dynamics.
This policy serves both as a starting for training and a benchmark for evaluating the agent's learned performance.
As it depends on a simpler model for the market to be implemented, our expectations are for it to beat the deep RL models for the simpler market model,
and by definition be unusable for the more complex one.

<citar artigo que usa modelo simplificado de mercado vs complexo sem versao analitica>
