\subsection{Market Model Description and Environment Dynamics}
\label{subsec:market-model-description-and-environment-dynamics}
Our approach to the problem of market making leverages \textbf{online reinforcement learning} by means of a simulator that models the dynamics of
a limit order book (LOB) according to a set of stylized facts observed in real markets.
For our model of the limit order book the timing of events follows a \textit{Hawkes process}
to represent a continuous-time MDP that captures the observed stylized fact of clustered order arrival times.
The Hawkes process is a \textit{self-exciting process}, where the intensity \( \lambda(t) \) depends on past events.
Formally, the intensity \( \lambda(t) \) evolves according to the following equation:
\begin{gather*}
    \lambda(t) = \mu + \sum_{t_i < t} \phi(t - t_i),\\
    \phi(t - t_i) = \alpha e^{-\beta (t - t_i)},\\
\end{gather*}
where \( \mu > 0 \) is the baseline intensity, and \( \phi(t - t_i) \) is the \textit{kernel function} that governs the decay of influence from past events \( t_i \).
A common choice for \( \phi \) is the exponential decay function, where \(\alpha\) controls the magnitude of the self-excitation and \(\beta\) controls the rate of decay.
\newline

The bid and ask prices for each new order are modeled by two separate \textit{Geometric Brownian Motion} processes to capture the normally distributed returns observed in real markets.
The underlying partial differential equation governing the ask and bid prices are given by:
\begin{gather*}
    dX_{t}^{ask} = (\text{rf} + s_t) X_{t}^{mid} dt + \sigma dW_t,\\
    dX_{t}^{bid} = (\text{rf} - s_t) X_{t}^{mid} dt + \sigma dW_t,
\end{gather*}

where $s_t$ is the mean spread, $\text{rf}$ is the risk-free rate, and $\sigma$ is the volatility.
The spread process follows a \textit{Ornstein-Uhlenbeck} process, which is a mean-reverting process that models the spread's tendency to return to a long-term average,
while the risk-free rate follows a \textit{Cox-Ingersoll-Ross} process, which is a mean-reverting process that models the risk-free rate's tendency to return to a long-term average.

Whenever a new limit order that narrows the bid-ask spread or a market order arrive the midprice is updated to reflect the top-of-book orders.
The midprice $X_{t}^{mid}$ is therefore obtained by averaging the current top-of-book bid and ask prices:

\[
    X_{t}^{mid} = \frac{X_{t}^{ask} + X_{t}^{bid}}{2}
\]
and while there are no orders on both sides of the book, the midprice is either the last traded price,
observed or an initial value set for the simulation, in the given order of priority.
By analyzing the midprice process, we can make sure the returns are distributed according to $\mathcal{N}\left(\text{rf}_t, \frac{\sigma^2}{2}\right)$,
and are therefore normally distributed, assuring the model reflects a stylized fact of LOBs commonly observed in markets~\cite{Gueant2022}.

Finally, the order quantities $q_t^{\text{ask}}$ and $q_t^{\text{bid}}$ are modeled as Poisson random variables, where the arrival rate $\lambda_q$ is a constant parameter.
\[
    q_t^{\text{ask}}, q_t^{\text{bid}} \sim \text{Poisson}(\lambda_q),
\]
