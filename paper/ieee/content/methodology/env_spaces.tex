\subsection{A Formal Description of the Environment}
\label{subsec:formal-description-of-the-rl-environment}
In modeling the environment, we initially utilize a continuous-time, continuous-state Markov Chain framework,
and later transition to a discrete implementation to address computational space constraints.
The specific case in which a Markov Chain also has an associated reward distribution $R$ for each state transition is called a Markov Reward Process
and given that the agent's decisions also affect the transition probabilities due to market impact it is therefore called a Markov Decision Process (MDP) in control literature.
A Markov Decision Process can generically be defined as a 4-tuple $ (\mathcal{S}, \mathcal{A}, \mathbb{P}, R) $, where:

\begin{itemize}
    \item $\mathcal{S}$ is a set of states called the state space.
    \item $\mathcal{A}$ is a set of actions called the action space.
    \item $\mathbb{P}: \mathcal{S} \times \mathcal{A} \times \mathcal{S} \to [0, 1]$ is the transition probability function for the MDP.
    \item $R: \mathcal{S} \times \mathcal{A} \times \mathcal{S}' \rightarrow \mathbb{R}$ is the reward function associated with each state transition.
\end{itemize}

Then, the agent's interaction with the environment is made through actions chosen from the action space $\mathcal{A}$ in response to observed states $s \in \mathcal{S}$,
according to a policy $\pi (s, a)$.
The agent's goal is to maximize cumulative rewards over time, which are obtained from the reward function $R$.
The return $G_t = \int_{t}^{T} \gamma^{k-t} R_{k} dk$ is used to estimate the value of a state-action pair by summing the rewards obtained from time $t$ to the end of the episode
(usually, as well as in our case, discounted by a factor $\gamma$ to favor immediate rewards),
where $R_{t}$ is the observed reward at time $t$ and $T$ is the time of the episode's end.

\subsubsection{Chosen State Space}
We choose a state space that tries to best incorporate the historical events of the limit order book into a single observable state using
market features deemed relevant and chosen through our initial bibliography research,
as well as guided by the works of
Gasparov et al. (2021)~\cite{Gasperov2021} and Gueant et al. (2017)~\cite{Gueant2017} on reinforcement learning for market-making.
We used the agent's current inventory and a set of indicators for the currently observed market state:
the Relative Strength Index (RSI); order imbalance (OI); and micro price (MP).
Additionally, for a fixed number $D$ of LOB price levels the pair $(\delta^d, Q^d)$, where $\delta^d$ is the half-spread distance for the level $d \leq D$,
and $Q^d$ the amount of orders posted at that level is added to the state as a set of tuples, for both the ask and bid sides of the book.
The state space can therefore be represented by the following expression:
\begin{equation*}
    \begin{aligned}
        s_{t} \in \mathcal{S} = \big\{ &\text{RSI}_t, \text{OI}_t, MP_{t}, \text{Inventory}_t, \\
        & \text{MA}_{10, t}, \text{MA}{15, t}, \text{MA}{30, t}, \\
        & (\delta_t^{d, ask}, Q_t^{d, ask})_{d=1}^{D}, \\
        & (\delta_t^{d, bid}, Q_t^{d, bid})_{d=1}^{D} \big\}
    \end{aligned}\label{eq:equation}
\end{equation*}
where $0 < t < T$, and $\text{MA}_{n, t}$ is the moving average of the price returns over the last $n$ time steps.
The indicators for our chosen market simulation framework are defined individually by values directly obtained from the observed LOB,
and serve as market state summaries for the agent to use:

\begin{itemize}
    \item \textbf{Order Imbalance (OI):} Measures the relative difference between buy and sell orders at a given time.
    \item \textbf{Relative Strength Index (RSI):} Momentum indicator that compares the magnitude of recent gains to recent losses to evaluate overbought or oversold conditions.
    \item \textbf{Micro Price (\( P_{\text{micro}} \)):} Weighted average of the best bid and ask prices, weighted by their respective quantities.
\end{itemize}

\begin{equation}
    \begin{aligned}
        \text{OI}_t &= \frac{Q_t^{\text{bid}} - Q_t^{\text{ask}}}{Q_t^{\text{bid}} + Q_t^{\text{ask}}}\\
        \text{RSI}_t &= 100 - \frac{100}{1 + \frac{\text{Average Gain}}{\text{Average Loss}}}\\
        \text{P}_{\text{micro}, t} &= \frac{P_t^{\text{ask}} Q_t^{\text{bid}} + P_t^{\text{bid}} Q_t^{\text{ask}}}{Q_t^{\text{bid}} + Q_t^{\text{ask}}}
    \end{aligned}
    \label{eq:features}
\end{equation}

    where \( P_t^{\text{ask}} \) and \( P_t^{\text{bid}} \) represent the best ask and bid prices at time \( t \), and
    \( Q_t^{\text{bid}} \) and \( Q_t^{\text{ask}} \) represent the total bid and ask quantities, respectively.
    The \textit{Average Gain} and \textit{Average Loss} are computed over a rolling window (in our case, fixed 5 minute observation intervals),
    and are defined as the average price increases and decreases during that window, respectively.
    \( \text{OI}_t \in [-1, 1] \), with \( \text{OI}_t = 1 \) indicating complete dominance of bid orders, and \( \text{OI}_t = -1 \) indicating ask order dominance.
    The RSI is bounded between 0 and 100, with values above 70 indicating overbought conditions and values below 30 indicating oversold conditions.

\subsubsection{Chosen Action Space}

The control, or agent, interacts with the environment choosing actions from the set of possible actions,
such that $a \in \mathcal{A}$ in response to observed states $s \in \mathcal{S}$ according to a policy $\pi (s, a)$ which we will define shortly,
and the end goal is to maximize cumulative rewards over time.
The agent's chosen action impacts the evolution of the system's dynamics by inserting orders into the LOB that might move the observed midprice,
to introduce features of market impact into our model.

The action space $\mathcal{A}$ includes the decisions made by the agent at time $t$, specifically the desired bid and ask spreads pair
$\delta^{\text{ask}}, \delta^{\text{bid}}$ and the corresponding posted order quantities $Q^{\text{ask}}, Q^{\text{bid}}$:
$$
\mathcal{A} = \left\{ (\delta^{\text{ask}}, \delta^{\text{bid}}, Q^{\text{ask}}, Q^{\text{bid}}), \forall \delta \in \mathbb{R}^+, \forall Q \in \mathbb{Z}\} \right.
$$

\subsubsection{Episodic Reward Function and Returns}

The episode reward function $R_t \in \mathbb{R}$ reflects the agent's profit and inventory risk obtained during a specific time in the episode.
It depends on the spread and executed quantities, as well as the inventory cost and was choosen according to the best performing reward structures taken from the literature review.
For our model the utility chosen is based on a risk-aversion enforced utility function, specifically the \textit{constant absolute risk aversion (CARA)}~\cite{Arrow1965, Pratt1964}
with the virtual running Profit and Loss (PnL) as input.
It depends on the realized spread $\delta$ and the realized quantity $q$\footnote{Differs from the agent's posted order quantity $Q$, as $q$ is a stochastic variable dependent on the underlying market dynamics.},
and is computed as follows, where the penalty for holding large inventory positions is discounted by a factor of \( \eta \) from the final score:
\begin{gather*}
    \text{Running PnL}_t = \delta_t^{\text{ask}} q_t^{\text{ask}} - \delta_t^{\text{bid}} q_t^{\text{bid}} + \text{I}_t \cdot \Delta M_t, \\
    \text{Penalty}_t = \eta \left( \text{Inventory}_t \cdot \Delta M_t \right)^+,\\
    \text{PnL}_t \coloneqq \text{Running PnL}_t - \text{Penalty}_t\\
\end{gather*}
Finally, the reward function is defined as the negative of the exponential of the running PnL, a common choice for risk-averse utility functions according to literature~\cite{Gueant2022, Selser2021a, FalcesMarin2022}.
The CARA utility function is defined as follows, where \( \gamma \) is the risk aversion parameter:
\[
    R_t = U(\text{PnL}_t) = -e^{-\gamma \cdot \text{PnL}_t},
\]

\subsection{State Transition Distribution}
\label{subsec:state-transition-distribution}

The previously mentioned transition probability density $P$ is given by a Stochastic Differential Equation expressed by the Kolmogorov
forward equation for Markov Decission Processes:

\begin{equation}
    \label{eq:equation2}
    \frac{\partial P(s', t | s, a)}{\partial t}  = \int_{\mathcal{S}} \mathcal{L}(x | s, a, t) P(s'| x, a, t) dx
\end{equation}

for all $s, s' \in \mathcal{S}$ and all times $t$ before market close $T$, that is, $t \le T$,
where $a$ is choosen by our control agent according to a policy $\pi (s)$.
$\mathcal{L}$ is the generator operator and governs the dynamics of the state transitions given the current time.

In continuous-time and state MDPs, the state dynamics is reflected by $\mathcal{L}$, and modern approaches to optimal control
solve analytically by obtaining a closed-form expression for the model's evolution equations, as originally tackled by
Avellaneda et al. (2008)~\cite{Avellaneda2008} and Gueant et al. (2017)~\cite{Gueant2017}.
or numerically by approximating its transition probabilities~\cite{Gueant2022, Selser2021a, FalcesMarin2022}.
Closed-form expressions for $\mathcal{L}$ are obtainable for simple environment models, by usually disconsidering market impact,
which is not the case for our proposed model, and solving for the generator operator is therefore outside the scope of this paper.
We describe in~\autoref{sec:implementation-and-model-description} our chosen approach to model the state transition distribution,
using the Proximal Policy Optimization (PPO) algorithm to approximate the optimal policy for the agent.
