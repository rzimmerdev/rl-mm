\section{Realized Experiments and Results}
\label{sec:realized-experiments-and-results}

\subsection{Experiment Setup}
\label{subsec:experiment-setup}

A max episode value of 10,000 episodes was used, with each episode consisting on average of 390 observations (or 1 event per corresponding market minute).
Per gathered trajectory, 10 epochs were used for the policy improvement step.
For training, we used a single NVIDIA GeForce RTX 3090 GPU with 24GB of memory, and an AMD Ryzen 9 5950X CPU with 16 cores and 32 threads,
and 32GB of DDR4 RAM.

% hyperparam optimization
% --gamma=0.9 --epsilon=0.25 --lambd=0.85 --entropy=0.0012 --lr_policy=3e-4 --lr_value=3e-4  --batch_size=256
We used the Adam optimizer with a learning rate of $3 \times 10^{-4}$ for both the policy and value networks.
The discount factor $\gamma$ was set to 0.9, the GAE parameter $\lambda$ was set to 0.85, and the PPO clipping parameter $\epsilon$ was set to 0.25.
The entropy coefficient was set to $1.2\times10^{-3}$, and the batch size set to 256 samples per episode/update.
We optimized the hyperparameters using a simple grid search approach.

For our market model, we chose the following parameters for each process:
\begin{itemize}
    \item The order arrival rate was set to $\lambda = 1$, with clustering parameters $\alpha = 0.1$ and $\beta = 0.1$.
    \item Mean spread was set to $s = 0.1$ (10 market price ticks, in our case, 10 cents) and annualized price drift to $\mu = 0.02$ ($2\%$), respectively.
    \item The price volatility parameters were set to $\omega = 0.5$, $\alpha = 0.1$, and $\beta = 0.1$,
    where $\omega$ is the constant term, $\alpha$ the autoregressive term, and $\beta$ the moving average term.
    \item The initial midprice was arbitrarily set to $100$.
\end{itemize}

\subsection{Experiment Results}
\label{subsec:experiment-results}

% Graphs:
% average financial return + confidence interval (+- volatility) x episode number
% average financial return (+- volatility) x current timestep (per 100x trajectories after training)
% average inventory x current timestep (per 100x trajectories after training)
% average reward moving average x episode number

% Table:
% Cols: rl-agent, benchmark agent
% Training
% Rows: Training time
%       3:08:54, , -
%       Time per episode +- std
%       0.8458 \pm 0.1044
%       Mean processing time actor +- std per episode
%       Mean processing time critic +- std per episode
%       Mean financial return +- std

% Mean PnL: -1.174902750662374e-05, Std PnL: 4.761125180139532e-05, Sortino Ratio: -0.5046819237780464
% Mean Stoikov PnL: -0.000423402308920138, Std Stoikov PnL: 0.0010584297592714442, Sortino Ratio: -0.6399344498955829

% Test (after last episode or convergence)
% Rows: Mean financial return +- std
%       Mean Sharpe ratio +- std
%       Agent action latency +- std
%       Mean inventory at market close

To evaluate the financial performance of the trained reinforcement learning agent, we analyzed the agent's
financial return, return volatility, and the Sortino ratio.
The results were averaged over $10^2$ trajectories using the same simulator hyperparameters used for training.

As shown in Table~\ref{tab:test-results}, the reinforcement learning agent exhibited a mean financial return of $-1.174 \times 10^{-5}$
(annualized return of about $-0.3\%$.), an almost neutral performance under adverse market conditions,
while the benchmark agent had a mean financial return of $-0.0004$ (annualized return of about $-10.5\%$),
a clear underperformance under the same conditions, but closely matching the expected performance.
The return volatility for the RL-agent was also lower, with an annualized volatility of about $1.2\%$,
compared to $25.2\%$ for the benchmark agent, indicating more stable financial returns.
Both Sortino ratios showed a negative value, indicating a higher risk of negative returns compared to the risk-free rate,
but the RL-agent ratio at $-0.5046$ still outperformed the benchmark's ratio of $-0.6399$.
Observing the reward curve during training, we can see a steady increase in the reward over time, as shown in Figure~\ref{fig:average-reward-moving-average},
indicating that the agent was learning to maximize its cumulative rewards over time, as expected.
As necessary for real-world applications, the agent's processing time was also measured, with the actor network taking on average $0.0029$ seconds per episode,
and the critic network taking $2.1 \times 10^{-4}$ seconds per episode, well within the acceptable range for medium-frequency trading.
Further training could be performed if necessary to improve the agent's performance, but the results obtained were already considered satisfactory
considering the fabricated adverse market conditions.

\begin{table}
    \centering
    \centering
    \small
    \begin{tabular}{|c|c|c|}
        \hline
        \textbf{Training}      & \textbf{Metric}                                     \\
        \hline
        Training Time          & $3\text{h}08\text{m}54\text{s}$                     \\
        Time per Episode       & $0.8458 \pm 0.1044$ \text{ (s)}                     \\
        Processing Time Actor  & $0.0029 \pm 0.001 \text{ (s)}$                      \\
        Processing Time Critic & $2.1\times10^{-4} \pm 3 \times 10^{-5} \text{ (s)}$ \\
        \hline
    \end{tabular}
    \caption{Test Results}
    \label{tab:test-results}
    \centering
    \vspace{0.5cm}
    \small
    \begin{tabular}{|c|c|c|}
        \hline
        \textbf{Test}     & \textbf{RL-Agent}       & \textbf{Benchmark} \\
        \hline
        Financial Return  & $-1.174 \times 10^{-5}$ & $-0.0004$          \\
        Return Volatility & $4.7611 \times 10^{-5}$ & $0.0010$           \\
        Sortino Ratio     & $-0.5046$               & $-0.6399$          \\
        \hline
    \end{tabular}
    \caption{Training Results}
    \label{tab:training-results}
\end{table}

% reward.png and returns.png

Overall, the training process maintained a steadily increasing reward, as shown in Figure~\ref{fig:average-reward-moving-average},
with the \textit{max\_iter} of $10^{4}$ episodes being reached in about 3 hours and 8 minutes.
The financial return of the RL-agent, as shown in Figure~\ref{fig:average-financial-return}, was stable around zero,
and even though we expected a slightly negative return due to the highly volatile market conditions and
somewhat low positive drift, the close to neutral performance was still considered a positive outcome,
especially when compared to the benchmark agent's performance.

\begin{figure}
    \centering
    \begin{minipage}{\columnwidth}
        \centering
        \includegraphics[width=1\textwidth]{images/reward}
        \caption{Exponential moving average of the training reward per episode, with a linear trend line.}
        \label{fig:average-reward-moving-average}
    \end{minipage}
    \vspace{0.04\textwidth} % Adjust horizontal space between figures
    \begin{minipage}{\columnwidth}
        \centering
        \includegraphics[width=1\textwidth]{images/returns}
        \caption{Financial return, averaged over 100 trajectories with a 1 standard deviation confidence interval.}
        \label{fig:average-financial-return}
    \end{minipage}
\end{figure}